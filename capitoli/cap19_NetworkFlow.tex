\chapter{Network Flow}
\label{cap:NetworkFlow}

I problemi di \textbf{Network Flow} sono una classe fondamentale di problemi che si presentano
nell'ambito del traffico in movimento in una una rete di trasporto.

Concettualmente, questa tipologia di problemi nasce nel contesto della gestione del traffico all'interno di una rete di trasporto fisica. L'intuizione di base è modellare lo spostamento di entità materiali (come veicoli, merci, pacchetti di rete) attraverso un sistema interconnesso soggetto a vincoli fisici. In questa analogia con le reti di trasporto:

\begin{itemize}
    \item \textbf{Gli archi (Edges):} rappresentano i canali di comunicazione o le strade che trasportano il traffico. Ogni arco possiede una capacità intrinseca che limita la quantità di flusso che può attraversarlo.
    \item \textbf{I nodi (Nodes):} fungono da giunzioni tra gli archi e operano come scambi (switches). Il loro compito è ricevere il flusso in entrata e ridistribuirlo sugli archi in uscita, rispettando solitamente un principio di conservazione (tutto ciò che entra deve uscire, salvo che per nodi speciali).
    \item \textbf{Il flusso (Flow):} è la quantità che deve essere spostata da specifici nodi di origine, detti \emph{sorgenti} (sources), verso nodi di destinazione, detti \emph{pozzi} (sinks).
\end{itemize}

\noindent
Sebbene l'origine intuitiva risieda nella logistica e nei trasporti, l'astrazione matematica fornita dalla teoria dei grafi permette di applicare i problemi di Network Flow a una vasta gamma di campi totalmente distinti.

\begin{itemize}
    \item \textbf{Reti di calcolatori:} instradamento dei pacchetti dati attraverso Internet massimizzando la banda passante.
    \item \textbf{Visione artificiale:} segmentazione delle immagini.
    \item \textbf{Scheduling e allocazione risorse:} assegnazione ottimale di compiti a macchine o persone.
    \item \textbf{Matching in grafi bipartiti:} accoppiamento massimo in contesti sociali o economici.
\end{itemize}


\clearpage
\section{Flow Networks}
Una \textbf{Flow Network}\footnote{
    Con \emph{Flow Network} (tradotto come \textbf{Rete di flusso}) si indica l'oggetto combinatorio statico, ovvero il grafo $G$ dotato di capacità; con \emph{Network Flow} (tradotto come \textbf{Flusso} o \textbf{Flusso su rete}) si intende invece la classe di problemi di ottimizzazione associata.
} 
(\emph{rete di flusso}) è un grafo orientato $G = (V, E)$ che rispetta dei vincoli specifici:
\begin{itemize}
    \item Ogni arco $e \in E$ è associato con un peso non negativo che rappresenta la \textbf{capacità} $c_e$ dell'arco. La capacità indica la quantità massima di flusso che può attraversare l'arco.
    \item Esiste un nodo sorgente (\textbf{source}) $s \in V$ da cui il flusso ha origine.
    \item Esiste un nodo destinazione (o pozzo) (\textbf{sink}) $t \in V$ dove il flusso termina.
    \item Tutti gli altri nodi sono detti \textbf{nodi interni}. Tutto il flusso in entrata in un nodo interno deve essere uguale al flusso in uscita da quel nodo.
\end{itemize}

\vspace{1\baselineskip}
\noindent
Per il resto del capitolo, ci basiamo su delle assunzioni che ci aiuteranno a rendere più semplice la trattazione dei problemi di Network Flow, senza perdere di generalità.
\begin{itemize}
    \item La capacità $c_e$ di ogni arco $e$ è un numero intero (non negativo).
    \item Il nodo sorgente $s$ non ha archi entranti.
    \item Il nodo destinazione $t$ non ha archi uscenti.
    \item Ciascun nodo interno ha almeno un arco incidente.
\end{itemize}

\begin{figure}[!ht]
    \centering
    \includegraphics[width=0.4\textwidth]{immagini/NetworkFlow/flow_network_ex.png}
    \caption{Esempio di Flow Network, con sorgente $s$ e destinazione $t$, e capacità associate agli archi.}
    \label{fig:flow_network_ex}
\end{figure}

\clearpage
\paragraph{Definizione di Flusso} 
Un \textbf{s-t Flow} in una Flow Network $G = (V, E)$ è una funzione
$ f: E \rightarrow R_{\geq 0} $ tale che:
\begin{itemize}
    \item \textbf{Capacity conditions:} Per ogni arco $e \in E$, $f(e)$ rappresenta la quantità di traffico trasportata dall'arco, e deve rispettare la condizione di capacità:
    \[
        0 \leq f(e) \leq c_e 
    \]
    \item \textbf{Conservation conditions:} Per ogni nodo interno $v$ (quindi, esclusi $s$ e $t$), si ha:
    \[
        \sum_{e \text{ into } v} f(e) = \sum _{e \text{ out of } v} f(e)
    \]
\end{itemize}

\noindent
Dove $\sum_{e \text{ into } v} f(e)$ somma il valore di flusso $f(e)$ su tutti gli archi che entrano nel nodo $v$, mentre $\sum _{e \text{ out of } v} f(e)$ è la somma dei valori di flusso su tutti gli archi che escono dal nodo $v$.

Dunque il flusso su un arco non può superare la capacità dell'arco stesso. Per ogni nodo diverso dalla sorgente e dalla destinazione, la quantità di flusso in entrata deve essere uguale alla quantità di flusso in uscita. La sorgente non ha archi in entrata (per la nostra assunzione), ma è permesso che abbia flusso in uscita; in altre parole, la sorgente può generare flusso. Simmetricamente, per la destinazione è permesso avere flusso in entrata, anche se non ha archi in uscita. 

Il valore di un flusso $f$, denotato $v(f)$, è definito come la quantità di flusso generata alla sorgente, ovvero la somma di tutti i flussi uscenti dalla sorgente:
\[
    v(f) = \sum_{e \text{ out of } s} f(e).
\]


\vspace{1\baselineskip}
\noindent
Per rendere la notazione più compatta, definiamo per ogni nodo $v \in V$ le seguenti quantità:
\[
    f^{in}(v) = \sum_{e \text{ into } v} f(e)
    \quad \text{e} \quad
    f^{out}(v) = \sum_{e \text{ out of } v} f(e)
\] 

\noindent
È possibile estendere questa definizione a insiemi di vertici: se $S \subseteq V$ è un insieme di nodi, definiamo: 
\[
    f^{in}(S) = \sum_{e \text{ into } S} f(e) 
    \quad \text{e} \quad 
    f^{out}(S) = \sum_{e \text{ out of } S} f(e).
\]

\noindent
In questa terminologia, la condizione di conservazione per i nodi $v \neq s, t$ diventa $f^{in}(v) = f^{out}(v)$; e possiamo scrivere $v(f) = f^{out}(s)$.

Si noti che la sorgente $s$ è l'unico nodo per cui $0 = f^{in}(s) < f^{out}(s)$, mentre la destinazione $t$ è l'unico nodo per cui $f^{in}(t) > f^{out}(t) = 0$.


\clearpage
\section{Maximum Flow Problem}




