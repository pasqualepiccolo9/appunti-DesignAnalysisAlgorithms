\chapter{Minimum Spanning Trees (MST)}
\label{cap:MST}

È utile cominciare richiamando alcune definizioni già viste nel capitolo sui grafi (capitolo \ref{cap:Graphs}).
\begin{itemize}[nosep]
    \item Una \textbf{forest} è un grafo non orientato e aciclico (ovvero privo di cicli). 
    \item Un \textbf{tree} è una forest connessa. In altre parole, un tree è un grafo non orientato, aciclico e connesso.
    \item Un \textbf{sottografo} $H$ di un grafo $G = (V,E)$ è un grafo $H = (V', E')$ tale che $V' \subseteq V$ e $E' \subseteq E$. In altre parole, un sottografo è ottenuto rimuovendo vertici e/o archi da $G$. Uno \textbf{spanning subgraph} (la traduzione corretta è "sottografo ricoprente") di un grafo $G = (V,E)$ è un sottografo $H = (V', E')$ tale che $V' = V$. In altre parole, uno spanning subgraph contiene tutti i vertici di $G$, ma potrebbe non contenere tutti gli archi di $G$. 
    \item Uno \textbf{spanning tree} di un grafo è uno spanning subgraph che è anche un tree (tree = grafo non orientato, aciclico e connesso). Si noti come uno spanning tree di un grafo con $n$ vertici \textbf{contenga esattamente} $n-1$ archi.
\end{itemize}

\vspace{1\baselineskip}
\noindent
Dato un grafo non orientato e pesato $G$, siamo interessati a trovare un tree $T$ che contenga tutti i vertici di $G$ e minimizzi la somma dei pesi degli archi in $T$. 

$$ w(T) = \sum_{(u,v) \in T} w(u,v) $$

\noindent
Un tale tree $T$ è chiamato \textbf{minimum spanning tree} (MST) di $G$.

\begin{figure}[!ht]
    \centering
    \includegraphics[width=0.9\textwidth]{immagini/MST/mst_ex.png}
    \caption{Esempio di grafo non orientato e pesato (a sinistra) e del suo Minimum Spanning Tree (a destra).}
    \label{fig:mst_ex}
\end{figure}

\clearpage
\noindent
Si tratta di un problema di ottimizzazione combinatoria molto importante, con numerose applicazioni pratiche, ad esempio nella progettazione di reti (telefoniche, elettriche, idriche, ecc.). 

Supponiamo, ad esempio, di dover connettere tutti i computer di un nuovo edificio per uffici utilizzando la minor quantità di cavo possibile. Possiamo modellare questo problema utilizzando un grafo non orientato e pesato $G$ i cui vertici rappresentano i computer, e i cui archi rappresentano tutte le possibili coppie $(u,v)$ di computer, dove il peso $w(u,v)$ dell'arco $(u,v)$ è pari alla quantità di cavo necessaria per connettere il computer $u$ al computer $v$. Piuttosto che calcolare un albero dei cammini minimi a partire da un particolare vertice $v$, siamo invece interessati a trovare un albero $T$ che contenga tutti i vertici di $G$ e che abbia il peso totale minimo tra tutti gli alberi di questo tipo. 

\paragraph{MST vs Shortest Paths:} è importante non confondere il problema del Minimum Spanning Tree con il problema degli Shortest Paths. Gli spanning tree calcolati dagli algoritmi di Shortest Paths (come Dijkstra o Bellman-Ford) dipendono dal vertice sorgente scelto, mentre il Minimum Spanning Tree è unico (a meno di casi particolari di pesi uguali) e non dipende da alcun vertice sorgente. Inoltre, gli Shortest Paths mirano a minimizzare la distanza da un singolo vertice sorgente a tutti gli altri vertici, mentre il Minimum Spanning Tree mira a minimizzare la somma totale dei pesi degli archi che connettono tutti i vertici del grafo.

\section{Proprietà di un MST - Cycle Property}
Sia $T$ un Minimum Spanning Tree di un grafo non orientato e pesato $G$, e sia $e$ un arco di $G$ non appartenente a $T$. Chiamiamo $C$ il ciclo ottenuto aggiungendo l'arco $e$ a $T$. Allora, l'arco $e$ è il più pesante tra tutti gli archi del ciclo $C$, ovvero $w(f) \leq w(e)$ per ogni arco $f \in C$.

\paragraph{Dimostrazione:} 
Supponiamo, per assurdo, che esista un arco $f \in C$ tale che $w(f) > w(e)$. Rimuovendo l'arco $f$ da $T$ e aggiungendo l'arco $e$, otteniamo un nuovo albero $T' = T - \{f\} + \{e\}$. Poiché $w(e) < w(f)$, il peso totale di $T'$ è minore di quello di $T$, ovvero $w(T') < w(T)$. Questo contraddice l'ipotesi che $T$ sia un Minimum Spanning Tree. Pertanto, l'arco $e$ deve essere il più pesante tra tutti gli archi del ciclo $C$.

\begin{figure}[!ht]
    \centering
    \includegraphics[width=1\textwidth]{immagini/MST/cycle_property.png}
    \caption{Dimostrazione della Cycle Property.}
    \label{fig:cycle_property}
\end{figure}


\clearpage
\section{Proprietà di un MST - Partitioning Property}
Sia $G$ un grafo non orientato, pesato e connesso. Sia $V_1$ e $V_2$ una partizione dei vertici di $G$ in due insiemi disgiunti e non vuoti. Inoltre, sia $e$ un arco in $G$ con peso minimo tra quelli che hanno un estremo in $V_1$ e l'altro in $V_2$ (cioè tra quelli che fanno da "ponte" tra le due partizioni). Allora, esiste un\footnote
{
    Si dice \emph{un} Minimum Spanning Tree poiché, in presenza di archi con lo stesso peso, il MST non è necessariamente unico: l'arco $e$ di peso minimo che attraversa la partizione appartiene ad almeno un MST, ma non è detto che appartenga a tutti. Se tutti gli archi hanno pesi distinti, allora il MST è unico e l'arco $e$ appartiene necessariamente a questo unico MST.
}
Minimum Spanning Tree $T$ che ha $e$ come uno dei suoi archi.

\begin{figure}[!ht]
    \centering
    \includegraphics[width=1\textwidth]{immagini/MST/partitioning_property.png}
    \caption{Dimostrazione della Partitioning Property.}
    \label{fig:partitioning_property}
\end{figure}

\paragraph{Dimostrazione:}
Sia $T$ un Minimum Spanning Tree di $G$. Se $e \in T$, la tesi è verificata.
Supponiamo quindi che $e \notin T$. Poiché $T$ è uno spanning tree, esiste un unico cammino in $T$ che collega gli estremi dell'arco $e$. L'aggiunta di $e$ a $T$ genera dunque un ciclo $C$.

Poiché l'arco $e$ ha un estremo in $V_1$ e l'altro in $V_2$, nel ciclo $C$ deve esistere almeno un arco $f \neq e$ appartenente a $T$ che ha un estremo in $V_1$ e l'altro in $V_2$. Per la definizione di $e$, si ha
\[
w(e) \leq w(f).
\]

\noindent
Consideriamo ora il grafo $T'$ ottenuto rimpiazzando l'arco $f$ con l'arco $e$ in $T$:
\[
T' = (T \cup \{e\}) - \{f\}.
\]

\noindent
Il grafo $T'$ è connesso, aciclico e copre tutti i vertici di $G$, quindi è uno spanning tree. 
Inoltre:
\[
w(T') = w(T) + w(e) - w(f) \leq w(T).
\]

\noindent
Poiché $T$ è un Minimum Spanning Tree, segue che $w(T') = w(T)$ e quindi anche $T'$ è un Minimum Spanning Tree. Per costruzione, $T'$ contiene l'arco $e$.

Pertanto, esiste un Minimum Spanning Tree di $G$ che contiene l'arco $e$.


\clearpage















\clearpage


% parallelismo tra cliente e azienda: il cliente vuole che il prodotto arrivi nel minor tempo possibile (shortest path), l'azienda vuole minimizzare i costi di produzione e distribuzione (minimum spanning tree).

% Prim: simile a dijkstra, cambia il modo in cui eseguiamo il rilassamento degli archi.