\chapter{Transitive Closure}
\label{cap:TransitiveClosure}

Come abbiamo visto, gli attraversamenti dei grafi possono essere utilizzati per rispondere a domande di base sulla raggiungibilità in un grafo orientato. In particolare, se siamo interessati a sapere se esiste un percorso dal vertice $u$ al vertice $v$ in un grafo, possiamo eseguire una \emph{DFS} o \emph{BFS} a partire da $u$ e osservare se $v$ viene scoperto. Se rappresentiamo un grafo con una \emph{Adjacency list} o \emph{Adjacency map}, possiamo rispondere alla domanda di raggiungibilità da $u$ a $v$ in tempo $O(n+m)$. 

In alcune applicazioni, potremmo voler rispondere a molte query di raggiungibilità in modo più efficiente, e dunque potrebbe valere la pena precomputare una rappresentazione più conveniente di un grafo. Ad esempio, il primo passo per un servizio che calcola le indicazioni stradali da un'origine a una destinazione potrebbe essere quello di valutare se la destinazione è raggiungibile. Allo stesso modo, in una rete elettrica, potremmo voler essere in grado di determinare rapidamente se la corrente fluisce da un particolare vertice a un altro. Motivati da tali applicazioni, introduciamo la seguente definizione. 

\paragraph{Definizione:} La \textbf{chiusura transitiva} (transitive closure) di un \emph{grafo orientato} $\vec{G}=(V,E)$ è a sua volta un \emph{grafo orientato} $\vec{G}^*=(V^*,E^*)$ tale che:
\begin{itemize}[nosep]
    \item $V^* = V$
    \item $(u,v) \in E^{*}$ se e solo se $v$ è raggiungibile da $u$ in $\vec{G}$.
\end{itemize}

\noindent
Dunque, la chiusura transitiva fornisce informazioni di raggiungibilità tra tutti i vertici in un grafo orientato, e in alcune applicazioni è più efficiente calcolare $\vec{G}^*$ piuttosto che eseguire un attraversamento (DFS o BFS) da ciascun vertice.


\begin{figure}[!ht]
    \centering
    \includegraphics[width=0.66\textwidth]{immagini/TransitiveClosure/ex_TransitiveClosure.png}
    \caption{Esempio di grafo orientato (a sinistra) e la sua chiusura transitiva (a destra).}
    \label{fig:ex_TransitiveClosure}
\end{figure}



\clearpage
\noindent
Se un grafo è rappresentato come una Adjacency List o Adjacency Map, possiamo calcolare la sua chiusura transitiva in tempo $O(n(n+m))$ facendo uso di $n$ attraversamenti del grafo, uno da ciascun vertice di partenza. Ad esempio, una DFS che inizia al vertice $u$ può essere utilizzata per determinare tutti i vertici raggiungibili da $u$, e quindi una collezione di archi che originano da $u$ nella chiusura transitiva.

\section{Algoritmo di Floyd-Warshall per la Chiusura Transitiva}
Un'alternativa efficace per calcolare la \emph{chiusura transitiva} di un grafo orientato sfrutta strutture che supportano la ricerca in tempo $O(1)$ per il metodo \texttt{get\_edge(u,v)}, come ad esempio una \emph{Adjacency Matrix}.

Si tratta di un algoritmo di programmazione dinamica, l'algoritmo di \textbf{Floyd-Warshall}. L'idea alla base di questo algoritmo è quella di costruire una funzione booleana $G(i,j,k) = 1$ (vero) se esiste un percorso da $v_i$ a $v_j$ che utilizza solo i vertici tra $v_1, v_2, \ldots, v_k$ come vertici intermedi, $G(i,j,k) = 0$ (falso) altrimenti. 

\begin{itemize}
    \item La relazione di raggiungibilità soddisfa le seguenti condizioni:
    \begin{itemize}[nosep]
        \item $G(i,j,k) = 1$ \textbf{if} $G(i,j,k-1) = 1$;
        \item $G(i,j,k) = 1$ \textbf{if}
        $G(i,k,k-1) = 1$ \textbf{AND} $G(k,j,k-1) = 1$;
        \item $G(i,j,k) = 0$ \textbf{otherwise}.
    \end{itemize}

    \item \textbf{Caso base} ($k = 0$):
    \[
        G(i,j,0) =
        \begin{cases}
            1 & \text{if } (v_i,v_j) \in E, \\
            0 & \text{otherwise}.
        \end{cases}
    \]

    \item \textbf{Equazione caratteristica} ($k \geq 1$):
    \[
        G(i,j,k) =
        G(i,j,k-1)
        \;\textbf{OR}\;
        \big( G(i,k,k-1) \;\textbf{AND}\; G(k,j,k-1) \big).
    \]
\end{itemize}

\noindent
L'algoritmo di Floyd-Warshall, descritto tramite la funzione booleana $G(i,j,k)$, va interpretato come un procedimento incrementale che costruisce gradualmente tutti i cammini possibili nel grafo.

Il caso base indica che, senza alcun vertice intermedio, un cammino tra $i$ e $j$ esiste solo se c'è un arco diretto, fornendo il punto di partenza per cammini più complessi.

La regola ricorsiva mostra come, ad ogni passo, aggiungere un nuovo vertice $v_k$ come possibile intermedio: un cammino tra $i$ e $j$ esiste se esiste già senza usare $v_k$, oppure se si possono concatenare due cammini più piccoli, da $i$ a $v_k$ e da $v_k$ a $j$, che non utilizzano $v_k$ come intermedio. In altre parole, ogni nuovo vertice intermedio permette di combinare cammini già noti per crearne di nuovi, senza introdurre ambiguità o cicli.

Considerando tutti i vertici come potenziali intermedi uno alla volta, alla fine $G(i,j,n)$ rappresenta tutti i cammini possibili tra ogni coppia di vertici, cioè la chiusura transitiva del grafo.


\clearpage
\noindent
Per comprendere meglio l'algoritmo, è utile presentare anche una versione alternativa basata sull'approccio del grafo incrementale, che realizza la stessa logica della costruzione booleana ma opera direttamente sugli archi e mostra in maniera più intuitiva come i nuovi percorsi vengono aggiunti passo dopo passo.

\vspace{1\baselineskip}
\noindent
Sia $\vec{G}$ un grafo orientato con $n$ vertici e $m$ archi. L'idea è quella di calcolare la chiusura transitiva di $\vec{G}$ in una serie di round successivi.

\begin{enumerate}
    \item Inizializziamo $\vec{G}_0 = \vec{G}$.
    \item Numeriamo arbitrariamente i vertici di $\vec{G}$ come $v_1,v_2, \ldots, v_n$.
    \item Per ogni round $k = 1, \dots, n$:
    \begin{itemize}[nosep]
        \item Un arco $(v_i,v_j)$ appartiene a $\vec{G}_k$ se e solo se: $(v_i,v_j)$ era già presente in $\vec{G}_{k-1}$ oppure $(v_i,v_k)$ e $(v_k,v_j)$ appartengono a $\vec{G}_{k-1}$.
        \item Costruiamo $\vec{G}_k$ copiando $\vec{G}_{k-1}$ e aggiungendo tutti gli archi $(v_i,v_j)$ che soddisfano la condizione precedente.
    \end{itemize}
    \item Dopo aver completato il round $n$, otteniamo $\vec{G}_n = \vec{G}^*$; restituiamo $\vec{G}_n$ come chiusura transitiva di $\vec{G}$.
\end{enumerate} 

\noindent
Questo approccio incrementale è facilmente implementabile seguendo lo pseudocodice riportato di seguito.

\vspace{1\baselineskip}
\hrule
\begin{verbatim}
Input: A directed graph G with n vertices
Output: The transitive closure G^* of G
1.  Algorithm FloydWarshall(G)
2.       i = 1
3.       for all v in G.vertices()       
4.           denote v as v_i              
5.           i = i + 1
6.       G_0 = G                          
7.       for k = 1 to n do                // k e' il vertice intermedio
8.           G_k = G_{k-1}                // Copia stato precedente
9.           for i = 1 to n (i != k) do   // Nodo sorgente
10.             for j = 1 to n (j != i, k) do // Nodo destinazione
11.                 if G_{k-1}.areAdjacent(v_i, v_k) AND 
12.                    G_{k-1}.areAdjacent(v_k, v_j)     
13.                    if NOT G_{k-1}.areAdjacent(v_i, v_j) 
14.                        G_k.insertDirectedEdge(v_i, v_j, k) 
15.      return G_n                       
\end{verbatim}
\hrule 
\vspace{1\baselineskip}

\clearpage
\subsection{Analisi e implementazione dell'algoritmo}
Dallo pseudocodice, possiamo facilmente analizzare il tempo di esecuzione dell'algoritmo di Floyd-Warshall assumendo che la struttura dati che rappresenta $G$ supporti i metodi \texttt{get\_edge} e \texttt{insert\_edge} in tempo $O(1)$ (\emph{come Adjacency Matrix}). Il ciclo principale viene eseguito $n$ volte e il ciclo interno considera ciascuna delle $O(n^2)$ coppie di vertici, eseguendo un calcolo a tempo costante per ciascuna. Dunque, il tempo totale di esecuzione dell'algoritmo di Floyd-Warshall è $O(n^3)$ (IMPORTANTE: valido date le assunzioni fatte sulla struttura dati).

\vspace{1\baselineskip}
\begin{lstlisting}[
    language=Python,
    caption={Implementazione Python dell'algoritmo di Floyd-Warshall.},
    captionpos=b,
    label={lst:Floyd-Warshall},
]
def floyd_warshall(g):
    """Return a new graph that is the transitive closure of g."""
    closure = deepcopy(g)               # imported from copy module
    verts = list(closure.vertices( ))   # make indexable list
    n = len(verts)
    for k in range(n):
        for i in range(n):
            # verify that edge (i,k) exists in the partial closure
            if i != k and closure.get_edge(verts[i],verts[k]) is not None:
                for j in range(n):
                    # verify that edge (k,j) exists in the partial closure
                    if i != j != k and closure.get_edge(verts[k],verts[j]) is not None:
                        # if (i,j) not yet included, add it to the closure
                        if closure.get_edge(verts[i],verts[j]) is None:
                            closure.insert_edge(verts[i],verts[j])
    return closure
\end{lstlisting}
\vspace{1\baselineskip}
Asintoticamente, il tempo di esecuzione $O(n^3)$ dell'algoritmo di Floyd-Warshall non è migliore di quello ottenuto eseguendo ripetutamente una DFS, una volta da ciascun vertice, per calcolare la raggiungibilità (tempo $O(n(n+m))$). Tuttavia, l'algoritmo di Floyd-Warshall eguaglia i limiti asintotici della DFS ripetuta quando un grafo è denso, o quando un grafo è sparso ma rappresentato come una Adjacency Matrix.

L'importanza dell'algoritmo di Floyd-Warshall risiede nel fatto che è molto più semplice da implementare rispetto alla DFS, e molto più veloce nella pratica perché ci sono relativamente poche operazioni di basso livello nascoste nella notazione asintotica. L'algoritmo è particolarmente adatto per l'uso di una Adjacency Matrix, poiché un singolo bit può essere utilizzato per designare la raggiungibilità modellata come un arco $(u,v)$ nella chiusura transitiva. 

Tuttavia, si noti che chiamate ripetute a DFS portano a prestazioni asintotiche migliori quando il grafo è sparso e rappresentato utilizzando una Adjacency List o Adjacency Map. In tal caso, una singola DFS viene eseguita in tempo $O(n+m)$, e quindi la chiusura transitiva può essere calcolata in tempo $O(n^2 + nm)$, che è preferibile a $O(n^3)$.

\clearpage
INSERISCI IMMAGINE ESEMPIO QUI
Screen già pronto

caption=Sequence of directed graphs computed by the Floyd-Warshall algorithm:
(a) initial directed graph G = G0 and numbering of the vertices; (b) directed
graph G1; (c) G2; (d) G3; (e) G4; (f) G5. Note that G5 = G6 = G7. If directed
graph Gk−1 has the edges (vi,vk) and (vk,vj), but not the edge (vi,vj), in the drawing
of directed graph Gk, we show edges (vi,vk) and (vk,vj) with dashed lines, and
edge (vi,vj) with a thick line. For example, in (b) existing edges (MIA,LAX) and
(LAX,ORD) result in new edge (MIA,ORD).

\clearpage
\section{Directed Acyclic Graph - DAG}



