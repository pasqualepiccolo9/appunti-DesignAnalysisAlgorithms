\chapter{Graphs Traversal}
\label{cap:GraphsTraversal}

Un \textbf{graph traversal algorithm} (algoritmo di attraversamento dei grafi) è una procedura sistematica per esplorare un grafo esaminando tutti i suoi vertici e archi. Un algoritmo di graph traversal è efficiente se visita tutti i vertici e archi in un tempo proporzionale al loro numero, cioè in tempo lineare.

Gli algoritmi di graph traversal sono fondamentali per rispondere a molte domande fondamentali sui grafi che coinvolgono la nozione di \emph{raggiungibilità}, cioè nel determinare come viaggiare da un vertice all'altro seguendo i percorsi di un grafo. 

Gli algoritmi di graph traversal più utilizzati sono la \textbf{depth-first search} (DFS, ricerca in profondità) e la \textbf{breadth-first search} (BFS, ricerca in ampiezza). Entrambi gli algoritmi iniziano da un vertice di partenza e visitano sistematicamente tutti i vertici raggiungibili da quel vertice, ma lo fanno in modi diversi come vedremo più avanti.


\vspace{1\baselineskip}
\noindent
Alcuni problemi interessanti che trattano la raggiungibilità in un grafo non orientato $G$ includono i seguenti:
\begin{itemize}
    \item Calcolare un percorso dal vertice $u$ al vertice $v$, o segnalare che non esiste tale percorso.
    \item Dato un vertice di partenza $s$ di $G$, calcolare, per ogni vertice $v$ di $G$, un percorso con il numero minimo di archi tra $s$ e $v$, o segnalare che non esiste tale percorso.
    \item Verificare se $G$ è connesso.
    \item Calcolare un albero di copertura (spanning tree) di $G$, se $G$ è connesso.
\end{itemize}

\noindent
Alcuni riguardano in modo specifico i grafi orientati.
\begin{itemize}
    \item Calcolare un percorso diretto dal vertice $u$ al vertice $v$, o segnalare che non esiste tale percorso.
    \item Trovare tutti i vertici di $G$ raggiungibili da un dato vertice $s$.
    \item Determinare se $G$ è aciclico.
    \item Determinare se $G$ è fortemente connesso.
\end{itemize}

\clearpage
\section{Depth-First Search (DFS)}
\label{sec:DFS}

La \textbf{Depth-First Search} (DFS) è una tecnica generale per l'attraversamento dei grafi che esplora "in profondità": parte da un nodo radice e avanza lungo ogni ramo il più possibile prima di tornare indietro (backtracking). 

L'algoritmo DFS applicato ad un grafo $G$ con $n$ vertici e $m$ archi visita tutti i vertici e archi del grafo in tempo $O(n + m)$.
In particolare, la DFS può essere usata per verificare se $G$ è connesso, calcolare le \emph{connected components} \footnote
{Una \textbf{connected component} (componente connessa) è un sottografo massimale di un \emph{grafo non orientato} in cui ogni coppia di vertici è collegata da almeno un cammino (sottografo massimale vuol dire che non può essere esteso aggiungendo altri vertici del grafo mantenendo la proprietà per cui tutti i vertici sono connessi tra loro).} 
di $G$ e calcolare uno \emph{spanning forest}\footnote
{Uno \textbf{spanning tree} è un sottoinsieme di archi che connette tutti i vertici di un grafo connesso senza cicli.
Uno \textbf{spanning forest} è un insieme di spanning tree, uno per ciascuna componente connessa di un grafo non connesso.
L'algoritmo DFS genera uno spanning tree se il grafo è connesso; se il grafo non è connesso, eseguita su tutte le componenti, genera uno spanning forest.
} 
di $G$.


\begin{itemize}
    \item \textbf{DFS su un singolo vertice in un grafo non connesso:} visita solamente i vertici della \emph{connected component} contenente il vertice di partenza e costruisce uno \emph{spanning tree} di tale componente. Non permette di visitare tutti i vertici del grafo, né di verificare la connettività globale o calcolare tutte le connected components.
    
    \item \textbf{DFS su tutti i vertici non ancora visitati in un grafo non connesso:} eseguendo DFS partendo da ciascun vertice non visitato si visitano tutti i vertici del grafo. Questo procedimento permette di calcolare tutte le \emph{connected components} e di costruire uno \emph{spanning forest} del grafo. 
    
    \item \textbf{DFS su un singolo vertice in un grafo connesso:} in questo caso l'esecuzione di DFS sul vertice di partenza visita tutti i vertici del grafo, costruendo uno \emph{spanning tree} dell'intero grafo e permettendo di verificare la connettività globale.
\end{itemize}

\vspace{1\baselineskip}
\noindent
L'idea di base dell'algoritmo DFS è la seguente:
\begin{itemize}
    \item DFS inizia da un vertice sorgente $s$ e marca $s$ come visitato.
    \item Quando DFS arriva in un vertice $u$ non ancora visitato:
    \begin{itemize}[nosep]
        \item Se esiste un arco $(u, v)$ che conduce a un vertice $v$ non ancora visitato, DFS prende tale arco, marca $v$ come visitato e applica ricorsivamente lo stesso algoritmo su $v$.
        \item Se non ci sono archi che conducono a vertici non ancora visitati, DFS torna indietro lungo il percorso fino a raggiungere un vertice che ha dei vicini non ancora visitati.
        \end{itemize}
    \item Quando una chiamata DFS termina:
    \begin{itemize}[nosep]
        \item Se ci sono vertici non ancora visitati, se ne sceglie uno e si ripete il processo. Ogni chiamata DFS completa esplora una singola \emph{connected component}.
        \item Se tutti i vertici sono stati visitati, l'algoritmo termina.
    \end{itemize}
\end{itemize}


\clearpage
\noindent
Per implementare DFS abbiamo bisogno di etichettare i vertici e gli archi del grafo:

\begin{itemize}
    \item Ogni vertice può essere in uno dei seguenti stati:
    \begin{itemize}[nosep]
        \item \textbf{Unexplored}: il vertice non è stato ancora visitato.
        \item \textbf{Visited}: il vertice è stato già visitato.
    \end{itemize}
    \item Ogni arco può essere in uno dei seguenti stati:
    \begin{itemize}[nosep]
        \item \textbf{Unexplored}: l'arco non è stato ancora esplorato.
        \item \textbf{Discovery}: l'arco è stato esplorato per raggiungere un vertice non ancora visitato. Questo arco fa parte dello spanning tree.
        \item \textbf{Back edge}: l'arco è stato esplorato e conduce a un vertice già visitato. Questo arco non fa parte dello spanning tree.
    \end{itemize}
\end{itemize}



\vspace{1\baselineskip}
\hrule
\begin{alltt}
Input: graph \(G\)
Output: labeling of the edges of \(G\) as discovery or back
1.  Algorithm DFS(\(G\)):
2.      \textbf{for all} \(u\) in \(G\).\text{vertices}()
3.          \text{setLabel}(\(u\), UNEXPLORED)
4.      \textbf{for all} \(e\) in \(G\).\text{edges}()
5.          \text{setLabel}(\(e\), UNEXPLORED)
6.      \textbf{for all} \(v\) in \(G\).\text{vertices}()
7.          \textbf{if} \text{getLabel}(\(v\)) = UNEXPLORED
8.              DFS(\(G, v\))
\end{alltt}
\hrule 

\vspace{1\baselineskip}

\hrule
\begin{alltt}
Input: graph \(G\) and a source vertex \(v\)
Output: labels for the edges in the connected component 
 of \(G\) containing \(v\)
1.  Algorithm DFS(\(G, v\)):
2.    \text{setLabel}(\(v\), VISITED)
3.    \textbf{for all} \(e\) in \(G\).\text{incidentEdges}(\(v\))
4.        \textbf{if} \text{getLabel}(\(e\)) = UNEXPLORED
5.            \(w = \text{opposite}(v, e)\)
6.            \textbf{if} \text{getLabel}(\(w\)) = UNEXPLORED
7.                \text{setLabel}(\(e\), DISCOVERY)
8.                DFS(\(G, w\))
9.            \textbf{else}
10.               \text{setLabel}(\(e\), BACK)
\end{alltt}
\hrule 
\vspace{1\baselineskip}

\clearpage
\begin{figure}[!ht]
    \centering
    \includegraphics[width=0.97\textwidth]{immagini/GraphsTraversal/DFS_ex.png}
    \caption{Esempio di DFS su un grafo non orientato a partire dal vertice A. Supponiamo che le adiacenze di un vertice siano considerate in ordine alfabetico. I vertici visitati e gli archi esplorati sono evidenziati, con gli archi di scoperta (discovery) disegnati come linee solide e gli archi non appartenenti all'albero (back edges) come linee tratteggiate; i vertici e gli archi non evidenziati sono tutti unexplored: (a) grafo di input; (b) percorso degli archi dell'albero, tracciato da A fino a quando viene esaminato l'arco di back (G,C); (c) raggiungimento di F, che è un vicolo cieco; (d) dopo il backtracking a I, riprendendo con l'arco (I,M), e raggiungendo un altro vicolo cieco in O; (e) dopo il backtracking a G, continuando con l'arco (G,L), e raggiungendo un altro vicolo cieco in H; (f) risultato finale.}
    \label{fig:DFS_ex}
\end{figure}

\clearpage
\noindent
Riportiamo alcune proprietà fondamentali dell'algoritmo DFS:
\begin{enumerate}
    \item Sia $G$ un \textbf{grafo non orientato} su cui è stata eseguita una DFS a partire da un vertice $s$. Allora la visita DFS copre tutti i vertici della componente connessa di $s$, e gli archi di discovery formano uno spanning tree della componente connessa di $s$.
    \begin{itemize}
        \item Calcolare uno spanning tree della componente connessa di $s$ vuol dire trovare un cammino da $s$ a ogni altro vertice raggiungibile da $s$.
    \end{itemize}
    \item Sia $G$ un \textbf{grafo orientato}. La DFS su $G$ a partire da un vertice $s$ visita tutti i vertici di $G$ raggiungibili da $s$. Inoltre, l'albero DFS contiene percorsi orientati da $s$ a ogni vertice raggiungibile da $s$.
\end{enumerate}


\subsection{Analisi e implementazione dell'algoritmo DFS}
Nell'algoritmo di visita in profondità (Depth First Search, DFS) ogni vertice del grafo viene etichettato esattamente due volte: inizialmente come \emph{UNEXPLORED}, e successivamente come \emph{VISITED} nel momento in cui viene effettivamente visitato dall'algoritmo. Analogamente, ogni arco viene etichettato due volte: la prima volta come \emph{UNEXPLORED} quando viene incontrato per la prima volta, e la seconda volta come \emph{DISCOVERY} oppure \emph{BACK}, a seconda che conduca rispettivamente a un vertice non ancora visitato oppure a un vertice già visitato. Le operazioni di lettura e scrittura delle etichette associate a vertici e archi richiedono tempo costante \(O(1)\). Inoltre, il metodo \texttt{incidentEdges(v)}, che restituisce tutti gli archi incidenti a un vertice \(v\), viene invocato una sola volta per ciascun vertice del grafo. Se il grafo è rappresentato mediante \emph{Adjacency List} o \emph{Adjacency Map}, il tempo necessario per eseguire \texttt{incidentEdges(v)} è proporzionale al grado del vertice \(v\), cioè \(O(\deg(v))\). Sommando tale costo su tutti i vertici del grafo, si ottiene \(\sum_{v} \deg(v) = 2m\), dove \(m\) è il numero di archi del grafo. Ne consegue che il tempo totale impiegato dall'algoritmo DFS è \(O(n + m)\), dove \(n\) è il numero di vertici e \(m\) il numero di archi.

Per completezza, se invece il grafo è rappresentato mediante una \emph{edge list}, per determinare gli archi incidenti a un vertice è necessario scandire l'intera lista degli archi, operazione che richiede tempo \(O(m)\) per ciascun vertice; poiché tale operazione viene eseguita per tutti i \(n\) vertici, il tempo totale dell'algoritmo diventa \(O(n \cdot m)\). Infine, nel caso di una rappresentazione tramite \emph{Adjacency Matrix} , il metodo \texttt{incidentEdges(v)} richiede la scansione dell'intera riga (o colonna) associata al vertice \(v\), con costo \(O(n)\); essendo tale operazione eseguita per ogni vertice, il tempo complessivo della DFS risulta \(O(n^2)\), indipendentemente dal numero di archi presenti nel grafo.

\clearpage
\begin{lstlisting}[
    language=Python,
    caption={Implementazione Python dell'algoritmo DFS.},
    captionpos=b,
    label={lst:DFS},
]
def DFS_complete(g):
    """Makes a DFS on the graph g and returns a dictionary where each vertex is mapped to the edge used to discover it (all the roots of the DFS trees are mapped to None)."""
    forest = {}
    for u in g.vertices():
        if u not in forest:
            forest[u] = None    # u is the root of the tree
            DFS(g, u, forest)
    return forest

def DFS(g, u, discovered):
    """Makes a DFS traversal from the vertex u that has been already visited (discovered is a dictionary mapping each visited vertex with the edge used to discover it). When new vertices are discovered they are added to the dictionary."""
    for e in g.incident_edges(u):   # for each edge outgoing from u
        v = e.opposite(u)
        if v not in discovered:     # if v is not visited
            discovered[v] = e       # edge e is added to the dictionary
            DFS(g, v, discovered)   # runs a DFS recursively from v
\end{lstlisting}


\subsection{DFS su grafi orientati}
L'algoritmo di DFS funziona per grafi orientati percorrendo ogni arco solo nella sua direzione.
\begin{itemize}
    \item In una DFS orientata distinguiamo quattro tipi di archi:
    \begin{itemize}[nosep]
        \item \textbf{discovery edges} (archi di scoperta): sono gli archi $(u, v)$ che portano a un vertice $v$ non ancora visitato. Questi archi formano la foresta (o l'albero) di esplorazione DFS.
        \item \textbf{back edges} (archi di ritorno): sono gli archi $(u, v)$ che collegano un vertice $u$ a un suo antenato\footnote{
            La visita DFS impone una struttura gerarchica "orientata" (un albero radicato) anche su grafi non orientati. Quando la visita passa da un vertice $v$ a un vertice non visitato $u$, si stabilisce una relazione diretta in cui $v$ è padre e $u$ è figlio. Questo orienta logicamente l'arco di scoperta (discovery edge) da $v$ verso $u$. In questo contesto: Un vertice $v$ è definito \textbf{antenato} di $u$ se esiste un percorso composto esclusivamente da discovery edges che collega $v$ a $u$ (seguendo la direzione padre $\to$ figlio). Viceversa, $u$ è definito \textbf{discendente} di $v$ se si trova nel sotto-albero che ha come radice $v$.
        } 
        $v$ nell'albero DFS. La presenza di un arco di ritorno durante l'esecuzione della DFS indica l'esistenza di almeno un ciclo nel grafo.
        \item \textbf{forward edges} (archi in avanti): sono gli archi $(u, v)$ che collegano un vertice $u$ a un suo discendente $v$ nell'albero DFS (che non sia però un figlio diretto, altrimenti sarebbe classificato come \textit{discovery edge}).
        \item \textbf{cross edges} (archi trasversali): sono tutti gli altri archi. Collegano un vertice $u$ a un vertice $v$ tale che $v$ non è né antenato né discendente di $u$. Possono collegare due rami diversi dello stesso albero DFS oppure due alberi diversi all'interno della foresta DFS.
    \end{itemize}
\end{itemize}


\clearpage
\subsection{Estensioni della DFS}
Attraverso l'utilizzo del \textit{Template Method Pattern}, è possibile specializzare l'algoritmo di DFS per risolvere una vasta gamma di problemi sui grafi mantenendo una complessità temporale efficiente di $O(n + m)$. 

Il \textit{Template Method} è un pattern che definisce lo scheletro di un algoritmo in un metodo base, delegando l'implementazione di specifici passaggi (o ``ganci'', hook) alle sottoclassi. Nel contesto della DFS, questo significa che la logica di attraversamento ricorsivo rimane invariata, ma vengono esposte delle funzioni personalizzabili che vengono invocate in momenti chiave dell'esecuzione (ad esempio: quando un vertice viene visitato per la prima volta, quando si attraversa un arco, o quando si termina la visita di un vertice). Sovrascrivendo questi metodi specifici, possiamo adattare la DFS generica per ottenere le seguenti informazioni:

\begin{itemize}[nosep]
    \item Calcolo delle \textbf{componenti connesse} di un grafo;
    \item Costruzione di una \textbf{spanning forest} del grafo;
    \item Individuazione di un \textbf{cammino} tra due vertici (se esistente);
    \item Rilevamento di un \textbf{ciclo} nel grafo (se esistente);
    \item Identificazione di tutti i vertici \textbf{raggiungibili} da un dato vertice in un grafo orientato;
    \item Calcolo delle \textbf{componenti fortemente connesse} in un grafo orientato.
\end{itemize}

\vspace{1\baselineskip}
\subsubsection{Connectivity Test}
Un grafo è connesso se ogni vertice è raggiungibile da tutti gli altri vertici del grafo. Possiamo quindi utilizzare DFS per verificare se $G$ è connesso:
\begin{itemize}[nosep]
    \item Eseguiamo una DFS a partire da un vertice arbitrario $v$;
    \item Se la DFS visita tutti i vertici del grafo (len(discovered) == n) allora $G$ è connesso.
\end{itemize}

\subsubsection{Strong Connectivity Test}
Possiamo utilizzare DFS per verificare se un grafo orientato è \emph{fortemente connesso}, cioè se ogni vertice è raggiungibile da tutti gli altri vertici attraverso cammini orientati.
\begin{itemize}[nosep]
    \item Eseguiamo una DFS a partire da un vertice arbitrario $v$;
    \item Se la DFS non visita tutti i vertici del grafo, allora $G$ non è fortemente connesso;
    \item Altrimenti, costruiamo il grafo trasposto $G'$ invertendo la direzione di tutti gli archi di $G$;
    \item Eseguiamo una DFS su $G'$ a partire da $v$;
    \item Se la DFS non visita tutti i vertici di $G'$, allora $G$ non è fortemente connesso; altrimenti, $G$ è fortemente connesso.
\end{itemize}

\noindent
Possiamo ragionare allo stesso modo per verificare se una \emph{componente connessa} di un grafo orientato è \emph{fortemente connessa} (\textbf{Strongly Connected Components}).

Una componente connessa di un grafo orientato è fortemente connessa se ogni vertice della componente è raggiungibile da tutti gli altri vertici della componente attraverso cammini orientati.


\clearpage
\subsubsection{Searching Paths}
Possiamo utilizzare DFS per trovare un cammino tra due vertici $v$ e $z$:
\begin{itemize}[nosep]
    \item Eseguiamo una DFS a partire da $v$;
    \item Utilizziamo uno stack $S$ per tenere traccia del cammino dal vertice di partenza $v$ al vertice corrente (e quindi degli archi di discovery attraversati);
    \item Quando raggiungiamo $z$, restituiamo tutti i vertici nello stack.
\end{itemize}

\vspace{1\baselineskip}
\hrule
\begin{alltt}
1.  Algorithm pathDFS(\(G, v, z\)):
2.      \text{setLabel}(\(v\), VISITED)    // Segna \(v\) come visitato
3.      \(S\).\text{push}(\(v\))               // Aggiunge \(v\) al percorso corrente
4.      \textbf{if} \(v = z\)               // Caso base: destinazione raggiunta
5.          \textbf{return} \(S\).\text{elements}() // Restituisce il percorso trovato
6.      \textbf{for all} \(e\) in \(G\).\text{incidentEdges}(\(v\))
7.          \textbf{if} \text{getLabel}(\(e\)) = UNEXPLORED
8.              \(w = \text{opposite}(v, e)\)
9.              \textbf{if} \text{getLabel}(\(w\)) = UNEXPLORED
10.                 \text{setLabel}(\(e\), DISCOVERY)
11.                 \(S\).\text{push}(\(e\))   // Aggiunge l'arco al percorso corrente
12.                 pathDFS(\(G, w, z\))
13.                 \(S\).\text{pop}(\(e\))    // Backtracking: strada fallimentare
14.             \textbf{else}
15.                 \text{setLabel}(\(e\), BACK)
16.     \(S\).\text{pop}(\(v\))                // Backtracking: vicolo cieco
\end{alltt}
\hrule 
\vspace{1\baselineskip}


\begin{lstlisting}[
    language=Python,
    caption={Implementazione Python della ricerca di un path tra due vertici. Nota: discovered è il dizionario restituito dalla DFS eseguita a partire da u (e non la DFS\_complete(g)).},
    captionpos=b,
    label={lst:DFS_Path},
]
def construct_path(u, v, discovered):
    """Return a list of vertices forming a path from u to v (the list is empty if v is not reachable from u). Discovered is the dictionary returned by the DFS."""
    path = []               # when we start the path is empty
    if v in discovered:     # if v is in the connected component of u
        # we build backward the path from v to u
        path.append(v)
        walk = v
        while walk is not u:
            e = discovered[walk]    # find the edge used to discover walk
            parent = e.opposite(walk)
            path.append(parent)     # add to the path the vertex opposite to walk
            walk = parent
        path.reverse()              # invert the path from u to v
    return path
\end{lstlisting}

\clearpage
\subsubsection{Searching Cycles}
Possiamo utilizzare l'algoritmo DFS per verificare l'esistenza di un ciclo semplice all'interno della componente connessa del grafo a partire da un vertice $v$:
\begin{itemize}[nosep]
    \item Eseguiamo una DFS a partire dal vertice $v$;
    \item Utilizziamo uno stack $S$ per tenere traccia del cammino attivo dal vertice di partenza al vertice corrente;
    \item Se durante l'esplorazione incontriamo un arco che punta a un vertice $w$ già visitato (e presente nello stack), abbiamo individuato un arco all'indietro (\emph{back edge});
    \item Questo implica l'esistenza di un ciclo, che può essere ricostruito estraendo gli elementi dallo stack fino a ritrovare $w$.
\end{itemize}

\vspace{1\baselineskip}
\hrule
\begin{alltt}
1.  Algorithm cycleDFS(\(G, v\)):
2.      \text{setLabel}(\(v\), VISITED)    // Marca \(v\) come visitato
3.      \(S\).\text{push}(\(v\))               // Aggiunge \(v\) al percorso corrente
4.      \textbf{for all} \(e\) in \(G\).\text{incidentEdges}(\(v\))
5.          \textbf{if} \text{getLabel}(\(e\)) = UNEXPLORED
6.              \(w = \text{opposite}(v, e)\)
7.              \(S\).\text{push}(\(e\))       // Aggiunge l'arco corrente allo stack
8.              \textbf{if} \text{getLabel}(\(w\)) = UNEXPLORED
9.                  \text{setLabel}(\(e\), DISCOVERY)
10.                 cycleDFS(\(G, w\)) // Chiamata ricorsiva
11.                 \(S\).\text{pop}(\(e\))    // Backtracking: nessun ciclo qui
12.             \textbf{else}            // \(w\) è già visitato: CICLO TROVATO
13.                 \(T\) = new empty stack
14.                 \textbf{repeat}  // Travasa lo stack per isolare il ciclo
15.                     \(o = S\).\text{pop}()
16.                     \(T\).\text{push}(\(o\))
17.                 \textbf{until} \(o = w\) 
18.                 \textbf{return} \(T\).\text{elements}() // Restituisce la sequenza
19.     \(S\).\text{pop}(\(v\))        // Backtracking: rimuove \(v\) (vicolo cieco)
\end{alltt}
\hrule 
\vspace{1\baselineskip}




\clearpage
\section{Breadth-First Search (BFS)}
\label{sec:BFS}

La \textbf{Breadth-First Search} (BFS) è una tecnica generale per l'attraversamento dei grafi che esplora "in ampiezza": parte da un nodo radice e visita tutti i suoi vicini prima di procedere ai nodi di livello successivo.

L'algoritmo BFS applicato ad un grafo $G$ con $n$ vertici e $m$ archi visita tutti i vertici e archi del grafo in tempo $O(n + m)$.
In particolare, la BFS può essere usata, come DFS, per verificare se $G$ è connesso, calcolare le \emph{connected components} di $G$ e calcolare uno \emph{spanning forest} di $G$. 

Una differenza sostanziale tra BFS e DFS è che l'algoritmo BFS può essere utilizzato per calcolare i percorsi più brevi (in termini di numero di archi) tra due vertici, se questo esiste. Come abbiamo visto in precedenza, DFS è in grado di trovare un percorso tra due vertici, ma non garantisce che questo sia il più breve.

\vspace{1\baselineskip}
\noindent
L'idea di base dell'algoritmo BFS è la seguente:
\begin{itemize}
    \item La visita procede per step e divide i vertici del grafico in livelli:
    \begin{itemize}[nosep]
        \item Comincia dal vertice sorgente $s$ che è l'unico vertice al livello 0, e marca $s$ come visitato.
        \item Al passo $1$ si marcano come visitati tutti i vicini di $s$ e li si aggiunge al livello $1$.
        \item Al passo $2$ si marcano come visitati tutti i vertici adiacenti ai vertici al livello 1 che non sono stati ancora visitati e li si aggiunge al livello $2$.
        \item Al passo $i$ si marcano come visitati tutti i vertici adiacenti ai vertici al livello $i-1$ che non sono stati ancora visitati e li si aggiunge al livello $i$.
        \item Il processo si ferma quando non ci sono più vertici non visitati adiacenti a vertici visitati.
    \end{itemize}
\end{itemize}

\noindent
Per implementare BFS abbiamo bisogno di etichettare i vertici e gli archi del grafo, come visto per DFS, ma con una differenza nei tipi di etichette utilizzate (Back edge DFS vs Cross edge BFS):

\begin{itemize}
    \item Ogni vertice può essere in uno dei seguenti stati:
    \begin{itemize}[nosep]
        \item \textbf{Unexplored}: il vertice non è stato ancora visitato.
        \item \textbf{Visited}: il vertice è stato già visitato.
    \end{itemize}
    \item Ogni arco può essere in uno dei seguenti stati:
    \begin{itemize}[nosep]
        \item \textbf{Unexplored}: l'arco non è stato ancora esplorato.
        \item \textbf{Discovery}: l'arco è stato esplorato per raggiungere un vertice non ancora visitato. Questo arco fa parte dello spanning tree.
        \item \textbf{Cross edge}: l'arco è stato esplorato per raggiungere un vertice già visitato. Questo arco non fa parte dello spanning tree. 
    \end{itemize}
\end{itemize}

\clearpage
\hrule
\begin{alltt}
Input: graph \(G\)
Output: labeling of the edges and partition of the vertices of \(G\)
1.  Algorithm BFS(\(G\))
2.      \textbf{for all} \(u\) in \(G\).\text{vertices}()
3.          \text{setLabel}(\(u\), UNEXPLORED)
4.      \textbf{for all} \(e\) in \(G\).\text{edges}()
5.          \text{setLabel}(\(e\), UNEXPLORED)
6.      \textbf{for all} \(v\) in \(G\).\text{vertices}()
7.          \textbf{if} \text{getLabel}(\(v\)) = UNEXPLORED
8.              BFS(\(G, v\))
\end{alltt}
\hrule 
\vspace{1\baselineskip}

\hrule
\begin{alltt}
Input: graph \(G\) and a source vertex \(s\)
Output: labeling of the edges and vertices in the connected component 
 of \(G\) containing \(s\)
1.  Algorithm BFS(\(G, s\))
2.      \(L\sb{0}\) = new empty sequence
3.      \(L\sb{0}\).\text{addLast}(\(s\))
4.      \text{setLabel}(\(s\), VISITED)
5.      \(i = 0\)
6.      \textbf{while} not \(L\sb{i}\).\text{isEmpty}()
7.          \(L\sb{i+1}\) = new empty sequence
8.          \textbf{for all} \(v\) in \(L\sb{i}\).\text{elements}()
9.          |    \textbf{for all} \(e\) in \(G\).\text{incidentEdges}(\(v\))
10.         |         \textbf{if} \text{getLabel}(\(e\)) = UNEXPLORED
11.         |             \(w = \text{opposite}(v, e)\)
12.         |             \textbf{if} \text{getLabel}(\(w\)) = UNEXPLORED
13.         |                 \text{setLabel}(\(e\), DISCOVERY)
14.         |                 \text{setLabel}(\(w\), VISITED)
15.         |                 \(L\sb{i+1}\).\text{addLast}(\(w\))
16.         |             \textbf{else}
17.         |                 \text{setLabel}(\(e\), CROSS)
18.          \(i = i + 1\)   // move to the next level
\end{alltt}
\hrule 
\vspace{1\baselineskip}

\clearpage
\begin{figure}[!ht]
    \centering
    \includegraphics[width=1\textwidth]{immagini/GraphsTraversal/BFS_ex.png}
    \caption{Esempio di BFS su un grafo non orientato a partire dal vertice A. Supponiamo che le adiacenze di un vertice siano considerate in ordine alfabetico. I vertici visitati e gli archi esplorati sono evidenziati, con gli archi di scoperta (discovery) disegnati come linee solide e gli archi non appartenenti all'albero (cross edges) come linee tratteggiate; i vertici e gli archi non evidenziati sono tutti unexplored: (a) inizio della ricerca in A; (b) scoperta del livello 1; (c) scoperta del livello 2; (d) scoperta del livello 3; (e) scoperta del livello 4; (f) scoperta del livello 5.}
    \label{fig:BFS_ex}
\end{figure}

\clearpage
\noindent
Riportiamo alcune proprietà fondamentali dell'algoritmo BFS. Sia $G$ un grafo (orientato o non orientato) su cui è stata eseguita una BFS a partire da un vertice $s$, BFS(G, s). Allora:
\begin{enumerate}
    \item La visita BFS(G, s) visita tutti gli archi e i vertici raggiungibili da $s$, ovvero tutti gli archi e i vertici della componente connessa di $s$, $G_s$.
    \item Gli archi di discovery della visita BFS(G, s) formano uno spanning tree $T_s$ di $G_s$, chiamato \textbf{BFS tree} (albero BFS) radicato in $s$.
    \item Per ogni vertice $v$ al livello $i$, il cammino nell'albero BFS $T_s$ tra $s$ e $v$ ha $i$ archi, e ogni altro cammino in $G$ da $s$ a $v$ ha almeno $i$ archi.
    \item Se $(u,v)$ è un arco che non appartiene all'albero BFS, allora il livello del vertice $v$ può essere, al massimo, di $1$ maggiore del livello del vertice $u$.
\end{enumerate}


\subsection{Analisi e implementazione dell'algoritmo BFS}
Nell'algoritmo di visita in ampiezza (Breadth First Search, BFS) ogni vertice del grafo viene etichettato esattamente due volte: inizialmente come \emph{UNEXPLORED}, e successivamente come \emph{VISITED}. Una proprietà fondamentale della procedura è che ogni vertice viene inserito in una e una sola sequenza di livello. Analogamente, ogni arco viene etichettato due volte: la prima volta come \emph{UNEXPLORED}, e la seconda volta come \emph{DISCOVERY} oppure \emph{CROSS}, a seconda della relazione tra i vertici collegati. Le operazioni di lettura e scrittura delle etichette associate a vertici e archi richiedono tempo costante $O(1)$. Inoltre, il metodo \texttt{incidentEdges(v)}, che restituisce tutti gli archi incidenti a un vertice $v$, viene invocato una sola volta per ciascun vertice del grafo. Se il grafo è rappresentato mediante \emph{Adjacency List} o \emph{Adjacency Map}, il tempo necessario per eseguire \texttt{incidentEdges(v)} è proporzionale al grado del vertice $v$, cioè $O(\deg(v))$. Sommando tale costo su tutti i vertici del grafo, si ottiene $\sum_{v} \deg(v) = 2m$, dove $m$ è il numero di archi del grafo. Ne consegue che il tempo totale impiegato dall'algoritmo BFS è $O(n + m)$, dove $n$ è il numero di vertici e $m$ il numero di archi.

Per completezza, se invece il grafo è rappresentato mediante una \emph{edge list}, per determinare gli archi incidenti a un vertice è necessario scandire l'intera lista degli archi, operazione che richiede tempo $O(m)$ per ciascun vertice; poiché tale operazione viene eseguita per tutti i $n$ vertici, il tempo totale dell'algoritmo diventa $O(n \cdot m)$. Infine, nel caso di una rappresentazione tramite \emph{Adjacency Matrix}, il metodo \texttt{incidentEdges(v)} richiede la scansione dell'intera riga (o colonna) associata al vertice $v$, con costo $O(n)$; essendo tale operazione eseguita per ogni vertice, il tempo complessivo della BFS risulta $O(n^2)$, indipendentemente dal numero di archi presenti nel grafo.

\clearpage
\begin{lstlisting}[
    language=Python,
    caption={Implementazione Python dell'algoritmo BFS.},
    captionpos=b,
    label={lst:BFS},
]
def BFS_complete(g):
    """ Makes a BFS on the graph g and returns adictionary where each vertex is mapped to the edgeused to discover it (all the roots of the BFS treesare mapped to None)"""
    forest = {}
    for u in g.vertices():
        if u not in forest:
            forest[u] = None  # u is the root of the tree
            BFS(g, u, forest)
    return forest

def BFS(g, s, discovered):
    """ Makes a BFS traversal from the vertex s that has been already visited (discovered is a dictionary mapping each visited vertex with the edge used to discover it). When new vertices are discovered they are added to the dictionary."""
    level = [s]  # level 0 includes only s
    while len(level) > 0:
        next_level = []  # list of the vertices in the next level
        for u in level:  # for each vertex u in the level
            for e in g.incident_edges(u):  # for each edge outgoing from u
                v = e.opposite(u)
                if v not in discovered:  # if v has not been visited yet
                    discovered[v] = e  # add e to the dictionary
                    next_level.append(v)  # add v to the next level
        level = next_level
\end{lstlisting}

\subsection{Estensioni della BFS}
Così come visto per DFS, attraverso l'utilizzo del \textit{Template Method Pattern}, è possibile specializzare l'algoritmo di BFS per risolvere una vasta gamma di problemi sui grafi mantenendo una complessità temporale efficiente di $O(n + m)$. 

\begin{itemize}
    \item Calcolo delle \textbf{componenti connesse} di un grafo;
    \item Costruzione di una \textbf{spanning forest} del grafo;
    \item Rilevamento di un \textbf{ciclo} nel grafo tra due vertici (se esistente);
    \item Calcolo di un \textbf{percorso con il numero minimo di archi} tra due vertici (se esistente);
    \begin{itemize}[nosep]
        \item Non è possibile con DFS.
    \end{itemize}
\end{itemize}





