% --- PREAMBOLO ---

% Definisce il tipo di documento (articolo, report, book, letter, ecc.)
% a4paper imposta la pagina in formato A4, 12pt la dimensione del font
\documentclass[a4paper, 11pt]{book}

% --- PACCHETTI MATEMATICI FONDAMENTALI ---
% NOTA: amsmath deve essere caricato PRIMA di newtxmath per evitare conflitti
\usepackage{amsmath}

% --- Font Times New Roman e lingua---
\usepackage[T1]{fontenc}
\usepackage[utf8]{inputenc}
\usepackage[italian]{babel}
\usepackage[normalem]{ulem}

\usepackage{newtxtext,newtxmath}
\usepackage{microtype}

% --- STILE DELL'INDICE ---
\makeatletter
% capitolo, livello 0: indentazione 0em, larghezza numero 1.5em
\renewcommand*\l@chapter{\@dottedtocline{0}{0em}{1.5em}}
% sezione, livello 1: indentazione 1.5em, larghezza numero 2.3em
\renewcommand*\l@section{\@dottedtocline{1}{1.5em}{2.3em}}
% sottosezione, livello 2: indentazione 3.8em, larghezza numero 3.2em
\renewcommand*\l@subsection{\@dottedtocline{2}{3.8em}{3.2em}}
\makeatother



% rinomina "Indice" in "Sommario"
\addto\captionsitalian{
  \renewcommand{\contentsname}{Sommario}
}



% --- Impaginazione ---
\usepackage[margin=3cm,bindingoffset=0.5cm]{geometry}
\usepackage{setspace}
\setstretch{1.2}
\setlength{\parindent}{1cm}

% --- Figure, tabelle ---
\usepackage{graphicx}
\usepackage{multirow}
\graphicspath{{./}{./immagini/}} % cerca immagini anche in ./immagini
\usepackage{float}
\usepackage{subcaption}
\usepackage{tabularx,booktabs}
\usepackage{enumitem}


% --- PACCHETTI NECESSARI ---
\usepackage{xcolor}      % gestione dei colori
\usepackage{listings}    % ambiente per codice

% --- COLORI TEMA DRACULA ---
\definecolor{dracula-bg}{HTML}{282A36}       % sfondo
\definecolor{dracula-current}{HTML}{44475A}  % riga corrente / cornice
\definecolor{dracula-fg}{HTML}{F8F8F2}       % testo base
\definecolor{dracula-comment}{HTML}{6272A4}  % commenti
\definecolor{dracula-cyan}{HTML}{8BE9FD}     % variabili / funzioni
\definecolor{dracula-green}{HTML}{50FA7B}    % valori booleani
\definecolor{dracula-orange}{HTML}{FFB86C}   % numeri
\definecolor{dracula-pink}{HTML}{FF79C6}     % keyword
\definecolor{dracula-purple}{HTML}{BD93F9}   % operatori opzionali
\definecolor{dracula-red}{HTML}{FF5555}      % errori / alert
\definecolor{dracula-yellow}{HTML}{F1FA8C}   % stringhe

% --- DEFINIZIONE STILE LISTINGS DRACULA ---
\lstdefinestyle{dracula}{
    language=Python,                             % linguaggio
    backgroundcolor=\color{dracula-bg},          % sfondo
    basicstyle=\ttfamily\footnotesize\color{dracula-fg}, % testo base
    keywordstyle=\color{dracula-pink}\bfseries,  % keyword
    commentstyle=\color{dracula-comment}\itshape, % commenti in corsivo
    stringstyle=\color{dracula-yellow},          % stringhe
    identifierstyle=\color{dracula-cyan},       % nomi di variabili / funzioni
    numbers=left,                                % numeri a sinistra
    numberstyle=\tiny\color{black},     % stile dei numeri di riga
    stepnumber=1,                                % numerazione ogni riga
    numbersep=5pt,                               % distanza numeri dal codice
    showstringspaces=false,                      % non mostra spazi nelle stringhe
    frame=single,                                % cornice semplice
    rulecolor=\color{dracula-current},          % colore della cornice
    breaklines=true,                             % va a capo automaticamente
    breakatwhitespace=true,                      % preferenza per andare a capo agli spazi
    tabsize=4                                    % tab = 4 spazi
}

% --- USO DELLO STILE ---
\lstset{style=dracula,
literate={à}{{\`a}}1 {è}{{\`e}}1 {é}{{\'e}}1 {ì}{{\`i}}1 {ò}{{\`o}}1 {ù}{{\`u}}1}  % applica lo stile Dracula a tutti i blocchi \begin{lstlisting}




% --- Bibliografia ---
\usepackage[backend=biber,sorting=none,style=numeric]{biblatex}
\addbibresource{bib.bib}
\usepackage{csquotes}

% --- Numerazione e intestazioni ---
\usepackage{fancyhdr}
\pagestyle{fancy}
\fancyhf{}
\fancyfoot[C]{\thepage}
\renewcommand{\headrulewidth}{0pt}
\renewcommand{\footrulewidth}{0pt}
\fancypagestyle{plain}{
  \fancyhf{}
  \fancyfoot[C]{\thepage}
  \renewcommand{\headrulewidth}{0pt}
  \renewcommand{\footrulewidth}{0pt}
}

\usepackage[bottom]{footmisc}
\setlength{\skip\footins}{8pt plus 2pt minus 1pt}

% --- Link e riferimenti ---
\usepackage[hidelinks]{hyperref}
\usepackage{cleveref}

% --- INFORMAZIONI SUL DOCUMENTO ---
\title{Alberi}
\author{Carmine}
\date{\today} % \today inserisce la data di compilazione. Puoi anche scrivere una data fissa, es. "Ottobre 2025"



% --- CORPO DEL DOCUMENTO ---
\begin{document}

% ================= FRONTESPIZIO =================
\begin{titlepage}
    \begin{center}
        \huge{\uppercase{Università degli Studi di Salerno}}\\[10mm]
        \uppercase{\normalsize DIPARTIMENTO DI INGEGNERIA DELL'INFORMAZIONE ED ELETTRICA\\ E MATEMATICA APPLICATA}\\[15mm]
        \normalsize{Corso: \\DESIGN AND ANALYSIS OF ALGORITHMS}\\[20mm]
        \includegraphics[width=0.35\textwidth]{immagini/logo_unisa}\\[20mm]
        \textbf{\large APPUNTI}\\[10mm]
    \end{center}

    \begin{center}
        \Large{\textbf{Carmine Terracciano}}\\[2mm]
        \large{Mat. IE22700109}\\[10mm]
    \end{center}

    \vfill
    \begin{center}
        \large \uppercase{Anno Accademico 2025/2026}
    \end{center}
\end{titlepage}

% Crea un sommario/indice
% =================================================================
% --- SOMMARIO INIZIA A SINISTRA ---
%
% 1. Salva il comando \cleardoublepage originale in \oldcleardoublepage
\let\oldcleardoublepage\cleardoublepage
%
% 2. Fai in modo che \cleardoublepage si comporti come \clearpage
%    (che va solo alla pagina *successiva*, non alla pagina *destra successiva*)
\let\cleardoublepage\clearpage

% Ora \tableofcontents userà la nostra versione "modificata" di \cleardoublepage
% e inizierà felicemente sulla pagina 2 (sinistra)
\tableofcontents

% 3. Ripristina il comando \cleardoublepage originale
%    così tutti i capitoli successivi (\input{capitoli...})
%    torneranno a iniziare sulla pagina destra, come da regola "book".
\let\cleardoublepage\oldcleardoublepage
%
% --- FINE ---
% =================================================================


% --- INIZIO DEL CONTENUTO ---
\cleardoublepage
\input{capitoli/cap1_BST}

\cleardoublepage
\input{capitoli/cap2_AVL}

\cleardoublepage
\chapter{Multi-Way Search Tree (MWST)}
\label{cap:MWST}

\paragraph{Definizione:} Un \textbf{Multi-Way Search Tree (MWST)} è un albero ordinato in cui:
\begin{itemize}
    \item Ogni nodo dell'albero ($d$-nodo) ha $d \ge 2$ figli e contiene $d-1$ elementi chiave valore $(k_i, v_i)$ ordinati in modo crescente per chiave.
    \item Sia $w$ un nodo con figli $w_1, w_2, \ldots, w_d$ e con chiavi $k_1, k_2, \ldots, k_{d-1}$. Allora:
    \begin{itemize}
        \item Tutte le chiavi nel sottoalbero radicato in $w_1$ sono minori di $k_1$.
        \item Tutte le chiavi nel sottoalbero radicato in $w_i$ sono comprese tra $k_{i-1}$ e $k_i$ ($i = 2, \ldots, d-1$).
        \item Tutte le chiavi nel sottoalbero radicato in $w_d$ sono maggiori di $k_{d-1}$.
    \end{itemize}
    \item I nodi foglia (None) non contengono elementi.
\end{itemize}

Si osservi come un MWST contenente $n$ elementi abbia $n+1$ nodi foglia (None).

\begin{figure}[ht!]
    \centering
    \includegraphics[width=0.8\textwidth]{immagini/MWST/MWST_inorder.png}
    \label{fig:MWST_inorder}
\end{figure}

\begin{figure}[ht!]
    \centering
    \includegraphics[width=0.8\textwidth]{immagini/MWST/MWST_search.png}
    \label{fig:MWST_search}
\end{figure}

\begin{figure}[ht!]
    \centering
    \includegraphics[width=0.8\textwidth]{immagini/MWST/MWST_exSearch.png}
    \label{fig:MWST_exSearch}
\end{figure}



\section{(a, b)-Tree}
\paragraph{Definizione:} Gli \textbf{(a, b)-Tree} sono dei \emph{MWST} che soddisfano le seguenti proprietà:
\begin{itemize}
    \item $2 \le a \le \left\lceil \frac{b-1}{2} \right\rceil$
    \item \textbf{Root Property:} La radice ha almeno $2$ figli e al più $b$ figli.
    \item \textbf{Node-Size Property:} Ogni nodo diverso dalla radice ha almeno $a$ figli e al più $b$ figli.
    \item \textbf{Depth Property:} Tutti i nodi foglia (None) sono allo stesso livello.
\end{itemize}


\clearpage
\begin{figure}[ht!]
    \centering
    \includegraphics[width=1\textwidth]{immagini/MWST/ab_example.png}
    \caption{Esempio di un (a, b)-Tree con $a=3$ e $b=6$.}
    \label{fig:ab_example}
\end{figure}

\subsection*{Altezza di un (a, b)-Tree e Ricerca di un elemento}
L'altezza di un (a, b)-Tree con $n$ elementi è: 
\begin{itemize}
    \item $\Omega(\log n / \log b)$ [== $\Omega(\log_b n)$].
    \item $O(\log n / \log a)$ [== $O(\log_a n)$].
\end{itemize}

\noindent
Per ogni nodo, è necessario eseguire una ricerca tra tutti i suoi elementi. Un nodo è una \emph{Map} di al più $b-1$ elementi, per cui possiamo denotare con $f(b)$ il costo della ricerca in un nodo. Se l'elemento non è stato trovato è necessario scendere in uno dei figli, per cui il costo totale della ricerca in un (a, b)-Tree è: $$O(f(b) \cdot \log n / \log a)$$
Se $f(b)$ è costante, il costo della ricerca è migliore di $O(\log n)$.


\clearpage
\section{Insert}
Per inserire un elemento ($k, v$) in un (a, b)-Tree, si procede come segue:
\begin{enumerate}
    \item Si esegue una ricerca per trovare la posizione corretta in cui inserire l'elemento: supponendo che l'albero non presenti già un elemento con chiave $k$, la ricerca termina senza successo restituendo il nodo foglia (None) $z$. Sia $w$ il genitore di $z$, inseriamo il nuovo elemento in $w$ e aggiungiamo una nuova foglia (None) $y$ come figlio di $w$.
    \item A questo punto, se $w$ ha meno di $b-1$ elementi (cioè ha meno di $b$ figli), l'inserimento è terminato.
    \item Altrimenti, se $w$ ha già $b-1$ elementi (cioè $b$ figli), si verifica un \textbf{overflow} e dobbiamo eseguire un \textbf{split} di $w$:
    \begin{itemize}
        \item Consideriamo il nodo $w$ con i suoi $b-1$ elementi, più il nuovo elemento appena inserito.
        \item Si esegue uno \textbf{split} di $w$ in tre parti: gli elementi minori di $k'$, l'elemento mediano $k'$ e gli elementi maggiori di $k'$.
        \item Si crea un nuovo nodo che conterrà gli elementi minori di $k'$. (Questo nodo è automaticamente valido perchè contiene almeno $\lceil (b-1)/2 \rceil -1 \ge a-1$ elementi e sicuramente meno di $b-1$ elementi).
        \item Vale lo stesso per il nodo che conterrà gli elementi maggiori di $k'$.
        \item L'elemento mediano $k'$ viene promosso al genitore $p$ di $w$ e diventa un nuovo elemento di $p$, con i due nodi risultanti dallo split come suoi figli.
        \begin{itemize}
            \item Se $p$ ha ora $b$ elementi, si ripete la procedura di split su $p$.
            \item Altrimenti, l'inserimento è terminato.
            \item Se il nodo splittato era la radice, si crea una nuova radice che contiene solo l'elemento mediano $m$ e ha come figli i due nodi risultanti dallo split.
        \end{itemize}
    \end{itemize}
\end{enumerate}

\begin{figure}[ht!]
    \centering
    \includegraphics[width=1\textwidth]{immagini/MWST/insert.png}
    \caption{Esempio di inserimento con overflow in un (a, b)-Tree con $a=2$ e $b=4$. (a) overflow in un 5-nodo $w$ [il massimo è un 4-nodo]; (b) e (c) split di $w$.}
    \label{fig:insert}
\end{figure}

\clearpage
\section{Delete}
Per eliminare un elemento ($k, v$) in un (a, b)-Tree, si procede come segue:
\begin{enumerate}
    \item Si esegue una ricerca per trovare il nodo $w$ contenente l'elemento da eliminare $k$. Se il nodo non esiste, l'operazione termina.
    \item Se il figlio a sinistra o a destra di $k$ non è vuoto, oppure $w$ è la radice, allora:
    \begin{itemize}
        \item Si trova il predecessore o il successore di $k$ (cioè l'elemento più grande del sottoalbero sinistro o l'elemento più piccolo del sottoalbero destro) e lo indichiamo con $k'$.
        \item Si scambia l'elemento con chiave $k$ con l'elemento con chiave $k'$.
        \item Si elimina l'elemento con chiave $k'$ dal sottoalbero (l'elemento originario con chiave $k'$).
    \end{itemize}
    \item Altrimenti, se entrambi i figli di $k$ sono vuoti, e:
    \begin{itemize}
        \item il nodo $w$ ha $ > a-1 $ elementi, eliminiamo l'elemento con chiave $k$ da $w$ e l'operazione termina.
        \item il nodo $w$ ha esattamente il numero minimo di $a-1$ elementi, si verifica un \textbf{underflow}.
    \end{itemize}
    \item Gestione dell'underflow:
    \begin{itemize}
        \item Se il nodo $x$ fratello sinistro di $w$ (con lo stesso genitore $p$) ha più di $a-1$ elementi, si esegue un \textbf{transfer}:
        \begin{itemize}
            \item Sia $k'$ la chiave salvata nel genitore $p$ "che sta tra il puntatore a $x$ e il puntatore a $w$".
            \item Sia $k''$ la chiave più grande nel nodo $x$(ricordiamo che $x$ è il fratello a sinistra di $w$).
            \item Cancella $k$ da $w$, cancella $k''$ da $x$ e sostituisci $k'$ con $k''$ in $p$, e aggiungi $k'$ in $w$.
        \end{itemize}
        \item Se il nodo $x$ fratello destro di $w$ (con lo stesso genitore $p$) ha più di $a-1$ elementi, si esegue un \textbf{transfer}:
        \begin{itemize}
            \item L'opposto del caso precedente: si prende la chiave più piccola dal fratello destro $x$ e la si sposta in $w$, aggiornando di conseguenza il genitore $p$.
        \end{itemize}
        \item Altrimenti, se entrambi i fratelli di $w$ hanno esattamente $a-1$ elementi, si esegue una \textbf{fusion}:
        \begin{itemize}
            \item Sia $x$ un fratello di $w$ (a sinistra o a       destra) con lo stesso genitore $p$.
            \item Sia $k'$ la chiave salvata nel genitore $p$ "che     sta tra il puntatore a $x$ e il puntatore a $w$".
            \item Si crea un nuovo nodo che contiene tutti gli     elementi di $w$ eccetto l'elemento con chiave $k$,     tutti gli elementi di $x$ e l'elemento con chiave $k'$.
            \begin{itemize}
                \item È un nodo valido perchè contiene un numero di        elementi pari a: $$a-1 \le (a-2)+(a-1)+1 = 2a-2 \le        b-1$$
            \end{itemize}
            \item Si elimina $k'$ da $p$.
            \begin{itemize}
                \item Se $p$ è la radice e $k'$ è il suo unico     elemento, il nuovo nodo creato diventa la radice.
            \end{itemize}
        \end{itemize}
    \end{itemize}
\end{enumerate}

\begin{figure}[ht!]
    \centering
    \includegraphics[width=1\textwidth]{immagini/MWST/delete.png}
    \caption{Una sequenza di rimozioni in un (a, b)-Tree con $a=2$ e $b=4$. (a) rimozione di 4, che causa un underflow; (b) un'operazione di transfer; (c) dopo l'operazione di transfer; (d) rimozione di 12, che causa un underflow; (e) un'operazione di fusion; (f ) dopo l'operazione di fusion; (g) rimozione di 13; (h) dopo la rimozione di 13.}
    \label{fig:delete}
\end{figure}

\subsection*{Complessità delle operazioni di inserimento e cancellazione}
\noindent
Si tenga presente che la ricerca di un nodo, come visto prima, impiega un tempo $O(f(b) \cdot \log n / \log a)$. Supponendo che la gestione di un overflow/underflow richieda al più $g(b)$ [$g(b)$ dipende dall'implementazione del nodo], e considerando che potrebbe essere necessario ripetere al più tali operazioni dal livello $h-1$ fino alla radice. Quindi, il costo totale dell'inserimento in un (a, b)-Tree è: 
$$O\left(\left(f(b) + g(b)\right) \log n / \log a\right)$$

\subsection*{Come scegliere a e b?}
Se i nodi sono troppo piccoli, l'albero sarà essenzialmente simile ad un albero binario di ricerca bilanciato. Nonostante ciò, vedremo che i (2, 4)-Tree sono particolarmente importanti perchè possono essere trasformati in degli alberi particolari chiamati \textbf{Red-Black Tree}.

\vspace{1\baselineskip}
\noindent
Se i nodi sono troppo grandi, l'albero diventa meno efficiente in termini di spazio e di tempo per le operazioni di ricerca, inserimento e cancellazione. Dipende da come è implementata la Map all'interno del nodo.
\begin{itemize}
    \item Se si tratta di Hash Tables, la ricerca e gli aggiornamenti sono $O(1)$, ma il calcolo della mediana è $O(b)$.
    \item Se si tratta di BST bilanciati, la ricerca e gli aggiornamenti sono $O(\log b)$ e il calcolo della mediana è $O(b)$.
    \item Se si tratta di vettori ordinati, la ricerca è $O(\log b)$, gli aggiornamenti sono $O(b)$ e il calcolo della mediana è $O(1)$.
\end{itemize}
Idealmente, vorremmo che tutte queste operazioni vadano come $O(1)$, ma ciò non è possibile. In pratica, si cerca di minimizzare $b$ in modo che le operazioni siano efficienti, ma abbastanza grande da ridurre l'altezza dell'albero.

\vspace{1\baselineskip}
\noindent
Se $a$ e $b$ sono troppo distanti tra loro, la complessità delle operazioni interne al nodo cancellano il vantaggio dato dalla riduzione dell'altezza [$f(b)$ e $g(b)$ crescono troppo rapidamente fino a diventare di gran lunga più pesanti di $\log a$]. 



\section{B-Tree}
Un \textbf{B-Tree} è un (a, b)-Tree con $a = \lceil (b-1)/2 \rceil$ e $b = d$.

\noindent
Per le considerazioni fatte sugli (a, b)-Tree, si ha che:
\begin{itemize}
    \item La ricerca ha una complessità di $O(f(d) \cdot \log n / (\log d - 1))$.
    \item L'inserimento e la cancellazione hanno una complessità di $O(g(d) \cdot \log n / (\log d - 1))$.
\end{itemize}


\subsection*{I/O Complexity}
Consideriamo il problema della gestione di una grande raccolta di elementi che non rientrano nella memoria principale, come un tipico database. In questo contesto, ci riferiamo ai blocchi di memoria secondaria come \emph{disk blocks}. Allo stesso modo, ci riferiamo al trasferimento di un blocco tra la memoria secondaria e la memoria primaria come \emph{disk transfer}. Ricordando la grande differenza di tempo che esiste tra gli accessi alla memoria principale e gli accessi al disco, l'obiettivo principale della gestione di tale raccolta nella memoria esterna è quello di ridurre al minimo il numero di trasferimenti su disco necessari per eseguire una query o un aggiornamento. Ci riferiamo a questo numero come \textbf{I/O complexity} dell'algoritmo coinvolto.

\vspace{1\baselineskip}
\noindent
Il modo migliore di ridurre al minimo il numero di trasferimenti su disco è quello di massimizzare il numero di elementi che possono essere memorizzati in un singolo blocco. Supponiamo che ogni blocco di disco possa contenere $B$ elementi. Nel caso di un B-Tree, scegliamo il suo \emph{ordine} $d$ (il numero massimo di figli per nodo) il più grande possibile, in modo tale che un nodo -- contenente $O(d)$ elementi e puntatori -- occupi al massimo un singolo blocco di disco.
%
Si ottiene così una relazione diretta tra $d$ e $B$, ovvero $d = \Theta(B)$.
%
Di conseguenza, ogni accesso a un nodo del B-Tree corrisponde a un singolo trasferimento su disco. Scegliendo $d$ così grande, si massimizza il fattore di diramazione (branching factor) e si minimizza l'altezza dell'albero.
%
Poiché l'altezza di un B-Tree con $n$ elementi è $O(\log_d n)$, e dato che $d = \Theta(B)$, la \textbf{I/O complexity} per la \textbf{ricerca} è $O(\log_B n)$.
%
Anche l'inserimento e la cancellazione mantengono questa complessità, poiché, oltre alla ricerca iniziale, richiedono un numero di operazioni di modifica (come divisioni o fusioni di nodi) proporzionale all'altezza dell'albero.
%
In conclusione, i B-Tree sono una struttura dati molto efficiente per la gestione di grandi raccolte di elementi nella memoria esterna.



\cleardoublepage
\chapter{Red-Black Trees e (2, 4) Trees}
\label{cap:RBT}

Abbiamo già parlato di alberi bilanciati nel Capitolo \ref{cap:AVL}, in particolare degli alberi AVL. Questo tipo di alberi consentiva di mantenere l'altezza dell'albero logaritmica rispetto al numero di nodi presenti, garantendo così operazioni di ricerca, inserimento e cancellazione efficienti. Tuttavia, una cancellazione in un albero AVL poteva richiedere molte rotazioni per mantenere l'equilibrio, rendendo l'operazione più costosa in termini di tempo.

\section{(2, 4) Trees}
I (2, 4) Trees sono semplicemente degli (a, b) Trees con $a = 2$ e $b = 4$. Ciò significa che ereditano tutte le proprietà degli (a, b) Trees discusse nel Capitolo \ref{cap:MWST}, e dal punto di vista di un'analisi asintotica, le performance di un (2, 4) Tree sono equivalenti a quelle di un albero AVL.
Per un (2, 4) Tree:
\begin{itemize}
    \item L'altezza è $O(\log n)$.
    \item Le operazioni di split, transfer e fusion hanno una complessità di $O(1)$.
    \item Le operazioni di ricerca, inserimento e cancellazione hanno una complessità di $O(\log n)$.
\end{itemize}

\noindent
All'interno di un (2, 4) Tree, ogni nodo può contenere da 1 a 3 chiavi e può avere da 2 a 4 figli. Questo consente di distinguere tra 2-nodi, 3-nodi e 4-nodi, sulla base del numero di figli (numero di chiavi + 1).
Questo tipo di alberi è particolarmente interessante per la sua relazione che ha con un tipo particolare di alberi, i \textbf{Red-Black Trees}.



\clearpage
\section{Red-Black Trees}
\paragraph{Definizione:} Un Red-Black Tree è un albero binario di ricerca in cui ogni nodo ha un colore: rosso o nero, e l'abero soddisfa le seguenti proprietà:
\begin{itemize}
    \item \textbf{Root Property:} La radice deve essere nera.
    \item \textbf{External Property:} Tutti i nodi foglia (None) sono neri.
    \item \textbf{Internal Property:} I figli di un nodo rosso sono neri.
    \item \textbf{Depth Property:} Tutti i nodi foglia (None) hanno la stessa \emph{black-depth}, ovvero il numero di antenati neri.
\end{itemize}

\begin{figure}[ht!]
    \centering
    \includegraphics[width=0.8\textwidth]{immagini/RBT/RBT_example.png}
    \caption{Esempio di Red-Black Tree, con i nodi rossi disegnati in bianco. La black-depth di ogni nodo foglia è 3. Da notare che non sono disegnati i nodi foglia (None), che sono tutti neri.}
    \label{fig:RBT_example}
\end{figure}

Come abbiamo detto, i Red-Black Trees sono strettamente correlati ai (2, 4) Trees. Infatti, ogni (2, 4) Tree può essere rappresentato come un Red-Black Tree e viceversa. Un Red-Black Tree può essere visto come una rappresentazione binaria di un (2, 4) Tree, dove i nodi rossi rappresentano i nodi con più di una chiave nel (2, 4) Tree. Proprio per questo motivo, i Red-Black Trees mantengono le stesse performance dei (2, 4)-Trees, con il beneficio aggiuntivo di una implementazione più semplice e di una maggiore efficienza nelle operazioni di inserimento e cancellazione, che richiedono al massimo una o due rotazioni per mantenere l'equilibrio dell'albero.


\clearpage
\section{Dal (2, 4)-Tree al Red-Black Tree}

\begin{itemize}
    \item Colora di nero tutti i nodi del (2, 4) Tree.
    \item Per ogni nodo $w$:
    \begin{itemize}
        \item Se $w$ è un 2-nodo, mantieni i figli (neri) di $w$ così come sono.
        \item Se $w$ è un 3-nodo, crea un nuovo nodo rosso $y$, figlio destro (o sinistro) di $w$, e fai in modo che gli ultimi due (o i primi due) figli di $w$ diventino figli di $y$, e il primo figlio (o l'ultimo) di $w$ rimanga figlio di $w$.
        \item Se $w$ è un 4-nodo con chiavi $k_1$, $k_2$ e $k_3$, rappresentalo come un nodo nero con due figli rossi contenenti le chiavi $k_2$ e $k_3$.
    \end{itemize}
\end{itemize}

\noindent
Da notare che seguendo l'algoritmo sopra descritto, un nodo rosso avrà sempre un genitore nero.

\begin{figure}[ht!]
    \centering
    \includegraphics[width=1\textwidth]{immagini/RBT/24toRB.png}
    \label{fig:24toRB}
\end{figure}



\section{Dal Red-Black Tree al (2, 4)-Tree}

\begin{itemize}
    \item  Ogni nodo rosso $w$ viene unito con il suo genitore nero $p$ per formare un unico nodo del (2, 4) Tree.
    \begin{itemize}
        \item L'elemento in $w$ viene aggiunto alle chiavi in $p$.
        \item I figli di $w$ diventano figli di $p$.
    \end{itemize}
\end{itemize}

\begin{figure}[ht!]
    \centering
    \includegraphics[width=1\textwidth]{immagini/RBT/RBto24.png}
    \label{fig:RBto24}
\end{figure}



\section{Altezza di un Red-Black Trees}
Sia $T$ un Red-Black Tree con $n$ nodi interni e altezza $h$, allora vale la seguente disuguaglianza:
\[ 
\log (n + 1) -1 \le h \le 2 \log(n + 1) - 2 
\]
Sia $d$ la black-depth di $T$. Sia $T'$ il (2, 4)-Tree associato a $T$, e sia $h'$ l'altezza di $T'$. Per via della corrispondenza tra $T$ e $T'$, vale la relazione $h' = d$. Quindi, si ha che $d = h' \le \log (n + 1) - 1$, da cui si ricava che $h \le 2d \le 2 \log(n + 1)$. Sappiamo inoltre che vale la seguente proprietà: $h' \le 2d$. Quindi, otteniamo $h \le 2 \log (n + 1) - 2$. L'altra disuguaglianza, $ \log (n + 1) -1 \le h $ deriva dalle proprietà di un qualsiasi albero binario.



\section{Insert}
Per inserire un nuovo nodo in un Red-Black Tree, si segue lo stesso procedimento di un normale BST:
\begin{itemize}[nosep]
    \item Se il nuovo nodo $z$ è la radice, coloralo di nero.
    \item Altrimenti, inseriscilo come un nodo rosso.
\end{itemize} 
\noindent
L'inserimento eseguito in questo modo mantiene di già le Root, External, e Depth Properties. Tuttavia, potrebbe violare la Internal Property in alcuni casi:
\begin{itemize}[nosep]
    \item Se il genitore di $z$ è nero, anche la Internal Property è mantenuta.
    \item Se il genitore di $z$ è rosso, la Internal Property viene violata in quanto si ottiene una sequenza di due nodi rossi, un \textbf{double red}. In questo caso, bisogna ristrutturare l'albero per ripristinare le proprietà dei Red-Black Trees.
\end{itemize}

\subsection{Come risolvere un double red}
Siano $z$ e il suo genitore $v$ entrambi rossi, e sia $w$ il fratello di $v$ (lo zio di $z$).
\begin{itemize}[nosep]
    \item Se $w$ è nero (o None), il double-red corrisponde ad una trasformazione sbagliata di un 4-nodo nel (2, 4)-Tree corrispondente.
    \begin{itemize}[nosep]
        \item Si esegue una \textbf{ristrutturazione} (Three-node restructuring, singola o doppia rotazione) su $z$.
        \item Dopo la ristrutturazione, il nodo che diventa la radice della porzione ristrutturata viene colorato di nero, mentre i suoi due figli vengono colorati di rosso.
        \item Una sola ristrutturazione è sufficiente per risolvere il problema del double-red.
    \end{itemize}
    \item Se $w$ è rosso, il double-red corrisponde ad un overflow e quindi in un 5-nodo nel (2, 4)-Tree corrispondente.
    \begin{itemize}[nosep]
        \item Si esegue una \textbf{ricolorazione} (recolouring) colorando $v$ e $w$ di nero, mentre il loro genitore $u$ (genitore di $v$ e $w$, e nonno di $z$) viene colorato di rosso (se $u$ non è la radice).
        \item In questo caso il double-red può propagarsi verso l'alto, quindi potrebbe essere necessario ripetere la procedura sul nodo $u$ e il suo genitore.
    \end{itemize}
\end{itemize}

\clearpage
\begin{figure}[ht!]
    \centering
    \includegraphics[width=1\textwidth]{immagini/RBT/RBT_insert.png}
    \caption{Una sequenza di inserimenti in un albero rosso-nero: (a) albero iniziale; (b) inserimento di 7; (c) inserimento di 12, che causa un double red; (d) dopo la ristrutturazione; (e) inserimento di 15, che causa un double red; (f ) dopo la ricolorazione (la radice rimane nera); (g) inserimento di 3; (h) inserimento di 5; (i) inserimento di 14, che causa un double red; (j) dopo la ristrutturazione; (k) inserimento di 18, che causa un double red; (l) dopo la ricolorazione.}
    \label{fig:RBT_insert}
\end{figure}


\subsection{Complessità dell'inserimento}
Come abbiamo visto, l'inserimento in un Red-Black Tree richiede una prima operazione di ricerca di $O(\log n)$ per trovare la posizione corretta del nuovo nodo, la creazione di un nuovo nodo in $O(1)$, e infine possono essere necessarie al massimo $O(\log n)$ ricolorazioni (ciascuna impiega $O(1)$) e al più una sola ristrutturazione ($O(1)$) per mantenere le proprietà dell'albero. 

Pertanto, la complessità totale dell'inserimento in un Red-Black Tree è $O(\log n)$.



\section{Delete}
Per eliminare un nuovo nodo con chiave $k$ in un Red-Black Tree, si segue lo stesso procedimento di un normale BST:

\begin{itemize}
    \item \uline{Ciò vuol dire che eliminiamo sempre un nodo con al più un solo figlio}. Il nodo eliminato contiene una chiave $k$ o il suo predecessore/successore (in base all'implementazione) in ordine. Il nodo figlio di quello eliminato (se esiste) viene promosso a figlio del genitore del nodo eliminato.
\end{itemize} 


\subsection{Caso 1: Delete di un nodo rosso}
Se il nodo eliminato è rosso, tutte le proprietà dei Red-Black Trees rimangono valide, poiché la rimozione di un nodo rosso non altera la black depth di alcun percorso dalla radice a una foglia, e poiché questa operazione non può introdurre un double red. 

\vspace{1\baselineskip}
Nel corrispondente (2, 4)-Tree, la rimozione di un nodo rosso equivale alla rimozione di una chiave da un 3-nodo o da un 4-nodo, il che è sempre consentito senza ulteriori modifiche.

\subsection{Caso 2: Delete di un nodo nero con un solo figlio (rosso)}
\uline{Ricordiamo che stiamo trattando la rimozione di nodi con al più un figlio (per via della delete in un BST).} 
Se il nodo da eliminare è nero e ha un figlio, questo sarà sicuramente rosso (altrimenti la black depth property non sarebbe soddisfatta e non si avrebbe un Red-Black Tree valido).
In questo caso, possiamo semplicemente rimuovere il nodo nero e promuovere il figlio rosso al suo posto, colorandolo di nero, ristabilendo tutte le proprietà dei Red-Black Trees. 

\vspace{1\baselineskip}
Nel corrispondente (2, 4)-Tree, questa operazione equivale alla rimozione del nodo nero da un 3-nodo.

\vspace{8\baselineskip}
\noindent
Infine, consideriamo il caso più complesso, in cui il nodo da eliminare è un nodo nero senza figli.

\clearpage
\subsection{Caso 3: Delete di un nodo nero senza figli}
Il caso più complesso si verifica quando il nodo da eliminare è un nodo nero senza figli. Nel corrispondente (2, 4)-Tree, questa situazione equivale alla rimozione di una chiave da un 2-nodo. Senza un ribilanciamento, una modifica del genere comporta un deficit di uno per la black depth lungo il percorso che porta al nodo eliminato, violando così la Depth Property dei Red-Black Trees. 

Per rimediare a questo scenario, consideriamo un contesto più generale con un nodo $z$ che è noto per avere due sottoalberi, $T_{\text{heavy}}$ e $T_{\text{light}}$, tale che la radice di $T_{\text{light}}$ (se presente) è nera e tale che la black depth di $T_{\text{heavy}}$ è esattamente uno in più rispetto a quella di $T_{\text{light}}$, come illustrato in Figura \ref{fig:RBT_delete_deficit}. Nel caso di una foglia nera rimossa, $z$ è il genitore di quella foglia e $T_{\text{light}}$ è banalmente il sottoalbero vuoto che rimane dopo la cancellazione. Descriviamo il caso più generale di un deficit perché il nostro algoritmo per il ribilanciamento dell'albero, in alcuni casi, spingerà il deficit più in alto nell'albero (proprio come la risoluzione di una cancellazione in un (2,4) tree a volte si propaga verso l'alto). Indichiamo con $y$ la radice di $T_{\text{heavy}}$ (Un tale nodo esiste perché $T_{\text{heavy}}$ ha altezza nera almeno uno).

\begin{figure}[ht!]
    \centering
    \includegraphics[width=0.4\textwidth]{immagini/RBT/RBT_delete_deficit.png}
    \caption{Illustrazione di un deficit tra le altezze nere dei sottoalberi del nodo $z$. Il colore grigio nell'illustrare $y$ e $z$ denota il fatto che questi nodi possono essere colorati sia di nero che di rosso.}
    \label{fig:RBT_delete_deficit}
\end{figure}

\noindent
Ricapitolando:
\begin{itemize}[nosep]
    \item $z$ è un nodo con due sottoalberi $T_{\text{heavy}}$ e $T_{\text{light}}$, tali che:
    \begin{itemize}[nosep]
        \item $T_{\text{heavy}}$ ha altezza nera $d$.
        \item $T_{\text{light}}$ ha altezza nera $d - 1$.
    \end{itemize}
    \item $y$ è la radice di $T_{\text{heavy}}$, la quale esiste sempre.
    \item $z$ e $y$ possono essere sia rossi che neri.
\end{itemize}

\vspace{1\baselineskip}
\noindent
Distinguiamo tre possibili casi:
\begin{itemize}
    \item Nodo $y$ nero con (almeno) un figlio rosso $x$.
    \item Nodo $y$ nero e entrambi i figli di $y$ sono neri (o None).
    \item Nodo $y$ rosso.
\end{itemize}


\clearpage
\subsubsection{Caso 3.1: Nodo $y$ nero con (almeno) un figlio rosso $x$}
\paragraph{Soluzione:} Si esegue una \emph{ristrutturazione} \texttt{restructure(x)} sui tre nodi $x$, il suo genitore $y$, e il nonno $z$, rinominati temporaneamente come $a$, $b$, e $c$ in ordine di chiave. Sostituiamo $z$ con il nodo etichettato $b$, rendendolo il genitore degli altri due. Coloriamo $a$ e $c$ di nero, e diamo a $b$ il colore precedente di $z$.

\begin{figure}[ht!]
    \centering
    \includegraphics[width=1\textwidth]{immagini/RBT/RBT_delete_case1.png}
    \caption{Risoluzione di un deficit nero in $T_\text{light}$ attraverso una ristrutturazione su tre nodi \texttt{restructure(x)}. Sono mostrate due possibili configurazioni (le altre due sono simmetriche). Il colore grigio di $z$ nelle figure a sinistra denota il fatto che questo nodo può essere colorato sia di rosso che di nero. La radice della porzione ristrutturata assume lo stesso colore, mentre i figli di quel nodo sono entrambi colorati di nero nel risultato.}
    \label{fig:RBT_delete_case1}
\end{figure}

\vspace{1\baselineskip}
Nel caso in cui $y$ abbia entrambi i figli rossi, possiamo scegliere arbitrariamente uno dei due come $x$. Altrimenti, scegliamo l'unico figlio rosso di $y$ come $x$. Da notare che il percorso verso $T_{\text{light}}$ include un nodo nero aggiuntivo dopo la ristrutturazione, risolvendo così il suo deficit. Al contrario, il numero di nodi neri sui percorsi verso ciascuno degli altri tre sottoalberi illustrati in Figura \ref{fig:RBT_delete_case1} rimane invariato.

\vspace{1\baselineskip}
Risolvere questo caso corrisponde a un'operazione di \emph{transfer} nell'albero (2,4) $T'$ tra i due figli del nodo con $z$. Il fatto che $y$ abbia un figlio rosso ci assicura che rappresenta o un 3-nodo o un 4-nodo. In effetti, l'elemento precedentemente memorizzato in $z$ viene declassato per diventare un nuovo 2-nodo per risolvere la carenza, mentre un elemento memorizzato in $y$ o nel suo figlio viene promosso per prendere il posto dell'elemento precedentemente memorizzato in $z$.



\clearpage
\subsubsection{Caso 3.2: Nodo $y$ nero e entrambi i figli di $y$ sono neri (o None)}
\paragraph{Soluzione:} Si esegue una \emph{ricolorazione}, per cui coloriamo $y$ di rosso e, se $z$ è rosso, lo coloriamo di nero. 
\begin{itemize}[nosep]
    \item Se $z$ era originariamente rosso, questa ricolorazione risolve il deficit.
    \item Se $z$ era originariamente nero, la ricolorazione non risolve il deficit, ma lo propaga più in alto nell'albero; dobbiamo ripetere la considerazione di tutti e tre i casi sul genitore di $z$ come rimedio.
\end{itemize}
\noindent
Questa ricolorazione non introduce alcun double-red, poiché $y$ non ha figli rossi.


\begin{figure}[ht!]
    \centering
    \includegraphics[width=0.8\textwidth]{immagini/RBT/RBT_delete_case2.png}
    \caption{Risoluzione di un deficit nero in $T_\text{light}$ tramite una ricolorazione. (a) quando $z$ è originariamente rosso, si invertono i colori di $y$ e $z$ per risolvere il deficit nero in $T_\text{light}$, terminando il processo; (b) quando $z$ è originariamente nero, la ricolorazione di $y$ causa un deficit nero nell'intero sottoalbero di $z$, trasportando il problema ad un livello superiore che andrà risolto seguendo uno dei tre casi descritti.}
    \label{fig:RBT_delete_case2}
\end{figure}

\vspace{1\baselineskip}
\noindent
La soluzione in questo caso corrisponde all'operazione di \emph{fusion} nell'albero (2, 4) $T'$, poiché $y$ deve rappresentare un 2-nodo. 
Nel caso in cui $z$ era originariamente rosso, e quindi il genitore nel corrispondente albero (2,4) è un 3-nodo o un 4-nodo, questa ricolorazione risolve il deficit. (Vedi Figura \ref{fig:RBT_delete_case2}a.) Il percorso che porta a $T_{\text{light}}$ include un nodo nero aggiuntivo nel risultato, mentre la ricolorazione non ha influenzato il numero di nodi neri sul percorso verso i sottoalberi di $T_{\text{heavy}}$. Nel caso in cui $z$ fosse originariamente nero, e quindi il genitore nel corrispondente albero (2, 4) è un 2-nodo, la ricolorazione non ha aumentato il numero di nodi neri sul percorso verso $T_{\text{light}}$; in effetti, ha ridotto il numero di nodi neri sul percorso verso $T_{\text{heavy}}$. (Vedi Figura \ref{fig:RBT_delete_case2}b.) Dopo questo passaggio, i due figli di $z$ avranno la stessa altezza nera. Tuttavia, l'intero albero radicato in $z$ è diventato carente, propagando così il problema più in alto nell'albero; dobbiamo ripetere la considerazione di tutti e tre i casi sul genitore di $z$ come rimedio.




\clearpage
\subsubsection{Caso 3.3: Nodo $y$ rosso}
\paragraph{Soluzione:} Si esegue una \emph{rotazione} su $y$ e $z$, seguita da una ricolorazione di $y$ in nero e di $z$ in rosso. Inoltre, poichè $y$ era originariamente rosso, il nuovo sottoalbero di $z$ deve avere una radice nera $y'$ e deve avere un'altezza nera uguale a quella originale di $T_{\text{heavy}}$. Pertanto, un deficit nero rimane nel nodo $z$ dopo la trasformazione, e quindi riapplichiamo l'algoritmo per risolvere il deficit in $z$, sapendo che il nuovo figlio $y'$, che è la radice di $T_{\text{heavy}}$ è ora nero, e quindi che si applica o il Caso 3.1 o il Caso 3.2. Inoltre, la prossima applicazione sarà l'ultima, perché il Caso 3.1 è sempre terminale e il Caso 3.2 sarà terminale dato che $z$ è rosso.

\begin{figure}[ht!]
    \centering
    \includegraphics[width=1\textwidth]{immagini/RBT/RBT_delete_case3.png}
    \caption{Una rotazione e una ricolorazione su un nodo rosso $y$ e un nodo nero $z$, assumendo un deficit nero in $z$. Questo equivale a un cambiamento di orientamento nel corrispondente 3-nodo di un albero (2,4). Questa operazione non influisce sull'altezza nera di alcun percorso attraverso questa porzione dell'albero. Inoltre, poiché $y$ era originariamente rosso, il nuovo sottoalbero di $z$ deve avere una radice nera $y'$ e deve avere un'altezza nera uguale a quella originale di $T_{\text{heavy}}$. Pertanto, un deficit nero rimane nel nodo $z$ dopo la trasformazione.}
    \label{fig:RBT_delete_case3}
\end{figure}

\vspace{1\baselineskip}
Da notare che inizialmente $y$ è rosso e $T_{\text{heavy}}$ ha altezza nera almeno 1, $z$ deve essere nero e i due sottoalberi di $y$ devono avere ciascuno una radice nera e un'altezza nera uguale a quella di $T_{\text{heavy}}$.

\vspace{1\baselineskip}
Le prime operazioni di rotazione e ricolorazione denotano una riorientazione di un 3-nodo nel corrispondente albero (2,4) $T'$.


\clearpage
\begin{figure}[ht!]
    \centering
    \includegraphics[width=1\textwidth]{immagini/RBT/RBT_delete_example.png}
    \caption{Una sequenza di cancellazioni da un Red-Black Tree: (a) albero iniziale; (b) rimozione di 3; (c) rimozione di 12, che causa un deficit nero a destra di 7 (risolto tramite ristrutturazione); (d) dopo la ristrutturazione; (e) rimozione di 17; (f) rimozione di 18, che causa un deficit nero a destra di 16 (risolto tramite ricolorazione); (g) dopo la ricolorazione; (h) rimozione di 15; (i) rimozione di 16, che causa un deficit nero a destra di 14 (risolto inizialmente tramite una rotazione); (j) dopo la rotazione il deficit nero deve essere risolto tramite una ricolorazione; (k) dopo la ricolorazione.}
    \label{fig:RBT_delete_example}
\end{figure}



\clearpage
\subsection{Complessità della cancellazione}
L'algoritmo per eliminare un elemento da un Red-Black Tree con $n$ elementi richiede $O(\log n)$ tempo e esegue $O(\log n)$ ricolorazioni e al massimo due operazioni di ristrutturazione.


\subsection*{Riepilogo Delete}
\begin{figure}[ht!]
    \centering
    \includegraphics[width=1\textwidth]{immagini/RBT/RBT_delete_summary.png}
    \caption{Riepilogo dei casi di delete in un Red-Black Tree.}
    \label{fig:RBT_delete_summary}
\end{figure}




\cleardoublepage
\input{capitoli/cap5_HashTables}

\cleardoublepage
\chapter{Priority Queues}
\label{cap:PriorityQueues}

Una \textbf{coda con priorità} (\emph{Priority Queue}) è una struttura dati astratta fondamentale, simile a una coda standard (\emph{Queue}), ma con una differenza cruciale nel criterio di estrazione. Mentre una coda standard opera secondo la logica \textbf{FIFO} (First-In, First-Out), rimuovendo l'elemento che è in attesa da più tempo, una coda a priorità rimuove sempre l'elemento con la \textbf{priorità} più alta (o più bassa, a seconda di quale logica si vuole utilizzare), indipendentemente dall'ordine di inserimento. 

È importante non confondere una coda a priorità con una mappa (Map) o dizionario. Lo scopo di una \textbf{Map} è l'associazione e la ricerca rapida: memorizza coppie $\langle Chiave, Valore \rangle$ e risponde efficientemente alla domanda: "Qual è il valore associato a questa specifica chiave?". Al contrario, lo scopo di una \textbf{Priority Queue} è l'estrazione efficiente dell'elemento più importante. Sebbene gli elementi in una coda a priorità siano spesso implementati come coppie, ad esempio $\langle Priorit\grave{a}, Dato \rangle$ (a cui spesso ci si riferisce comunque come (chiave, valore)), questa coppia ha una funzione diversa: la $Priorit\grave{a}$ non serve per cercare il $Dato$ (come farebbe una chiave in una mappa), ma serve solo alla struttura interna della coda (spesso uno \textit{Heap}) per determinare l'ordine di estrazione. In sintesi, si usa una mappa per \emph{trovare} un elemento tramite un identificatore unico (la chiave), mentre si usa una coda a priorità per \emph{estrarre} l'elemento con il grado di urgenza massimo o minimo.


\paragraph{Definizione:} Una \textbf{Priority Queue} è una raccolta di elementi con priorità che consente l'inserimento arbitrario di elementi e la rimozione dell'elemento con priorità primaria. Quando un elemento viene aggiunto in una Priority Queue questo assume una determinata priorità rappresentata dalla chiave ad esso associata. L'elemento con la chiave minima sarà il successivo a essere rimosso dalla coda (quindi, a un elemento con chiave 1 verrà data priorità su un elemento con chiave 2).



\section{Chiavi (Priorità)}
In una Priority Queue, ogni elemento è associato a una chiave che determina la sua priorità. Le chiavi in una Priority Queue possono essere oggetti di qualunque tipo tale che sia possibile definire un ordinamento totale su di esse. Con tale generalità, le applicazioni possono sviluppare la propria nozione di priorità per ciascun elemento.

A marcare la differenza con la struttura dati Map, in una Priority Queue \textbf{le chiavi non sono necessariamente uniche}. È possibile avere più elementi con la stessa chiave (priorità). In questi casi, la politica di estrazione per gli elementi con chiavi identiche può variare a seconda dell'implementazione specifica della Priority Queue. Alcune implementazioni potrebbero adottare una politica FIFO per gli elementi con la stessa priorità, mentre altre potrebbero non garantire alcun ordine specifico tra di essi.



\section{Possibili implementazioni di Priority Queue}
Esistono diverse strategie per implementare una Priority Queue. Due approcci semplici utilizzano liste (array o liste collegate) per memorizzare gli elementi, con differenze significative nelle prestazioni delle operazioni di inserimento e rimozione.

\begin{figure}[ht!]
    \centering
    \includegraphics[width=0.9\textwidth]{immagini/PriorityQueues/array_based.png}
    \label{fig:array_based}
\end{figure}

Le due strategie per implementare una Priority Queue ADT dimostrano un interessante compromesso. Quando si utilizza una lista non ordinata per memorizzare gli elementi, possiamo eseguire inserimenti in tempo O(1), ma trovare o rimuovere un elemento con chiave minima richiede un ciclo in tempo O(n) attraverso l'intera collezione. Al contrario, se si utilizza una lista ordinata, possiamo banalmente trovare o rimuovere l'elemento minimo in tempo O(1), ma aggiungere un nuovo elemento alla coda potrebbe richiedere tempo O(n) per ripristinare l'ordinamento.

Esistono però implementazioni più efficienti, seppur più complesse, che consentono di eseguire sia inserimenti che rimozioni in tempo logaritmico. Una di queste implementazioni si basa su una struttura dati chiamata \textbf{heap binario}, la quale utilizza la struttura di un albero binario per trovare un compromesso tra elementi completamente non ordinati e perfettamente ordinati.



\section{Heap}
\paragraph{Definizione:}Un \textbf{Heap} è un albero binario che soddisfa le seguenti proprietà:
\begin{itemize}
    \item \textbf{Albero binario completo}: L'albero è un albero binario completo (vedi pag.6), cioè tutti i livelli dell'albero sono completamente riempiti, tranne eventualmente l'ultimo livello, che è riempito da sinistra a destra.
        \begin{itemize}
            \item Sia $h$ l'altezza dell'albero, per $i = 0, 1, \ldots, h-1$, il livello $i$-esimo contiene esattamente $2^i$ nodi. Se il livello $h-1$ (cioè nel livello più basso) non è pieno, tutti i suoi nodi sono riempiti da sinistra a destra.
        \end{itemize}
    \item \textbf{Heap-Order}: Per ogni posizione (nodo) $p$, la chiave di $p$ è minore o uguale alle chiavi dei suoi figli. 
        \begin{itemize}
            \item Questo implica che la chiave minima si trova sempre nella radice dell'albero.
        \end{itemize}
\end{itemize}
Dal momento che un Heap è un albero binario completo, la sua altezza è sempre logaritmica rispetto al numero di nodi nell'albero. Questo fatto è cruciale per garantire che le operazioni di inserimento e rimozione possano essere eseguite in tempo $O(\log n)$.
Si ha quindi: 
\[
h = \lfloor \log_2(n) \rfloor
\]



\section{Implementazione di una Priority Queue attraverso Heap}
Un Heap può essere efficacemente utilizzato per implementare una Priority Queue, sfruttando le sue proprietà di albero binario completo e heap-order. Le operazioni principali di una Priority Queue, ovvero l'inserimento di un elemento e la rimozione dell'elemento con priorità minima, possono essere eseguite in tempo logaritmico grazie alla struttura dell'Heap.

\begin{itemize}
    \item In ogni nodo dello Heap viene memorizzata una coppia chiave-valore, dove la chiave rappresenta la priorità dell'elemento.
    \begin{itemize}
        \item N.B.: è possibile avere più nodi con la stessa chiave, in quanto si tratta di un'implementazione di una Priority Queue.
    \end{itemize}
    \item Manteniamo un riferimento all'ultimo nodo dello Heap, cioè il nodo più a destra sull'ultimo livello dell'albero.
\end{itemize}


\clearpage
\subsection*{Inserimento \texttt{add(k,v)}}
Si vuole inserire una nuova coppia chiave-valore (k, v) nell'Heap. Per mantenere la proprietà di albero binario completo, dobbiamo inserire il nuovo nodo nella posizione corretta, ovvero appena oltre il nodo più a destra al livello più basso dell'albero, o come posizione più a sinistra di un nuovo livello, se il livello più basso è già pieno (o se l'heap è vuoto).

La posizione del nodo nel quale inserire il nuovo elemento può essere trovato in un tempo $O(\log n)$ a partire dal riferimento all'ultimo nodo:
\begin{itemize}
    \item Si parte dal riferimento all'ultimo nodo.
    \item Si risale l'albero fino alla radice o ad un nodo che è figlio sinistro del suo genitore.
    \item Se il nodo in cui si è giunti è il figlio sinistro del suo genitore, vai al nodo fratello.
    \item Infine, scendi sempre a sinistra fino a raggiungere un nodo foglia (None). Questa sarà la posizione in cui inserire il nuovo nodo.
\end{itemize}



\subsection*{Up-Heap Bubbling dopo l'inserimento}
Dopo aver inserito il nuovo nodo nella posizione corretta al fine di mantenere la proprietà di albero binario completo, è possibile che la proprietà di heap-order venga violata, poiché la chiave del nuovo nodo potrebbe essere minore della chiave del suo genitore. Per ripristinare la proprietà di heap-order, si esegue un processo chiamato \textbf{up-heap bubbling}.

\begin{itemize}
    \item Partendo dal nuovo nodo con chiave $k$, e risalendo l'albero verso la radice, si scambia il nodo corrente con il suo genitore fino a quando la heap-order property non è ristabilita.
    \item L'algoritmo termina quando la chiave $k$ raggiunge la radice dell'albero o quando la chiave $k$ si trova in un nodo il cui padre ha una chiave minore o uguale a $k$.
\end{itemize}
Dal momento che l'altezza dell'Heap è $O(\log n)$, l'operazione di up-heap bubbling richiede un tempo $O(\log n)$ nel caso peggiore.

\begin{figure}[ht!]
    \centering
    \includegraphics[width=1\textwidth]{immagini/PriorityQueues/up_heap_bubbling.png}
    \label{fig:up_heap_bubbling}
\end{figure}



\clearpage
\subsection*{Cancellazione \texttt{remove\_min()}}
Il metodo \texttt{remove\_min()} rimuove e restituisce l'elemento con la chiave minima dalla Priority Queue. In un Heap, l'elemento con la chiave minima si trova sempre nella radice dell'albero. Per mantenere la proprietà di albero binario completo dopo la rimozione della radice, dobbiamo sostituire la radice con l'ultimo nodo dell'Heap (cioè il nodo più a destra nell'ultimo livello dell'albero) e poi rimuovere l'ultimo nodo.


\subsection*{Down-Heap Bubbling dopo la cancellazione}
Dopo aver effettuato la sostituzione della radice con l'ultimo nodo, è possibile che la proprietà di heap-order venga violata, poiché la chiave del nuovo nodo radice potrebbe essere maggiore della chiave di uno o entrambi i suoi figli. Per ripristinare la proprietà di heap-order, si esegue un processo chiamato \textbf{down-heap bubbling}.

\begin{itemize}
    \item Partendo dalla radice, si scambia il nodo corrente con il figlio che ha la chiave minima, a condizione che la chiave del figlio sia minore della chiave del nodo corrente.
    \item Questo processo continua fino a quando la chiave del nodo corrente è minore o uguale alle chiavi dei suoi figli, o fino a quando il nodo corrente diventa un nodo foglia (None), per cui la heap-order property è ristabilita.
\end{itemize}
Dal momento che l'altezza dell'Heap è $O(\log n)$, l'operazione di down-heap bubbling richiede un tempo $O(\log n)$ nel caso peggiore.

\begin{figure}[ht!]
    \centering
    \includegraphics[width=1\textwidth]{immagini/PriorityQueues/down_heap_bubbling.png}
    \label{fig:down_heap_bubbling}
\end{figure}


\clearpage
\section{Implementazione Array-based di un Heap}
È sempre possibile rappresentare un albero binario utilizzando un array, sfruttando la relazione tra gli indici dei nodi genitori e figli. In generale questa rappresentazione è meno efficiente in termini di spazio rispetto a una rappresentazione basata su nodi collegati, poiché richiede spazio per tutti i nodi, compresi quelli vuoti. 

Tuttavia, per un albero binario \emph{completo} come un Heap, questa rappresentazione è particolarmente efficiente, poiché non ci sono nodi vuoti tra i nodi effettivamente presenti nell'albero. Sia $n$ il numero di nodi nell'albero, nel caso di un albero binario qualsiasi l'array può avere nel caso peggiore $N = 2^n -1$ elementi, mentre nel caso di albero binario completo l'array avrà esattamente $N = n$ elementi.

\vspace{1\baselineskip}
Questo procedimento vale per ogni albero binario, ma è particolarmente efficiente per un Heap, poiché l'albero è completo. La mappatura tra i nodi dell'albero e gli indici dell'array avviene seguendo queste regole:

\begin{itemize}

    \item Il nodo radice dell'albero viene memorizzato all'indice 0 dell'array.
    \item Per un nodo situato all'indice $i$ nell'array:
    \begin{itemize}
        \item Il figlio sinistro del nodo si trova all'indice $2i + 1$.
        \item Il figlio destro del nodo si trova all'indice $2i + 2$.
        \item Il genitore del nodo si trova all'indice $\lfloor (i - 1) / 2 \rfloor$, a condizione che $i > 0$.
    \end{itemize}
\end{itemize}

\begin{figure}[ht!]
    \centering % Centra l'intera figura
    
    % --- Prima Immagine (a) ---
    \begin{subfigure}{0.48\textwidth}
        \centering
        \includegraphics[width=\textwidth]{immagini/PriorityQueues/heap_example.png}
        \caption{Esempio di uno Heap con 13 elementi. L'ultima posizione è occupata dal nodo con chiave 13.}
        \label{fig:heap_example}
    \end{subfigure}
    % --- Seconda Immagine (b) ---
    \begin{subfigure}{0.5\textwidth}
        \centering
        \includegraphics[width=\textwidth]{immagini/PriorityQueues/heap_array.png}
        \caption{Sua rappresentazione come array. }
        \label{fig:heap_array}
    \end{subfigure}
    
    \label{fig:figura_completa}
\end{figure}


\clearpage
\section{Analisi delle prestazioni}
\begin{figure}[ht!]
    \centering
    \includegraphics[width=0.8\textwidth]{immagini/PriorityQueues/complexity.png}
    \label{fig:complexity}
\end{figure}

In figura sono riassunte le complessità temporali delle operazioni principali di una Priority Queue implementata tramite un Heap binario, assumendo che due chiavi possano essere confrontate in tempo costante $O(1)$, e che l'Heap sia implementato come array-based o linked-based tree. L'analisi è basata sulle seguenti considerazioni:
\begin{itemize}
    \item L'Heap ha $n$ nodi, ognuno dei quali contiene una coppia chiave-valore.
    \item L'altezza dell'Heap è $O(\log n)$ [In particolare, $h = \lfloor \log_2 n \rfloor$], dal momento che si tratta di un albero binario completo.
    \item L'operazione \texttt{min()} viene eseguita in $O(1)$ perché la radice dell'albero contiene tale elemento.
    \item Nel caso peggiore, up-heap e down-heap eseguono un numero di scambi pari all'altezza dell'Heap.
\end{itemize}

Concludiamo che la struttura dati Heap è una realizzazione molto efficiente dell'ADT Priority Queue, indipendentemente dal fatto che l'heap sia implementato come array-based o linked-based tree. L'implementazione basata su heap raggiunge tempi di esecuzione rapidi sia per l'inserimento che per la rimozione, a differenza delle implementazioni basate sull'utilizzo di una sorted-list o unsorted-list.



\section{Costruzione di un Heap da una lista di elementi: Heapify}
A partire da un Heap vuoto, $n$ inserimenti successivi al suo interno richiederebbero un tempo complessivo di $O(n \log n)$, poiché ogni inserimento richiede un tempo $O(\log n)$. Tuttavia, se le $n$ coppie chiave-valore sono note a prescindere, esiste un algoritmo più efficiente chiamato \textbf{Heapify} che consente di costruire un Heap a partire da una lista di $n$ elementi in tempo lineare $O(n)$.

\subsection*{Fusione di due Heap}
Supponiamo di avere due Heap della stessa altezza $h$, e un nuovo elemento con chiave $k$. Vogliamo creare un nuovo Heap di altezza $h + 1$ che contenga tutti gli elementi dei due Heap e l'elemento con chiave $k$. Per fare ciò, possiamo seguire questi passaggi:
\begin{itemize}
    \item Creiamo un nuovo nodo radice con chiave $k$.
    \item Assegniamo i due Heap esistenti come figli sinistro e destro della nuova radice.
    \item Eseguiamo l'operazione di down-heap bubbling a partire dalla radice per ripristinare la proprietà di heap-order.
\end{itemize}

\subsection*{Heapify}
Per semplicità ipotizziamo che il numero di elementi $n$ sia un intero tale che $n = 2^{h+1} - 1$ per qualche intero $h \geq 0$, in modo che l'Heap risultante sia un albero binario completo, con anche l'ultimo livello completamente pieno, per cui l'Heap ha altezza $h = \log_2(n+1) - 1$. L'algoritmo Heapify può essere visto come una sequenza di $h + 1 = \log_2(n + 1)$ step [dato dal fatto che in un albero di altezza $h$ ci sono $h + 1$ livelli, numerati da 0 a $h$]:

\begin{itemize}[labelwidth=!, leftmargin=*, align=right]
    \item[1.] Nel primo step (Figura 9.5b), costruiamo $(n+1)/2$ Heap di altezza 0 (cioè nodi singoli).

    \item[2.] Nel secondo step (Figura 9.5c-d), costruiamo $(n+1)/4$ Heap di altezza 1, ciascuno contenente tre nodi, unendo coppie di Heap elementari (ottenuti al passo precedente) e aggiungendo un nuovo elemento. Il nuovo nodo viene posizionato alla radice e potrebbe dover essere scambiato con un elemento figlio per preservare la heap-order property.

    \item[3.] Nel terzo step (Figura 9.5e-f), costruiamo $(n+1)/8$ Heap di altezza 2, ciascuno contenente sette nodi, unendo coppie di Heap (ottenuti al passo precedente) e aggiungendo un nuovo elemento. Il nuovo nodo viene posizionato alla radice e potrebbe dover essere scambiato attraverso un down-heap bubbling per preservare la heap-order property.
    
    \item[$\vdots$]

    \item[i.] Nel generico $i$-esimo step, con $2 \le i \le h$, costruiamo $(n+1)/2^i$ Heap di altezza $i-1$, ciascuno contenente $2^i - 1$ nodi, unendo coppie di Heap (ottenuti al passo precedente) e aggiungendo un nuovo elemento. Il nuovo nodo viene posizionato alla radice e potrebbe dover essere scambiato attraverso un down-heap bubbling per preservare la heap-order property.

    \item[$\vdots$]

    \item[$h+1$.] Nell'ultimo step (Figura 9.5g-h), costruiamo un singolo Heap di altezza $h$, contenente tutti e $n$ gli elementi, unendo gli ultimi due Heap che risultano essere di $(n - 1)/2$ elementi e di altezza $h - 1$ (ottenuti al passo precedente) e aggiungendo un nuovo elemento. Il nuovo nodo viene posizionato alla radice e potrebbe dover essere scambiato attraverso un down-heap bubbling per preservare la heap-order property.
\end{itemize}

\begin{figure}[ht!]
    \centering
    \includegraphics[width=1\textwidth]{immagini/PriorityQueues/heapify.png}
    \caption{Costruzione bottom-up di un Heap con 15 elementi: (a, b) iniziamo costruendo Heap elementari di un solo elemento; (c, d) combiniamo questi Heap in Heap di 3 elementi, e poi (e, f) in Heap di 7 elementi, fino a (g, h) dove otteniamo l'Heap finale di 15 elementi. Il path del down-heap bubbling è evidenziato in (d, f, h). Per semplicità è mostrata solamente la chiave in ogni nodo anziché la coppia chiave-valore.}
    \label{fig:heapify}
\end{figure}

\clearpage
\section{Heap-Sort}
L'algoritmo \textbf{Heap-Sort} sfrutta la struttura dati Heap: data la lista di partenza, inserisco tutti gli elementi in un \textbf{Max-Heap} (al fine di avere gli elementi in ordine crescente alla fine dell'algoritmo) e poi estraggo ripetutamente l'elemento massimo (la radice del Max-Heap) per costruire la lista ordinata. Questo algoritmo viene eseguito già in tempo $O(n \log n)$, poiché l'inserimento di $n$ elementi in un Heap richiede un tempo $O(n)$ (grazie all'algoritmo Heapify) e ogni estrazione dell'elemento massimo richiede un tempo $O(\log n)$, per un totale di $n$ estrazioni.

\subsection*{L'idea per un ordinamento in-place}
L'algoritmo che presentiamo di seguito è una versione più raffinata dello stesso Heap-sort che anziché utilizzare memoria aggiuntiva permette di avere un \textbf{ordinamento in-place} (cioè occupa al più memoria aggiuntiva $O(1)$) dell'Heap-sort.

L'idea chiave è dividere l'array $C$ in due parti contigue:
\begin{enumerate}
    \item \textbf{La parte sinistra (Heap):} Da $C[0]$ a $C[i-1]$.
    \item \textbf{La parte destra (Sequenza):} Da $C[i]$ a $C[n-1]$.
\end{enumerate}
Durante l'algoritmo, il confine $i$ tra queste due parti si sposta.

\clearpage
\subsection*{Fase 1: costruzione di un Max Heap}
\textbf{Obiettivo:} Trasformare l'array $C$ in un unico Max-Heap.

\vspace{1\baselineskip}
\noindent
In questa fase, si parte da un heap vuoto e si sposta il confine $i$ \textbf{da sinistra verso destra}, da $1$ a $n$. La parte sinistra (l'Heap) cresce, mentre la parte destra (la Sequenza di input) si riduce.

Per ogni passo $i$, da $i=1$ fino a $n$:
\begin{enumerate}
    \item \textbf{Azione (Espansione):} Il confine si sposta. L'elemento $C[i-1]$ (il primo della Sequenza) viene aggiunto all'Heap, diventandone l'ultima foglia.
    \item \textbf{Aggiustamento (Up-Heap Bubbling):} L'aggiunta di $C[i-1]$ potrebbe violare la proprietà del Max-Heap. Si esegue quindi un \textbf{Up-Heap Bubbling} a partire da $C[i-1]$. Questo elemento viene scambiato con il suo genitore finché non è più piccolo del genitore o non raggiunge la radice $C[0]$.
\end{enumerate}

\noindent
Alla fine della Fase 1, $i=n$. La parte Heap occupa l'intero array $C[0 \dots n-1]$ e la Sequenza è vuota. L'intero array è ora un Max-Heap valido.

\subsection*{Fase 2: estrazione degli elementi in ordine}
\textbf{Obiettivo:} Estrarre gli elementi dall'Heap in ordine decrescente per costruire l'array ordinato.

\vspace{1\baselineskip}
\noindent
In questa fase, si parte con tutti gli elementi nell'Heap e la Sequenza vuota. Si sposta il confine \textbf{da destra verso sinistra}.

Per ogni passo $i$, da $i=1$ fino a $n$:
\begin{enumerate}
    \item \textbf{Azione (Estrazione):} L'elemento massimo dell'Heap si trova sempre alla radice, $C[0]$. Questo elemento deve essere spostato nella sua posizione ordinata finale, che in questo passo è $C[n-i]$ (la prima posizione libera a sinistra della Sequenza ordinata). Si \textbf{scambiano} $C[0]$ e $C[n-i]$.
    \item \textbf{Aggiustamento (Down-Heap Bubbling):} Dopo lo scambio, $C[n-i]$ contiene il massimo ed è ora "bloccato" (fa parte della Sequenza ordinata). L'Heap si è ridotto (ora va da $C[0]$ a $C[n-i-1]$). L'elemento che è finito in $C[0]$ (quello che era $C[n-i]$) è probabilmente fuori posto e viola la proprietà del max-Heap. Si esegue quindi un \textbf{Down-Heap Bubbling} a partire dalla radice $C[0]$. Questo elemento "scende" scambiandosi con il suo figlio \textit{maggiore}, finché non è più grande di entrambi i figli o non raggiunge una foglia dell'Heap.
\end{enumerate}

\noindent
Alla fine della Fase 2, $i=n$. La parte "Heap" è vuota e la "Sequenza" (ora ordinata in senso crescente) occupa l'intero array.

\begin{figure}
    \centering
    \includegraphics[width=1\textwidth]{immagini/PriorityQueues/heap_sort.png}
    \caption{Fase 2 di un Heap-Sort in-place. La porzione Heap è evidenziata in grigio all'interno dell'array, per ogni iterazione dell'algoritmo. L'albero binario equivalente alla porzione Heap per ogni iterazione è rappresentato graficamente con il percorso più recente di Down-Heap Bubbling evidenziato.}
    \label{fig:heap_sort}
\end{figure}



\clearpage
\section{Adaptable Priority Queue}
L'ADT Priority Queue è sufficiente per molte applicazioni, ma in alcuni casi è indispensabile avere delle funzionalità aggiuntive per gestire in modo più flessibile gli elementi all'interno della coda, per esempio modificare la chiave (priorità) di un elemento esistente o rimuovere un elemento specifico (non necessariamente quello con priorità minima). Per questo motivo, presentiamo una variante chiamata \textbf{Adaptable Priority Queue} che estende l'ADT Priority Queue con queste funzionalità aggiuntive.

\subsection*{Locators}
Per poter implementare in modo efficiente le nuove operazioni di modifica e cancellazione, è necessario trovare un meccanismo che ci permetta di trovare uno specifico elemento all'interno della coda senza dover scorrere l'intera struttura dati. Per questo motivo, quando un elemento viene inserito nella coda, viene restituito uno speciale oggetto chiamato \textbf{locator} al chiamante. Di conseguenza, ogni volta che si desidera modificare o rimuovere un elemento specifico di una coda a priorità $P$ è necessario utilizzare il locator associato a quell'elemento per accedervi direttamente.

\vspace{1\baselineskip}
\noindent
\begin{tabularx}{\textwidth}{@{} r @{ : } X @{}}
    % La colonna 1 è 'r' (left-aligned)
    % La colonna 2 è 'X' (testo con a-capo automatico)
    % '@{ : }' inserisce i due punti allineati tra le colonne

    \texttt{P.update(loc, k, v)} & Sostituisce la chiave $k$ e il valore $v$ dell'elemento associato al locator $loc$ nella coda a priorità $P$. \\

    \texttt{P.remove(loc)} & Rimuove l'elemento associato al locator $loc$ dalla coda a priorità $P$ e lo restituisce. \\
\end{tabularx}

\vspace{1\baselineskip}
\noindent
L'astrazione del \emph{locator} è in qualche modo simile all'astrazione della \emph{position}. Tuttavia, facciamo una distinzione tra un locator e una position perché un locator per una coda prioritaria non rappresenta una collocazione tangibile di un elemento all'interno della struttura. Nella nostra coda prioritaria, un elemento può essere ricollocato all'interno della nostra struttura dati durante un'operazione che non sembra direttamente rilevante per quell'elemento. \uline{Un locator per un elemento rimarrà valido fintanto che quell'elemento rimarrà da qualche parte nella coda}.

\subsection*{Implementazione di un Adaptable Priority Queue}
L'implementazione della classe Locator estende la classe \_Item dell'ADT Priority Queue per includere un campo aggiuntivo che tiene traccia della posizione corrente dell'elemento all'interno della rappresentazione basata su array del nostro Heap. Questo campo aggiuntivo consente di accedere rapidamente alla posizione dell'elemento nell'Heap, facilitando le operazioni di aggiornamento e rimozione, come mostrato Nella Figura \ref{fig:locator1}.

\clearpage
\begin{figure}[ht!]
    \centering
    \includegraphics[width=1\textwidth]{immagini/PriorityQueues/locator1.png}
    \caption{Rappresentazione di un Heap utilizzando una sequenza di Locator. Il terzo elemento di ciascuna istanza di Locator corrisponde all'indice dell'elemento all'interno dell'array. Si presume che l'identificatore "token" sia un riferimento al localizzatore nello scope dell'utente.}
    \label{fig:locator1}
\end{figure}

La lista è una sequenza di riferimenti a istanze di Locator, ognuna delle quali memorizza una chiave, un valore, e l'indice corrente dell'elemento all'interno dell'array che rappresenta l'Heap. All'utente verrà fornito un riferimento al Locator corrispondente per ciascun elemento inserito, come illustrato dall'identificatore "token" nella Figura \ref{fig:locator1}.

Quando eseguiamo delle operazioni sulla Priority Queue che potrebbero alterare la posizione di un elemento all'interno dell'Heap (come l'inserimento che comporta un Up-Heap Bubbling, la rimozione che comporta un Down-Heap Bubbling o l'aggiornamento della chiave), dobbiamo assicurarci di aggiornare il campo dell'indice all'interno del Locator corrispondente. Questo garantisce che il Locator rimanga valido e punti sempre alla posizione corretta dell'elemento nell'Heap. 

\begin{figure}[ht!]
    \centering
    \includegraphics[width=1\textwidth]{immagini/PriorityQueues/locator2.png}
    \caption{È possibile osservare lo stato dell'Heap dopo aver eseguito una \texttt{remove\_min()}. Questa operazione causa un Down-Heap Bubbling che ricolloca gli elementi all'interno dell'Heap, e di conseguenza aggiorna gli indici nei Locator associati agli elementi coinvolti nel bubbling.}
    \label{fig:locator2}
\end{figure}

\begin{figure}[ht!]
    \centering
    \includegraphics[width=0.7\textwidth]{immagini/PriorityQueues/complexity_APQ.png}
    \label{fig:complexity_APQ}
\end{figure}

\cleardoublepage
\chapter{Pattern Matching}
\label{cap:PatternMatching}



Introduciamo di seguito la terminologia di base:
\begin{itemize}
    \item \(\Sigma\) : l'alfabeto, ovvero l'insieme di caratteri possibili.
    \item \(|\Sigma|\) : la dimensione dell'alfabeto.
    \item Stringa \( S \) : una sequenza finita di caratteri appartenenti all'alfabeto \(\Sigma\), di lunghezza \( m \).
    \item \(S[i]\) : il carattere alla posizione \( i \) della stringa \( S \).
    \item \(S[i..j]\) : la sottostringa di \( S \) che va dall'indice \( i \) all'indice \( j \).
        \begin{itemize}
            \item In Python: $S$[i:j+1] 
        \end{itemize}
    \item \(S[0..k]\) : prefisso di lunghezza \( k+1 \) della stringa \( S \).
        \begin{itemize}
            \item In Python: $S$[:k+1]
        \end{itemize}
    \item \(S[j..m-1]\) : suffisso di lunghezza \( m-j \) della stringa \( S \). 
        \begin{itemize}
            \item In Python: $S$[j:]
        \end{itemize}
\end{itemize}

\noindent
Nel classico problema di pattern matching, ci viene data una stringa di testo $T$ di lunghezza $n$ e una stringa di pattern $P$ di lunghezza $m$, e vogliamo scoprire se $P$ è una sottostringa di $T$. In tal caso, potremmo voler trovare l'indice più basso $j$ all'interno di $T$ in cui inizia $P$, in modo che $T[j..j+m-1]$ sia uguale a $P$, o forse trovare tutti gli indici di $T$ in cui inizia il pattern $P$.

\clearpage
\section{Brute Force}
Il metodo più semplice per risolvere il problema del pattern matching è il metodo \textit{brute force}. L'idea alla base di questo metodo è di confrontare il pattern $P$ con ogni possibile sottostringa di $T$ di lunghezza $m$. In particolare, per ogni indice $i$ da $0$ a $n-m$, confrontiamo la sottostringa $T[i..i+m-1]$ con il pattern $P$. Se troviamo una corrispondenza, restituiamo l'indice $i$.

\vspace{1\baselineskip}
\begin{lstlisting}
def find_brute(T, P):
    """Return the lowest index of T at which substring P begins (or else -1)."""
    n, m = len(T), len(P)  # introduce convenient notations
    for i in range(n-m+1):  # try every potential starting index within T
        k = 0  # an index into pattern P
        while k < m and T[i + k] == P[k]:  # kth character of P matches
            k += 1
        if k == m:  # if we reached the end of pattern,
            return i  # substring T[i:i+m] matches P
    return -1  # failed to find a match starting with any i
\end{lstlisting}
\vspace{1\baselineskip}


\subsection*{Performance}
L'algoritmo consiste in due cicli annidati, con il ciclo esterno che scorre tutti i possibili indici iniziali del pattern nel testo $T$, e il ciclo interno che scorre ogni carattere del pattern $P$, confrontandolo con il suo potenziale carattere corrispondente nel testo. Pertanto, la correttezza dell'algoritmo deriva direttamente da questo approccio di ricerca esaustiva.

Il tempo di esecuzione del pattern matching tramite \emph{Brute Force} nel caso peggiore non è buono poiché per ogni indice candidato in $T$, possiamo eseguire fino a $m$ confronti di caratteri per scoprire che $P$ non corrisponde a $T$ all'indice corrente. Dal blocco di codice si può osservare che il ciclo $for$ esterno viene eseguito al massimo $n-m+1$ volte e il ciclo $while$ interno viene eseguito al massimo $m$ volte. Pertanto, il tempo di esecuzione nel caso peggiore è $O(n m)$.

\clearpage
\subsection*{Esempio}
Supponiamo di avere un testo 
$$ T = \text{"abacaabaccabacabaabb"} $$
e un pattern 
$$ P = \text{"abacab"} $$

\begin{figure}[!ht]
    \centering
    \includegraphics[width=1\textwidth]{immagini/PatternMatching/ex_BruteForce.png}
    \caption{Esempio di Pattern Matching con algoritmo Brute Force. L'algoritmo esegue 27 confronti tra caratteri, numerati in figura.}
\end{figure}


\clearpage
\section{L'algoritmo di Boyer-Moore}
Come vedremo tra poco, non è sempre necessario confrontare ogni carattere del pattern con il testo. L'algoritmo di \emph{Boyer-Moore} sfrutta questa osservazione per saltare alcune posizioni nel testo, riducendo così il numero di confronti necessari. 

L'idea principale dell'algoritmo di \emph{Boyer-Moore} è di migliorare l'efficienza dell'algoritmo \emph{Brute Force} utilizzando due tecniche (euristiche) principali:

\begin{itemize}
    \item \textbf{Looking-Glass Heuristic}: Quando si confrontano i caratteri del pattern con il testo, si inizia dal carattere più a destra del pattern e si procede verso sinistra.
    \item \textbf{Character-Jump Heuristic}: Durante la verifica di un possibile piazzamento di $P$ in $T$, un mismatch tra $T[i] = c$ e $P[k]$ viene gestito come segue:
    
    Supponiamo che $T[i] \neq P[k]$ e $T[i] = c$.
        \begin{itemize}
            \item Se $c$ non appare in $p$, $p$ può essere "spostato" completamente oltre $T[i]$ ($P[0]$ viene allineato con $T[i+1]$).
            \item Altrimenti, $T[i]$ viene allineato con l'ultima occorrenza di $c$ in $P$.
        \end{itemize}
\end{itemize}

\begin{figure}[!ht]
    \centering
    \includegraphics[width=0.8\textwidth]{immagini/PatternMatching/ex_BoyerMoore.png}
    \caption{Una semplice dimostrazione dell'algoritmo di Boyer-Moore. Nel primo confronto si ha $T[4] \neq P[4]$ con $T[4] = \text{'e'}$ che non è presente in $P$, per cui spostiamo $P$ oltre $T[4]$. Nel secondo confronto si ha $T[9] \neq P[4]$ con $T[9] = \text{'s'}$ che è presente in $P$, in particolare l'ultima occorrenza di 's' è in $P[2]$, per cui allineiamo $P[2]$ con $T[9]$.}
    \label{exBoyerMoore}
\end{figure}

\begin{figure}[!ht]
    \centering
    \includegraphics[width=0.8\textwidth]{immagini/PatternMatching/ex2_BoyerMoore.png}
    \caption{Esmepio completo che mostra il numero di confronti.}
    \label{ex2BoyerMoore}
\end{figure}

\clearpage
\noindent
Per formalizzare l'algoritmo di \emph{Boyer-Moore} possiamo generalizzare il funzionamento come di seguito:

\begin{itemize}
    \item Quando viene trovata una corrispondenza (a partire dall'ultimo carattere del pattern), l'algoritmo continua cercando di estendere la corrispondenza con il penultimo carattere del pattern nel suo allineamento corrente. Questo processo continua fino a quando tutti i caratteri del pattern sono stati confrontati con esito positivo o fino a quando si verifica un mismatch.
    \item Quando si verifica un mismsatch, e il carattere del testo che ha causato il mismatch non è presente nel pattern, il pattern viene spostato completamente oltre quel carattere del testo. Se il carattere del testo è presente da qualche altra parte nel pattern, dobbiamo considerare due diversi casi a seconda che la sua ultima occorrenza sia $(a)$ precedente o $(b)$ successiva al carattere del pattern che era allineato con il carattere del testo che ha causato il mismatch.
\end{itemize}

Questi due casi sono rappresentati in figura \ref{fig:BM_mismatch_cases} e approfonfiti di seguito:




\cleardoublepage
\chapter{Tries}
\label{cap:Tries}

Il problema del \emph{pattern matching} presentato nel capitolo \ref{cap:PatternMatching} velocizza la ricerca in un testo effettuando una pre-elaborazione del pattern (per calcolare la funzione di fallimento nell'algoritmo di Knuth-Morris-Pratt o la funzione last nell'algoritmo di Boyer-Moore). In questa sezione, adottiamo un approccio complementare, presentando algoritmi di ricerca di stringhe che effettuano una pre-elaborazione del testo. Questo approccio è adatto per applicazioni in cui viene eseguita una serie di query su un testo fisso, in modo che il costo iniziale della pre-elaborazione del testo sia compensato da un'accelerazione in ogni query successiva. A questo proposito introduciamo i \emph{tries} (pronunciato "try"), una struttura dati basata su alberi per memorizzare stringhe al fine di supportare un rapido pattern matching. La principale applicazione dei tries è nel recupero delle informazioni, da cui il nome "trie" che deriva dalla parola "retrieval".


\section{Standard Tries}
Sia $S$ un insieme di $s$ stringhe su un alfabeto $\Sigma$ tale che nessuna stringa in $S$ sia prefisso di un'altra stringa. Uno \textbf{Standard Trie} per $S$ è un albero ordinato $T$ con le seguenti proprietà:
\begin{itemize}
    \item Ogni nodo di $T$, eccetto la radice, è etichettato con un carattere di $\Sigma$.
    \item I figli di un nodo interno di $T$ hanno etichette distinte.
    \item $T$ ha $s$ foglie, ciascuna associata a una stringa di $S$, tale che la concatenazione delle etichette dei nodi sul percorso dalla radice a una foglia $v$ di $T$ dia la stringa di $S$ associata a $v$.
\end{itemize}

\noindent
Dunque, un trie $T$ rappresenta le stringhe di $S$ tramite i percorsi dalla radice alle foglie di $T$. È importante assumere che nessuna stringa in $S$ sia prefisso di un'altra sintra poichè ciò garantisce che ogni stringa di $S$ sia univocamente associata a una foglia di $T$. Possiamo sempre soddisfare questa assunzione aggiungendo un carattere speciale che non è nell'alfabeto originale $\Sigma$ alla fine di ogni stringa.

\clearpage
\noindent
Uno standard trie che memorizza una collezione $S$ di $s$ stringhe di lunghezza totale $n$ da un alfabeto $\Sigma$ ha le seguenti proprietà:
\begin{itemize}[nosep]
    \item L'altezza di $T$ è uguale alla lunghezza della stringa più lunga in $S$.
    \item Ogni nodo interno di $T$ può avere da 1 a $|\Sigma|$ figli.
    \item $T$ ha $s$ foglie.
    \item Il numero di nodi di $T$ è al più $n+1$. 
        \begin{itemize}
            \item Infatti, il caso peggiore per il numero di nodi di un trie si verifica quando nessuna coppia di stringhe condivide un prefisso non vuoto; cioè, ad eccezione della radice, tutti i nodi interni hanno un solo figlio.
        \end{itemize}
\end{itemize}

\subsection*{Ricerca}
Un trie $T$ per un insieme $S$ di stringhe permette di implementare una mappa in cui le chiavi sono le stringhe stesse; la ricerca di una stringa $X$ avviene tracciando dalla radice il percorso indicato dai caratteri di $X$ e, se tale percorso termina in un nodo foglia, la stringa è presente nella mappa, mentre se il percorso si interrompe o termina in un nodo interno la stringa non è una chiave valida. 

Il tempo di esecuzione per cercare una stringa di lunghezza $m$ è limitato superiormente da $O(m \cdot |\Sigma|)$, in quanto possiamo visitare al più $m+1$ nodi di $T$ e spendiamo $O(|\Sigma|)$ tempo in ogni nodo per determinare il figlio che ha come etichetta il carattere successivo. Tuttavia, possiamo migliorare il tempo speso in un nodo a $O(\log |\Sigma|)$ o atteso $O(1)$, mappando i caratteri ai figli utilizzando una tabella hash in ogni nodo, oppure utilizzando una lookup table diretta di dimensione $|\Sigma|$ in ogni nodo, se $|\Sigma|$ è sufficientemente piccolo (come nel caso delle stringhe di DNA). Per questi motivi, ci aspettiamo tipicamente che una ricerca per una stringa di lunghezza $m$ venga eseguita in tempo $O(m)$.

\subsection*{Word Matching}
Grazie a queste caratteristiche, il trie è adatto per il \textbf{word matching} esatto e per le query sui prefissi (per via dell'operazione di ricerca appena descritta), ma non per il pattern matching di sottostringhe arbitrarie. Il \emph{word matching} è un caso particolare di pattern matching in cui vogliamo determinare se un dato pattern corrisponde esattamente a una delle parole del testo. Il word matching si differenzia dal pattern matching standard analizzato nel capitolo \ref{cap:PatternMatching} perché in questo caso il pattern non può corrispondere a una sottostringa arbitraria del testo, ma solo a una delle sue parole.

\subsection*{Costruzione di uno Standard Trie}
Per costruire uno standard trie per un insieme $S$ di stringhe, possiamo utilizzare un algoritmo incrementale che inserisce le stringhe una alla volta. Ricordiamo l'assunzione che nessuna stringa di $S$ sia prefisso di un'altra stringa. Per inserire una nuova stringa $X$ nel trie, seguiamo il percorso dei suoi caratteri finché esistono nodi corrispondenti. Nel punto in cui il percorso esistente si interrompe (ovvero non troviamo il carattere successivo), creiamo una nuova sequenza di nodi per tutti i caratteri restanti di $X$. Il tempo di esecuzione per inserire $X$ di lunghezza $m$ è simile a una ricerca, con prestazioni nel caso peggiore di $O(m \cdot |\Sigma|)$, o attese $O(m)$ se si utilizzano tabelle hash secondarie in ogni nodo. Pertanto, la costruzione dell'intero trie per l'insieme $S$ richiede un tempo atteso di $O(n)$, dove $n$ è la lunghezza totale delle stringhe di $S$.

\vspace{1\baselineskip}
\begin{figure}[!ht]
    \centering
    \includegraphics[width=1\textwidth]{immagini/Tries/std_tries_ex.png}
    \caption{Word Matching tramite uno Standard Trie: (a) testo da cercare (articoli e preposizioni, noti anche come stop words, esclusi); (b) standard trie per le parole nel testo, con le foglie che mantengono l'informazione relativa all'indice in cui la data parola inizia nel testo. Ad esempio, la foglia per la parola \emph{stock} indica che la parola inizia agli indici 17, 40, 51 e 62 del testo.}
    \label{fig:std_tries_ex}
\end{figure}

\vspace{1\baselineskip}
Come si può notare anche dall'esempio in Figura \ref{fig:std_tries_ex}, c'è una potenziale inefficienza di spazio nello standard trie dovuta alla presenza di molti nodi che hanno un solo figlio, e l'esistenza di tali nodi rappresenta uno spreco. La ricerca di una soluzione a questo problema ha portato allo sviluppo del \emph{Compressed Trie}, o \emph{Patricia Trie}.  


\clearpage
\section{Compressed Tries}
Un \textbf{Compressed Trie} è una variante dello standard trie che garantisce che ogni nodo interno del trie abbia almeno due figli. Questa regola viene applicata comprimendo le catene di nodi con un solo figlio. 

Sia $T$ un trie standard. Diciamo che un nodo interno $v$ di $T$ è \textbf{ridondante} se $v$ ha un solo figlio e non è la radice. Diciamo anche che una \emph{catena} di $k \geq 2$ archi,
$$ (v_0, v_1)(v_1, v_2) \cdots (v_{k-1}, v_k) $$
è \textbf{ridondante} se:
\begin{itemize}[nosep]
    \item $v_i$ è ridondante per $i = 1, \ldots, k-1$.
    \item $v_0$ e $v_k$ non sono ridondanti.
\end{itemize}

\noindent
Si può trasformare $T$ in un \emph{Compressed Trie} sostituendo ogni catena ridondante $(v_0,v_1) \cdots (v_{k-1},v_k)$ di $k \geq 2$ archi con un singolo arco $(v_0,v_k)$, rietichettando $v_k$ con la concatenazione delle etichette dei nodi $v_1, \ldots, v_k$.

\begin{figure}[!ht]
    \centering
    \includegraphics[width=1\textwidth]{immagini/Tries/compressed_tries_ex.png}
    \caption{Compressed Trie per le stringhe \{bear, bell, bid, bull, buy, sell, stock, stop\}. (Confronta questo con lo standard trie mostrato in Figura \ref{fig:std_tries_ex}.) Oltre alla compressione ai nodi foglia, nota il nodo interno con etichetta "to" condivisa dalle parole "stock" e "stop".}
    \label{fig:compressed_tries_ex}
\end{figure}

I nodi di un compressed trie sono etichettati con delle stringhe, che sono sottostringhe delle stringhe nella collezione, piuttosto che con singoli caratteri. Il vantaggio di un compressed trie rispetto a uno standard trie è che il numero di nodi del compressed trie è proporzionale al numero di stringhe e non alla loro lunghezza totale.

\vspace{1\baselineskip}
\noindent
Un compressed trie che memorizza una collezione $S$ di $s$ stringhe in un alfabeto di dimensione $d$ ha le seguenti proprietà:

\begin{itemize}
    \item Ciascun nodo interno di $T$ ha almeno due figli e al massimo $d$ figli.
    \item $T$ ha $s$ nodi foglia.
    \item Il numero di nodi di $T$ è $O(s)$.
\end{itemize}

\clearpage
\noindent
Il \emph{Compressed Trie} offre un reale vantaggio quando viene utilizzato come struttura di indicizzazione ausiliaria per una collezione di stringhe già memorizzate in una struttura primaria.

Supponiamo, ad esempio, che la collezione $S$ di stringhe sia un array di stringhe $S[0], S[1], \ldots, S[s-1]$. Invece di memorizzare esplicitamente l'etichetta $X$ di un nodo, la rappresentiamo implicitamente mediante una combinazione di tre interi $(i, j : k)$, tali che $X = S[i][ j : k]$; cioè, $X$ è la porzione di $S[i]$ costituita dai caratteri dalla posizione $j$ fino a, ma non includendo, la posizione $k$.

\begin{figure}[!ht]
    \centering
    \includegraphics[width=0.9\textwidth]{immagini/Tries/compressed_tries_usage.png}
    \caption{(a) Collezione $S$ di stringhe memorizzata in un array. (b) Rappresentazione compatta del compressed trie per $S$.}
    \label{fig:compressed_tries_usage}    
\end{figure}

\noindent
In questo modo è possibile ridurre lo spazio totale per il trie stesso da $O(n)$ per lo standard trie a $O(s)$ per il compressed trie, dove $n$ è la lunghezza totale delle stringhe in $S$ e $s$ è il numero di stringhe in $S$. Naturalmente, dobbiamo comunque memorizzare le diverse stringhe in $S$, ma riduciamo comunque lo spazio per il trie.

La ricerca in un compressed trie non è necessariamente più veloce che in un albero standard, poiché è ancora necessario confrontare ogni carattere del pattern desiderato con le etichette, potenzialmente multi-carattere, durante la traversata dei percorsi nel trie.




\cleardoublepage
\chapter{Greedy Algorithms}
\label{cap:Greedy}

L'\emph{approccio Greedy} è un paradigma di progettazione di algoritmi utilizzato per risolvere \textbf{problemi di ottimizzazione}. Un algoritmo greedy funziona bene quando una soluzione ottimale può essere raggiunta attraverso una \emph{serie di scelte locali ottimali}. A partire da una \emph{configurazione iniziale} (soluzione parziale), l'algoritmo effettua una scelta (la migliore possibile localmente) in una classe di possibili opzioni, e ripete lo stesso procedimento aggiornando di volta in volta la configurazione corrente, fino a raggiungere una soluzione completa.  

In particolare, non possiamo utilizzare questo approccio per tutti i problemi di ottimizzazioni. Diciamo che un problema di ottimizzazione ammette una \textbf{soluzione greedy} se il problema soddisfa la proprietà:
\begin{itemize}
    \item \textbf{Greedy-choice property}: la soluzione completa ottimale può sempre essere raggiunta effettuando una serie di progressi, che rappresentano delle scelte locali ottimali, a partire da una configurazione iniziale.
\end{itemize}

\noindent
Ad ogni passo, la "scelta" deve essere:
\begin{itemize}
    \item \textbf{Realizzabile}: deve soddisfare i vincoli dettati dal problema.
    \item \textbf{Localmente ottima}: deve essere la scelta migliore tra tutte le scelte possibili (e realizzabili) in quel momento.
    \begin{itemize}[nosep]
        \item Non è necessario che questa scelta sia ottima rispetto alla soluzione globale.
        \item Da notare che non è sempre detto che effettuare scelte localmente ottimali porti ad una soluzione globale ottimale. In questi casi, l'algoritmo greedy non funziona correttamente.
    \end{itemize}
    \item \textbf{Irrevocabile}: una volta effettuata una scelta, questa non può essere modificata in futuro.
\end{itemize}

\noindent
Un algoritmo greedy costruisce una soluzione in piccoli passi successivi, scegliendo ad ogni passo una decisione che riguarda esclusivamente la configurazione corrente. Spesso si possono progettare molti algoritmi greedy diversi per lo stesso problema, ognuno dei quali ottimizza localmente e in modo incrementale qualche misura diversa nel suo cammino verso una soluzione. 
È facile inventare algoritmi greedy per quasi tutti i problemi; trovare i casi in cui funzionano bene, e dimostrare che effettivamente funzionano bene, è la sfida interessante.



\clearpage
\section{Un modello generale}
Indichiamo con $S$ la soluzione parziale corrente, e con $P$ il sotto problema che rimane da risolvere. Inizialmente, $S$ è vuota e $P$ coincide con il problema originale. Ad ogni passo, l'algoritmo greedy:

\vspace{1\baselineskip}
\hrule

\begin{verbatim}
1. Generate all candidate choices as list L for current sub-problem P.
2. While (L is not empty or other finish condition is not met)
3.     Compute the feasible value of each choice in L;
4.     Modify S and P by taking choice with the highest feasible value;
5.     Update L according to S and P;
6. Endwhile
7. Return the resulting complete solution.
\end{verbatim}

\hrule 
\vspace{1\baselineskip}

\noindent
Sia $A$ l'insieme di tutti i possibili elementi del problema. Ad ogni step, l'algoritmo greedy mantiene una partizione $<X, Y, W>$ di $A$, dove:
\begin{itemize}[nosep]
    \item $X$: insieme degli elementi selezionati fino a quel momento (soluzione parziale corrente).
    \item $Y$: insieme degli elementi valutati ma non ancora selezionati.
    \item $W$: insieme degli elementi non ancora valutati.
\end{itemize}
Inizialmente, $W = A$ e $X = Y = \emptyset$. Alla fine dell'algoritmo, $X$ conterrà la soluzione completa, $Y = A/X$ contiene tutti gli elementi di $A$ che non sono stati selezionati, e $W = \emptyset$.

\vspace{1\baselineskip}
\noindent
Gli algoritmi greedy sono spesso \textbf{estremamente intuitivi} al tal punto da rappresentare la soluzione più semplice e naturale per molti problemi. Sono anche molto \textbf{efficienti}, con complessità temporali che vanno da $O(n \log n)$ a $O(n)$, dove $n$ è la dimensione dell'input. Tuttavia, la loro efficienza dipende fortemente dalla natura del problema e dalla struttura dei dati utilizzati per implementare l'algoritmo.

La parte più complicata nella progettazione di un algoritmo greedy è la \textbf{dimostrazione della correttezza} dell'algoritmo, che spesso richiede tecniche di dimostrazione specifiche per ogni problema. Le tecniche più comuni sono:
\begin{itemize}
    \item \textbf{Greedy stays ahead}: si dimostra che, ad ogni passo dell'algoritmo, la soluzione parziale costruita dall'algoritmo greedy è almeno altrettanto buona quanto qualsiasi altra soluzione parziale possibile.
    \item \textbf{Exchange}: si dimostra che qualsiasi soluzione ottimale può essere trasformata nella soluzione prodotta dall'algoritmo greedy attraverso una serie di scambi di elementi, senza peggiorare la qualità della soluzione.
\end{itemize}


\clearpage
\section{Coin Change: The Cashier Algorithm}
Il problema del \emph{Coin Change} (cambio di monete) consiste nel trovare il numero minimo di monete necessarie per rappresentare un dato importo di denaro, utilizzando un insieme predefinito di tagli di monete. 

Il problema prende in input un importo R (in centesimi di euro) e richiede in output il numero minimo di monete necessarie per rappresentare tale importo, utilizzando solo, ad esempio, monete di taglio 2€, 1€, 50c, 20c, 10c, 5c, 2c e 1c. Si assume di avere a disposizione un numero illimitato di monete per ogni taglio.

\vspace{1\baselineskip}
\hrule

\begin{verbatim}
Sort coins denominations by value: c[1] > c[2] > ... > c[k] 
1. S = {}           // inizializza soluzione parziale (insieme vuoto)
2. while (x != 0) { // finché l'importo da cambiare non è zero
3.    let k be the largest integer such that c[k] <= x
4.    if (k = 0)
5.        return "no solution found"
6.    x = x - c[k]  // riduci l'importo rimanente
7.    S = {S, c[k]} // aggiungi c[k] alla soluzione
8. }
9. return S         // restituisci la soluzione completa
\end{verbatim}

\hrule 
\vspace{1\baselineskip}

\begin{lstlisting}
def coin_change(amount_rem):
    coin_combinations = [50, 20, 10, 5, 2, 1]   # Valori in centesimi
    result = []                  

    for coin in coin_combinations:
        coin_count = amount_rem // coin # Divisione intera per ottenere il numero massimo di monete di questo taglio
        result += [coin] * coin_count   # Aggiungi le monete alla soluzione
        amount_rem -= coin * coin_count # Aggiorna l'importo rimanente
        if amount_rem == 0:
            return result       # Restituisci la soluzione completa

    if amount_rem > 0:      # Se non e' stato possibile coprire l'importo
        return "No solution found"
\end{lstlisting}
\vspace{1\baselineskip}

\noindent
In generale, questo algoritmo restituisce sempre una soluzione (non necessariamente ottimale) se $c[k] = 1$ (cioè se esiste una moneta di taglio unitario). Tuttavia, l'algoritmo è ottimale solo per alcuni sistemi di monete specifici, come quello europeo, i cosiddetti \textbf{sistemi canonici}. 

Un insieme di monete è un \textbf{sistema canonico} se l'algoritmo del cassiere fornisce la soluzione ottimale per ogni importo $R$ da cambiare. In generale, non tutti i sistemi di monete sono canonici e determinare se un sistema di monete è canonico può essere un problema complesso.


\clearpage
\section{Scheduling}
Il problema dello \emph{scheduling} riguarda la pianificazione di un insieme di attività o compiti su risorse limitate, come macchine, lavoratori o tempo. L'obiettivo è ottimizzare l'uso delle risorse per massimizzare l'efficienza, minimizzare i tempi di completamento o soddisfare altre metriche di performance. Le attività possono avere vincoli di tempo, priorità diverse e requisiti specifici, rendendo il problema complesso e variegato.

In generale, dato un insieme di $n$ attività ognuna con un \emph{tempo di inizio} $s_i$ e un \emph{tempo di fine} $f_i$ (con $s_i < f_i$), si chiede di:

\begin{itemize}
    \item \textbf{Task Scheduling}: Minimizzare il numero di macchine (in generale risorse) necessarie per completare tutte le attività senza sovrapposizioni.
    \item \textbf{Interval Scheduling}: Massimizzare il numero di attività che possono essere completate senza sovrapposizioni, utilizzando una singola macchina.
\end{itemize}

\subsection{Task Scheduling}
In questa tipologia di problemi, abbiamo delle risorse identiche limitate (useremo il termine "macchine" per semplicità) e desideriamo assegnare un insieme di attività a queste macchine in modo tale che nessuna attività si sovrapponga temporalmente su una stessa macchina. L'obiettivo è minimizzare il numero di macchine utilizzate per completare tutte le attività.

\begin{figure}[!ht]
    \centering
    \includegraphics[width=0.99\textwidth]{immagini/Greedy/ex_task_scheduling.png}
    \caption{(a) Una istanza del problema di Task Scheduling, in cui le risorse sono delle aule e le attività sono lezioni da assegnare. (b) Una soluzione in cui tutte le attività sono pianificate utilizzando tre risorse: ogni riga rappresenta un insieme di attività che possono essere tutte pianificate su una singola risorsa.}
    \label{ex:task_scheduling}
\end{figure}

\clearpage
\subsection*{Soluzione Greedy}
Una soluzione greedy molto intuitiva per questo problema consiste nell'ordinare le attività in ordine crescente di tempo di inizio. Successivamente, si itera attraverso l'elenco delle attività e si assegna ciascuna attività alla prima macchina disponibile che non abbia conflitti di orario con le attività già assegnate. Se nessuna macchina è disponibile, si aggiunge una nuova macchina.

\vspace{1\baselineskip}
\hrule

\begin{verbatim}
Sort intervals by starting time so that s[1] <= s[2] <= ... <= s[n]
1.  d = 0    // inizializza il numero di macchine
2.  for j = 1 to n {
3.      if (task j is compatible with some machine k)
4.          schedule task j on machine K
5.      else 
6.          allocate a new machine d + 1
7.          schedule task j on machine d + 1
8.          d = d + 1
9.  }
10. return d  // restituisci il numero di macchine utilizzate
\end{verbatim}

\hrule
\vspace{2\baselineskip}

\noindent
L'algoritmo ha complessità complessiva $O(n \log n)$, dominata
dall'ordinamento iniziale delle attività per tempo di inizio.

Nella fase di assegnazione si utilizza una \emph{Min-Priority Queue} (min-heap) contenente, per ciascuna macchina, il tempo di fine dell'ultimo task eseguito. In questo modo la radice del heap rappresenta sempre la macchina che si libera per prima.

Per ogni attività $j$, processata in ordine crescente di $s[j]$, è sufficiente confrontare $s[j]$ con il minimo del heap:

\begin{itemize}
    \item se $s[j] \ge f_{\min}$, la macchina si è liberata: si estrae il minimo e si inserisce il nuovo tempo di fine ($O(\log n)$);
    \item altrimenti, tutte le macchine sono occupate: si alloca una nuova macchina e si inserisce il suo tempo di fine nel heap
          ($O(\log n)$).
\end{itemize}

\noindent
Poiché ogni attività effettua al più un'estrazione e un'inserzione nel heap, la fase di scansione richiede $O(n \log n)$, in linea con il costo dell'ordinamento.


\clearpage
\subsection{Interval Scheduling}
In questa tipologia di problemi, abbiamo una singola macchina e desideriamo selezionare un sottoinsieme di attività da eseguire su questa macchina in modo tale che nessuna attività si sovrapponga con un'altra. L'obiettivo è massimizzare il numero di attività completate senza sovrapposizioni.

\begin{figure}[!ht]
    \centering
    \includegraphics[width=1\textwidth]{immagini/Greedy/ex_interval_scheduling.jpg}
    \caption{Una istanza del problema di Interval Scheduling in cui si hanno 8 attività da collocare su una singola macchina.}
    \label{ex:interval_scheduling}
\end{figure}

\subsection*{Soluzione Greedy}
Nel cercare una soluzione ottimale per questo problema, possiamo considerare diverse strategie greedy intuitive, e selezionare la prima attività compatibile con quelle già scelte. Alcune possibili strategie includono:
\begin{itemize}
    \item Ordinamento per tempo di inizio crescente. Seleziono le attività compatibili in ordine di inizio $s[i]$. 
    \item \textbf{Ordinamento per tempo di fine crescente}. Seleziono le attività compatibili in ordine di fine $f[i]$.
    \item Ordinamento per durata crescente. Seleziono le attività compatibili in ordine di durata $f[i] - s[i]$.
    \item Ordinamento per numero di conflitti crescente. Per ogni attività $i$, conto il numero $c_i$ di altre attività che non sono compatibili con $i$ (cioè che si sovrappongono temporalmente con $i$). Seleziono le attività compatibili in ordine crescente di $c_i$.
\end{itemize}

\clearpage
\begin{figure}[!ht]
    \centering
    \includegraphics[width=1\textwidth]{immagini/Greedy/interval_scheduling_counterex.png}
    \caption{Controesempi per le strategie greedy di Interval Scheduling basate su (a) tempo di inizio crescente, (b) durata crescente, (c) numero di conflitti crescente.}
    \label{ex:interval_scheduling_counterex}
\end{figure}

\noindent
Tra queste strategie, solo l'ordinamento per tempo di fine crescente garantisce una soluzione ottimale per il problema di Interval Scheduling. Le altre strategie possono portare a soluzioni subottimali, come mostrato nei controesempi della Figura \ref{ex:interval_scheduling_counterex}. Dal momento che per confutare una strategia greedy è sufficiente trovare un singolo controesempio, possiamo concludere che l'unica strategia ottimale tra quelle elencate è quella basata sul tempo di fine crescente (ci siamo limitati a confutare le strategie mostrate in Figura \ref{ex:interval_scheduling_counterex}, ma per essere certi che la strategia basata sul tempo di fine crescente sia ottimale, sarebbe necessario dimostrarne la correttezza).

\vspace{1\baselineskip}
\hrule

\begin{verbatim}
Sort intervals by finishing time so that f[1] < f[2] < ... < f[n]
1.  n = s.length    // number of activities
2.  A = {a[1]}      // initialize solution with first activity
3.  k = 1           // index of last activity added to A
4.  for m=2 to n {
5.      if (s[m] >= f[k])
6.          A = {A, a[m]} // add activity a[m] to A
7.          k = m   // update index of last activity added to A
8.  }
9.  return A        // return the set of accepted activities 
\end{verbatim}

\hrule
\vspace{2\baselineskip}

\noindent
L'algoritmo ha complessità complessiva $O(n \log n)$, dominata
dall'ordinamento iniziale delle attività per tempo di fine. L'algoritmo sfrutta una semplice \emph{scansione lineare} delle attività ordinate, selezionando ogni volta la prima attività compatibile con l'ultima selezionata.


\clearpage
\section{Fractional Knapsack}




\cleardoublepage
\chapter{Dynamic Programming}
\label{cap:DynamicProgramming}

La \emph{programmazione dinamica} è una tecnica di progettazione di algoritmi per la risoluzione di problemi di ottimizzazione (così come l'approccio Greedy discusso nel capitolo precedente). 

Questa tecnica è simile alla tecnica \emph{divide-et-impera}, e proprio per questo motivo può essere applicata a una vasta gamma di problemi differenti. La programmazione dinamica può spesso essere utilizzata per trasformare problemi che sembrano richiedere un tempo esponenziale in algoritmi che li risolvono in tempo polinomiale. Inoltre, gli algoritmi che risultano dall'applicazione della tecnica di programmazione dinamica sono solitamente piuttosto semplici a livello concettuale.


\section{Modello generale}
Quando si progetta un algoritmo di programmazione dinamica è importante seguire in ordine i seguenti passi:

\begin{enumerate}
    \item Definire i \textbf{sotto-problemi}.
    \item Definire come la soluzione ottimale può essere ottenuta dalle soluzioni ottimali dei sotto-problemi ... e, ricorsivamente, come la soluzione ottimale di un sotto-problema può essere ottenuta dalla soluzione ottimale dei suoi sotto-problemi.
    \item Descrivere la soluzione ottimale attraverso un'\textbf{equazione caratteristica}.
    \begin{itemize}[nosep]
        \item Questa relazione lega la soluzione globale con le soluzioni dei sotto-problemi, e definisce inoltre i \emph{casi base} (\textbf{boundary conditions}), ovvero quei sotto-problemi le cui soluzioni sono banali e immediate, fondamentali per risolvere i sotto-problemi più grandi.
    \end{itemize}
\end{enumerate}

\noindent
È fondamentale scegliere i sotto-problemi in modo "intelligente", così come vedremo nei prossimi esempi, in modo da facilitare la definizione dell'equazione caratteristica e la risoluzione dei sotto-problemi stessi.


\clearpage
La programmazione dinamica è per alcuni aspetti simile alla tecnica \emph{divide-et-impera}. Sebbene entrambe le tecniche suddividano il problema originale in porzioni più piccole, la differenza sostanziale risiede nella relazione tra questi sotto-problemi.
Nel paradigma \emph{divide-et-impera}, i sotto-problemi sono \textbf{indipendenti} (o disgiunti): la soluzione di un ramo non influenza né è necessaria per la risoluzione degli altri. 
Al contrario, la \emph{programmazione dinamica} è applicabile quando i sotto-problemi sono \textbf{sovrapposti} (\emph{overlapping subproblems}), ovvero quando condividono a loro volta dei sotto-problemi comuni.

Mentre un approccio \emph{divide-et-impera} puro ricalcolerebbe più volte la soluzione per lo stesso sotto-problema condiviso (portando spesso a una complessità esponenziale), la programmazione dinamica risolve ogni sotto-problema una sola volta e ne memorizza il risultato (\emph{memoization} o tabulazione) per riutilizzarlo in futuro, garantendo così l'efficienza polinomiale.

\begin{figure}[!ht]
    \centering
    \includegraphics[width=0.75\textwidth]{immagini/Dynamic/pd_vs_dei.png}
    \caption{Esempio di risoluzione del problema della \emph{sequenza di Fibonacci} con divide-et-impera. Questa soluzione non è ottimale dal punto di vista computazionale, in quanto uno stesso sotto-problema viene risolto più volte (ad esempio, in figura f(6) viene calcolato 4 volte).}
    \label{fig:pd_vs_dei}
\end{figure}

\noindent
Una possibile soluzione che sfrutta la programmazione dinamica:
\begin{lstlisting}
def fibonacci(n):
    # Gestione caso base immediato
    if n <= 1:
        return n
    # Creazione della tabella (array) per memorizzare irisultati
    # Inizializzata a 0, dimensione n+1 per ospitarel'indice n
    table = [0] * (n + 1)
    # Impostazione dei casi base noti
    table[0] = 0
    table[1] = 1
    # Riempimento della tabella dal basso verso l'alto
    for i in range(2, n + 1):
        table[i] = table[i-1] + table[i-2]
    return table[n]
\end{lstlisting}


\clearpage
\section{Longest Common Subsequence (LCS)}
Una \emph{sottosequenza} di una stringa $x_0x_1x_2 \ldots x_{n-1}$ è una stringa $x_{i_0}x_{i_1} \ldots x_{i_k}$, dove $i_j < i_{j+1}$. In altre parole, una sottosequenza si ottiene eliminando alcuni caratteri dalla stringa originale senza cambiare l'ordine dei caratteri rimanenti.

Da notare che è diverso dal concetto di \emph{sottostringa}: una sottostringa è una sequenza contigua di caratteri all'interno della stringa originale, mentre una sottosequenza può essere formata da caratteri non contigui. Per cui, una sottostringa è sempre una sottosequenza, ma non viceversa.
\textbf{Esempio}: Data la stringa "AGGTAB", alcune delle sue sottosequenze sono "GTA", "ATAB", "GAB", mentre alcune delle sue sottostringhe sono "AGG", "GGTA", "TAB".

\vspace{1\baselineskip}
\noindent
Uno dei problemi classici che può essere risolto in modo efficiente tramite la programmazione dinamica è il problema della \emph{Longest Common Subsequence} (LCS), ovvero la ricerca della sottosequenza comune più lunga tra due sequenze date.

\begin{itemize}[nosep]
    \item Date due stringhe $X = x_0 x_1 \ldots x_{n-1}$ e $Y = y_0 y_1 \ldots y_{m-1}$ su di un alfabeto $\Sigma$, trovare la stringa più lunga che è una sottosequenza di entrambe le stringhe.
\end{itemize}

\vspace{2\baselineskip}
\subsection*{Soluzione Brute-Force}
Tutte le possibili sottosequenze di una stringa $X$ di lunghezza $n$ sono $2^n$. Quindi, un approccio brute-force per risolvere il problema LCS sarebbe generare tutte le sottosequenze di $X$, e per ognuna di esse verificare se è anche una sottosequenza di $Y$, tenendo traccia della più lunga trovata. Questo approccio ha una complessità temporale esponenziale di $O(2^n \cdot m)$, dove $m$ è la lunghezza della stringa $Y$.

\clearpage
\subsection*{Soluzione Dynamic Programming}
Sia $L_{n,m}$ la lunghezza della LCS tra le stringe $X = x_0 x_1 \ldots x_{n-1}$ e $Y = y_0 y_1 \ldots y_{m-1}$. $L_{n,m}$ rappresenta quindi la soluzione ottimale del problema.

\subsubsection*{Scomposizione in sotto-problemi}
Sia $L_{j,k}$ la lunghezza della LCS tra i prefissi $x_0 x_1 \ldots x_{j-1}$ e $y_0 y_1 \ldots y_{k-1}$, con $0 \leq j \leq n$ e $0 \leq k \leq m$. $L_{j,k}$ rappresenta quindi la soluzione ottimale al sotto-problema che considera solo i primi $j$ caratteri di $X$ e i primi $k$ caratteri di $Y$.

Osserviamo l'ultimo carattere di entrambe le stringhe considerate nello specifico sotto-problema, e da qui ricaviamo le due \textbf{equazioni caratteristiche} che ci permettono di esprimere $L_{j,k}$ in funzione dei sotto-problemi più piccoli:
\begin{itemize}[nosep]
    \item Se $x_{j-1} = y_{k-1}$, allora questo carattere fa parte della LCS, e possiamo scrivere:
    $$L_{j,k} = 1 + L_{j-1,k-1}$$
    \item Se $x_{j-1} \neq y_{k-1}$, allora l'ultimo carattere di almeno una delle due stringhe non fa parte della LCS, e possiamo scrivere:
    $$L_{j,k} = \max(L_{j-1,k}, L_{j,k-1})$$
\end{itemize}

\begin{figure}[!ht]
    \centering
    \includegraphics[width=0.9\textwidth]{immagini/Dynamic/lcs_equazioni.png}
    \caption{Rappresentazione grafica delle equazioni caratteristiche per il calcolo di $L_{j,k}$.}
    \label{fig:lcs_equazioni}
\end{figure}

\subsubsection*{Boundary Conditions}
Il caso base si verifica quando una delle due stringhe è vuota, ovvero quando $j = 0$ o $k = 0$. In questi casi, la LCS è anch'essa vuota, quindi:
$$L_{0,k} = 0 \quad \text{per } 0 \leq k \leq m$$
$$L_{j,0} = 0 \quad \text{per } 0 \leq j \leq n$$

\noindent
Notiamo che la soluzione $L_{j,k}$ appare nella computazione di:
$$ L_{j+1,k} \quad L_{j,k+1} \quad L_{j+1,k+1} $$
per cui i sotto-problemi si sovrappongono, rendendo la programmazione dinamica una tecnica adatta per risolvere questo problema. Quindi, anziché utilizzare un approccio ricorsivo che ricalcola più volte gli stessi sotto-problemi (divide-et-impera), possiamo utilizzare una tabella bidimensionale per memorizzare i risultati dei sotto-problemi già calcolati. 

\subsubsection*{Calcolo della tabella}
La tabella avrà dimensioni $(n+1) \times (m+1)$, dove l'elemento nella cella $(j,k)$ conterrà il valore di $L_{j,k}$, ovvero la LCS tra i primi $j$ caratteri di $X$ e i primi $k$ caratteri di $Y$. Inizializziamo la prima riga e la prima colonna della tabella con i valori dei casi base, e poi riempiamo la tabella utilizzando le equazioni caratteristiche definite sopra.

Anche l'ordine in cui viene riempita la tabella è importante: in questo caso (e in molti altri, ma dipende dal problema specifico) possiamo procedere per righe, in quanto la computazione di $L_{j,k}$ dipende solo dai valori presenti nelle equazioni caratteristiche, ovvero:
\begin{itemize}[nosep]
    \item $L_{j-1,k-1}$ (riga e colonna precedenti).
    \item $L_{j-1,k}$ (riga precedente e stessa colonna).
    \item $L_{j,k-1}$ (stessa riga e colonna precedente).
\end{itemize}

\vspace{1\baselineskip}
\begin{lstlisting}
def LCS(X, Y):
""" Returns table such that L[j][k] is length of LCS for X[0:j] and Y[0:k] """
n, m = len(X), len(Y)   
L = [[0] * (m + 1) for k in range(n + 1)]   # (n+1) x (m+1) table
for j in range(1, n + 1):           # j from 1 to n
    for k in range(1, m + 1):       # k from 1 to m
        if X[j-1] == Y[k-1]:        # match
            L[j][k] = 1 + L[j-1][k-1]
        else:                       # no match
            L[j][k] = max(L[j-1][k], L[j][k-1])
return L
\end{lstlisting}
\vspace{1\baselineskip}

\noindent
L'algoritmo che si occupa di calcolare la tabella $L$ impiega due cicli annidati, iterando rispettivamente su $j$ e $k$. All'interno del ciclo più interno, viene eseguita una semplice operazione di confronto e un'assegnazione, entrambe con complessità $O(1)$, quindi la complessità totale dell'algoritmo è $O(n \cdot m)$, dove $n$ e $m$ sono le lunghezze delle stringhe $X$ e $Y$ rispettivamente. 


\subsubsection*{Estrazione della soluzione}
Come già detto, la tabella $L$ contiene le lunghezze delle LCS per tutti i sotto-problemi, ma non restituisce direttamente la LCS stessa. Tuttavia, è possibile ricostruire la LCS a partire dalla tabella $L$. La soluzione può essere ricavata, a partire dalla cella $L[n][m]$ nel modo seguente:

\begin{itemize}[nosep]
    \item Considerando una generica cella $L[j][k]$:
    \begin{itemize}[nosep]
        \item Se $x_{j-1} = y_{k-1}$, significa che questo carattere comune ha contribuito alla lunghezza della sottosequenza basandosi sul valore precedente $L_{j-1,k-1}$. Possiamo quindi registrare $x_{j-1}$ come parte della soluzione e proseguire l'analisi dalla cella $L_{j-1,k-1}$.
        \item Se $x_{j-1} \neq y_{k-1}$, allora la lunghezza della LCS dipende dal massimo tra $L[j][k-1]$ e $L[j-1][k]$. In questo caso, dobbiamo spostarci nella direzione del massimo valore per continuare la ricerca della LCS.
        \item Continuiamo questo processo fino a raggiungere una cella $L[j][k] = 0$.
    \end{itemize}
\end{itemize}


\vspace{1\baselineskip}
\begin{lstlisting}
def LCS_solution(X, Y, L):
""" Returns the LCS of X and Y, given LCS table L """
solution = []
j, k = len(X), len(Y)
while L[j][k] > 0:          # common characters remain
    if X[j-1] == Y[k-1]:          # match
        solution.append(X[j-1])   # add to solution
        j -= 1
        k -= 1
    elif L[j-1][k] >= L[j][k-1]:  # no match
        j -= 1
    else:
        k -= 1
return ''.join(reversed(solution))  # return left-to-right LCS
\end{lstlisting}
\vspace{1\baselineskip}

\noindent
L'algoritmo per ricostruire la LCS ha una complessità temporale di $O(n + m)$, poiché in ogni iterazione del ciclo while si decrementa almeno uno tra $j$ e $k$, e il ciclo termina quando uno dei due raggiunge zero.

\begin{figure}[!ht]
    \centering
    \includegraphics[width=1\textwidth]{immagini/Dynamic/lcs_table.png}
    \caption{Illustrazione dell'algoritmo per la costruzione di una longest common subsequence a partire dall'array L. Un passo diagonale da $L_{j,k}$ a $L_{j-1,k-1}$ sul percorso evidenziato rappresenta l'uso di un carattere comune, ovvero il carattere $c = x_{j-1} = y_{k-1}$.}
    \label{fig:lcs_table}
\end{figure}


\clearpage
\section{Edit Distance}

\cleardoublepage
\chapter{Local Search}
\label{cap:LocalSearch}

La \textbf{Local Search} (Ricerca Locale) rappresenta una delle tecniche fondamentali per la risoluzione di problemi di ottimizzazione complessi. A differenza degli algoritmi visti in precedenza, come l'approccio \textit{Greedy} o la \textit{Programmazione Dinamica}, che costruiscono la soluzione da zero, la ricerca locale opera su soluzioni complete. L'idea centrale è partire da una soluzione completa iniziale (spesso generata casualmente o ricavata da semplici euristiche) e migliorarla iterativamente esplorando un "intorno" locale di soluzioni simili.

L'ottimizzazione di un problema può essere rappresentata graficamente come una curva o superficie in cui ogni punto corrisponde ad una soluzione del problema e la sua “altezza” rappresenta il costo associato a quella soluzione. Il problema diventa quello di individuare il punto più basso possibile, ovvero il minimo costo per quel problema di ottimizzazione.

\begin{figure}[!ht]
    \centering
    \includegraphics[width=0.3\textwidth]{immagini/LocalSearch/curva1.png}
    \caption{Rappresentazione grafica della curva delle soluzioni in un problema di ottimizzazione.}
    \label{fig:local_search_landscape}
\end{figure}

\noindent
Formalmente, consideriamo un problema di ottimizzazione definito da:
\begin{itemize}
    \item $C$: l'insieme delle soluzioni ammissibili.
    \item $c$: una funzione di costo che associa ad ogni soluzione $S \in C$ un valore reale $c(S)$.
    \item $N$: una funzione che definisce l'intorno di una soluzione, ovvero l'insieme delle soluzioni "vicine" a $S$, denotato come $N(S) \subseteq C$.
\end{itemize}

\noindent
Graficamente, possiamo immaginare lo spazio delle soluzioni come un paesaggio: le soluzioni sono le coordinate e il costo è l'altitudine. L'algoritmo cerca di scendere verso il punto più basso (la valle più profonda).

\clearpage
\noindent
A questo proposito, una distinzione cruciale è quella tra ottimi locali e globali:
\begin{itemize}
    \item Un \textbf{Minimo Globale} è una soluzione $S^*$ con costo minimo assoluto su tutto $C$.
    \item Un \textbf{Minimo Locale} è una soluzione $S$ tale che $c(S) \leq c(S')$ per ogni vicino $S' \in N(S)$.
    \item Il problema principale della ricerca locale è che l'algoritmo può rimanere intrappolato in un minimo locale, incapace di "vedere" una soluzione migliore che si trova oltre una "collina" di costi crescenti.
\end{itemize}

\begin{figure}[!ht]
    \centering
    \includegraphics[width=0.3\textwidth]{immagini/LocalSearch/curva2.png}
    \caption{La ricerca locale esplora l'intorno delle soluzioni per trovare minimi locali e globali.} 
    \label{fig:local_vs_global_minimum}
\end{figure}

\noindent
Ad ogni iterazione, l'algoritmo di ricerca mantiene una soluzione corrente $S \in C$. Ad ogni step, sceglie un vicino $S'$ di $S$, dichiara $S'$ come nuova soluzione corrente se $c(S') \leq c(S)$, e itera. Durante l'esecuzione dell'algoritmo, ricorda la soluzione a costo minimo che ha visto finora, $S^*$; quindi, man mano che procede, trova soluzioni sempre migliori e aggiorna $S^*$ di conseguenza. L'algoritmo termina quando non riesce più a trovare un vicino$S'$ migliore della soluzione corrente $S$, restituendo $S^*$ come soluzione finale.

Il \emph{punto cruciale} di un algoritmo di ricerca locale risiede nella scelta della \textbf{neighbor relation} (relazione di vicinato) e nella progettazione della regola per scegliere una soluzione vicina ad ogni passo. Da un lato, la neighbor relation deve essere sufficientemente ampia da permettere all'algoritmo di uscire da minimi locali indesiderati; dall'altro, deve essere sufficientemente ristretta da mantenere l'efficienza computazionale dell'algoritmo.

A differenza dell'approccio Greedy, la ricerca locale permette di rivalutare le scelte fatte in precedenza (come già detto, si parte da una soluzione iniziale completa ma non necessariamente buona) e di esplorare soluzioni alternative. Questo rende la ricerca locale particolarmente adatta per problemi complessi dove le soluzioni ottimali sono difficili da trovare direttamente. 
Il vantaggio dunque è che con Greedy si può facilmente incappare in qualche minimo locale senza possibilità di uscirne, mentre con la ricerca locale si ha la possibilità di esplorare l'intorno delle soluzioni e potenzialmente trovare soluzioni migliori.

Va detto però che un algoritmo di ricerca locale che sia efficiente non esiste per tutti i problemi di ottimizzazione.

\clearpage
\section{Vertex Cover}



\cleardoublepage
\chapter{Graphs}
\label{cap:Graphs}

Un \textbf{grafo} è una struttura matematica che può essere utilizzata per rappresentare un insieme di relazioni binarie tra coppie di oggetti in una collezione: gli oggetti sono chiamati \textbf{vertici} (o nodi) e le relazioni tra di essi sono chiamate \textbf{archi}. 

\paragraph{Definizione:} Un grafo $G$ è una coppia ordinata $(V,E)$ dove $V = \{v_1, v_2, \ldots, v_n\}$ è un insieme non vuoto di vertici e $E = \{u, v\}$ è una collezione di coppie di vertici (archi) di $V$. 

\noindent
Un arco può essere \emph{orientato} o \emph{non orientato}. \begin{itemize}[nosep]
    \item Un arco $(u,v)$ è orientato se la coppia $(u,v)$ è ordinata, con $u$ che precede $v$. Il primo vertice, $u$, è chiamato \emph{sorgente} e il secondo vertice, $v$, è chiamato \emph{destinazione}. Rappresenta una relazione unidirezionale (asimmetrica) da $u$ a $v$.
    \item Un arco $(u,v)$ è non orientato se la coppia $(u,v)$ non è ordinata. Entrambi i vertici sono chiamati \emph{estremi} dell'arco. Rappresenta una relazione bidirezionale (simmetrica) tra $u$ e $v$.
\end{itemize}

\noindent
Se tutti gli archi di un grafo sono orientati, allora si dice che il grafo è un \emph{grafo orientato}. Allo stesso modo, un \emph{grafo non orientato} è un grafo i cui archi sono tutti non orientati. Un grafo che ha sia archi orientati che non orientati è chiamato \emph{grafo misto}.

\vspace{2\baselineskip}
\noindent
La teoria dei grafi rappresenta una pietra miliare dell'informatica teorica e applicata, offrendo un linguaggio formale per modellare le relazioni tra oggetti. Questa struttura è onnipresente: dai \textit{social network}, dove i nodi rappresentano gli utenti e gli archi le amicizie, alle reti di calcolatori e ai sistemi di navigazione satellitare. Algoritmi fondamentali, come quello di \textbf{Dijkstra} per il calcolo del cammino minimo o l'algoritmo \textbf{PageRank} di Google, si basano interamente sulle proprietà topologiche dei grafi. Senza l'astrazione fornita dai grafi, la risoluzione efficiente di problemi complessi di connettività, flusso e ottimizzazione sarebbe pressoché impossibile.



\clearpage
\paragraph{Terminologia sui Grafi:}
Con riferimento al grafo mostrato in Figura \ref{fig:term_graph}:
\begin{itemize}[nosep]
    \item \textbf{Estremi di un arco}: Gli estremi di un arco sono i vertici collegati dallo stesso arco. $u$ e $v$ sono gli estremi dell'arco $a$. 
    \item \textbf{Vertici adiacenti}: Due vertici sono adiacenti se esiste un arco che li collega. $u$ e $v$ sono vertici adiacenti. 
    \item \textbf{Archi incidenti in un vertice}: Gli archi incidenti in un vertice sono gli archi che hanno quel vertice come estremo. $a$, $b$, $d$ sono archi incidenti in $v$. 
    \item \textbf{Grado di un vertice}: Il grado di un vertice $v$ è il numero di archi incidenti in $v$. $x$ ha grado 5.
    \item \textbf{Archi paralleli}: Due archi sono paralleli se collegano gli stessi vertici. $h$ e $i$ sono archi paralleli.
\end{itemize}

\begin{figure}[!ht]
    \centering
    \includegraphics[width=0.4\textwidth]{immagini/Graphs/term_graph.png}
    \caption{}
    \label{fig:term_graph}
\end{figure}


\paragraph{Terminologia sui Grafi Orientati:}
Con riferimento al grafo mostrato in Figura \ref{fig:term_digraph}:
\begin{itemize}[nosep]
    \item \textbf{Archi entranti in un vertice}: Gli archi entranti in un vertice sono gli archi che hanno quel vertice come destinazione. $b$, $e$, $h$ sono archi entranti in $x$.
    \item \textbf{Archi uscenti da un vertice}: Gli archi uscenti da un vertice sono gli archi che hanno quel vertice come sorgente. $g$, $i$, sono archi uscenti da $x$.
    \item \textbf{In-degree}: L'in-degree di un vertice è il numero di archi entranti in esso. $x$ ha in-degree 3.
    \item \textbf{Out-degree}: L'out-degree di un vertice è il numero di archi uscenti da esso. $vx$ ha out-degree 3.
\end{itemize}

\begin{figure}[!ht]
    \centering
    \includegraphics[width=0.4\textwidth]{immagini/Graphs/term_digraph.png}
    \caption{}
    \label{fig:term_digraph}
\end{figure}


\clearpage
\noindent
Un \textbf{cammino} (o percorso, path) in un grafo è una sequenza di vertici collegati da archi, ovvero una sequenza di vertici $(v_1, v_2, \ldots, v_k)$ tale che $(v_i, v_{i+1})$ è un arco del grafo, per $i = 1, 2, \ldots, k-1$. 
È anche possibile vedere un cammino come una sequenza di archi $(e_1, e_2, \ldots, e_{k-1})$ tale che l'estremo di destinazione di $e_i$ è l'estremo sorgente di $e_{i+1}$, per $i = 1, 2, \ldots, k-1$.
Un \textbf{cammino semplice} non visita uno stesso vertice più di una volta. 


Un \textbf{cammino orientato} in un grafo orientato è una sequenza di vertici $(v_1, v_2, \ldots, v_k)$ tale che ogni arco $(v_i, v_{i+1})$ è un arco orientato del grafo, per $i = 1, 2, \ldots, k-1$.

\begin{figure}[!ht]
    \centering
    % Inizio prima figura (sinistra)
    \begin{minipage}[t]{0.4\textwidth}
        \centering
        \includegraphics[width=\linewidth]{immagini/Graphs/term_paths.png}
        \caption{$P_1 = (v, x, z)$ è un cammino semplice; $P_2 = (u, w, x, y, w, v)$ è un cammino ma non semplice.}
        \label{fig:term_paths}
    \end{minipage}
    \hfill % Spazio flessibile tra le due figure
    % Inizio seconda figura (destra)
    \begin{minipage}[t]{0.4\textwidth}
        \centering
        \includegraphics[width=\linewidth]{immagini/Graphs/term_paths1.png}
        \caption{$P_1 = (v, x, z)$ è un cammino orientato semplice; $P_2 = (u, w, x, y, w, v)$ è un cammino orientato ma non semplice.}
        \label{fig:term_paths1}
    \end{minipage}
\end{figure}


\noindent
Un \textbf{ciclo} in un grafo è un cammino che inizia e termina nello stesso vertice, con almeno un arco. Un \textbf{ciclo semplice} non visita uno stesso vertice più di una volta, ad eccezione del vertice iniziale/finale. Un \textbf{auto-ciclo} è un ciclo che consiste in un singolo arco che collega un vertice a se stesso.
\begin{figure}[!ht]
    \centering
    \includegraphics[width=0.4\textwidth]{immagini/Graphs/term_loops.png}
    \caption{$C_1 = (v, x, y, w, u, v)$ è un ciclo semplice; $C_2 = (u, w, x, y, w, v, u)$ è un ciclo ma non semplice.}
    \label{fig:term_loops}
\end{figure}


\clearpage
\section*{Proprietà di un Grafo non orientato}
Siano $n$ il numero di vertici e $m$ il numero di archi di un grafo non orientato (Figura \ref{fig:prop_graph}).
\begin{enumerate}
    \item $\sum\limits_{v \in V} \text{deg}(v) = 2m$, dove la somma è calcolata su tutti i vertici $v$ del grafo. 
    \begin{itemize}
        \item Questo perché ogni arco contribuisce a incrementare il grado di due vertici.
    \end{itemize}
    \item In un grafo non orientato senza archi paralleli e auto-cicli, il numero massimo di archi è dato da $m \le \frac{n(n-1)}{2}$. 
    \begin{itemize}
        \item Per via delle ipotesi di cui sopra, ciascun nodo può avere grado massimo $(\le) n-1$. Riprendendo la proprietà precedente, si ha quindi che $2m \le n(n-1)$.
    \end{itemize}
\end{enumerate}

\vspace{1\baselineskip}
\begin{figure}[!ht]
    \centering
    % Inizio prima figura (sinistra)
    \begin{minipage}[t]{0.4\textwidth}
        \centering
        \includegraphics[width=\linewidth]{immagini/Graphs/prop_graph.png}
        \caption{}
        \label{fig:prop_graph}
    \end{minipage}
    \hfill % Spazio flessibile tra le due figure
    % Inizio seconda figura (destra)
    \begin{minipage}[t]{0.4\textwidth}
        \centering
        \includegraphics[width=\linewidth]{immagini/Graphs/prop_digraph.png}
        \caption{}
        \label{fig:prop_digraph}
    \end{minipage}
\end{figure}


\section*{Proprietà di un Grafo orientato}
Siano $n$ il numero di vertici e $m$ il numero di archi di un grafo orientato (Figura \ref{fig:prop_digraph}).
\begin{enumerate}
    \item $\sum\limits_{v \in V} \text{in-deg}(v) = \sum\limits_{v \in V} \text{out-deg}(v) = m$, dove la somma è calcolata su tutti i vertici del grafo.
    \begin{itemize}
        \item Questo perché ogni arco contribuisce di uno all'in-degree di un vertice e di uno all'out-degree di un altro vertice.
    \end{itemize}
    \item In un grafo non orientato senza archi paralleli e auto-cicli, il numero massimo di archi è dato da $m \le n(n-1)$. 
    \begin{itemize}
        \item Per via delle ipotesi di cui sopra, ciascun nodo può avere grado massimo $(\le) n-1$. Riprendendo la proprietà precedente, si ha quindi che $m \le n(n-1)$.
    \end{itemize}
\end{enumerate}


\clearpage
\noindent
Due vertici $u,v$ di un grafo (orientato) sono \textbf{connessi} se esiste un cammino (orientato) da $u$ a $v$. In un grafo non orientato, la \emph{connettività} è una relazione simmetrica. In un grafo orientato, invece, la connettività non è necessariamente simmetrica: potrebbe esistere un cammino da $u$ a $v$ ma non da $v$ a $u$.

Un grafo si dice \textbf{connesso} se, per ogni coppia di vertici, esiste un cammino tra di essi. Un grafo orientato è \textbf{fortemente connesso} se per ogni coppia di vertici $u,v$, $u$ raggiunge $v$ e $v$ raggiunge $u$.

\begin{figure}[!ht]
    \centering
    \includegraphics[width=0.6\textwidth]{immagini/Graphs/connectivity.png}
    \caption{Esempi di grafi connessi e non connessi.}
    \label{fig:connectivity}
\end{figure}

\vspace{1\baselineskip}
\noindent
Una \textbf{forest} è un grafo non orientato e aciclico (ovvero privo di cicli). Un \textbf{tree} è una forest connessa. In altre parole, un tree è un grafo non orientato, aciclico e connesso.


\begin{figure}[!ht]
    \centering
    \includegraphics[width=0.6\textwidth]{immagini/Graphs/tree_forest.png}
    \caption{Esempi di tree e forest.}
    \label{fig:tree_forest}
\end{figure}

\noindent
Un \textbf{sottografo} $H$ di un grafo $G = (V,E)$ è un grafo $H = (V', E')$ tale che $V' \subseteq V$ e $E' \subseteq E$. In altre parole, un sottografo è ottenuto rimuovendo vertici e/o archi da $G$. Uno \textbf{spanning subgraph} (la traduzione corretta è "sottografo ricoprente") di un grafo $G = (V,E)$ è un sottografo $H = (V', E')$ tale che $V' = V$. In altre parole, uno spanning subgraph contiene tutti i vertici di $G$, ma potrebbe non contenere tutti gli archi di $G$. 

Uno \textbf{spanning tree} di un grafo è uno spanning subgraph che è anche un tree (ricordiamo, tree = grafo non orientato, aciclico e connesso).


\vspace{1\baselineskip}
\noindent
Dato un grafo non orientato $G$ con $n$ vertici e $m$ archi:
\begin{itemize}[nosep]
    \item Se $G$ è connesso, allora $m \ge n-1$.
    \item Se $G$ è una forest, allora $m \le n-1$.
    \item Se $G$ è un tree, allora $m = n-1$.
\end{itemize}



\clearpage
\section{Rappresentazioni dell'ADT Grafo}
Esistono diverse rappresentazioni possibili di un grafo. In ognuna di esse, mantieniamo una collezione per memorizzare i vertici di un grafo. Tuttavia, le rappresentazioni differiscono notevolmente nel modo in cui organizzano gli archi.

\begin{itemize}
    \item \textbf{Edge List}: in un edge list, manteniamo una lista non ordinata di tutti gli archi. Questo è il minimo indispensabile, ma non esiste un modo efficiente per localizzare un particolare arco (u,v), o l'insieme di tutti gli archi incidenti a un vertice v.
    \item \textbf{Adjacency List}: in un adjacency list, manteniamo, per ogni vertice, una lista separata contenente quegli archi che sono incidenti al vertice. L'insieme completo degli archi può essere determinato prendendo l'unione dei set più piccoli, mentre l'organizzazione consente di trovare in modo più efficiente tutti gli archi incidenti a un dato vertice.
    \item \textbf{Adjacency Map}: un adjacency map è molto simile a un adjacency list, ma il contenitore secondario di tutti gli archi incidenti a un vertice è organizzato come una mappa, anziché come una lista, con il vertice adiacente che funge da chiave. Ciò consente l'accesso a un arco specifico (u,v) in tempo O(1) atteso.
    \item \textbf{Adjacency Matrix}: un adjacency matrix fornisce un accesso nel caso peggiore O(1) a un arco specifico (u,v) mantenendo una matrice n×n, per un grafo con n vertici. Ogni voce è dedicata a memorizzare un riferimento all'arco (u,v) per una particolare coppia di vertici u e v; se non esiste tale arco, la voce sarà None.
\end{itemize}

\vspace{1\baselineskip}
\begin{figure}[!ht]
    \centering
    \includegraphics[width=1\textwidth]{immagini/Graphs/adt_graph.png}
    \caption{Un riepilogo dei tempi di esecuzione per i metodi dell'ADT grafo, utilizzando le rappresentazioni del grafo discusse in questa sezione. Indichiamo con $n$ il numero di vertici, $m$ il numero di archi, e $d_v$ il grado del vertice $v$. Si noti che la matrice di adiacenza utilizza uno spazio $O(n^2)$, mentre tutte le altre strutture utilizzano uno spazio $O(n+m)$.}
    \label{fig:adt_graph}
\end{figure}


\clearpage
\section{Rappresentazione mediante Edge List}
L'\emph{edge list} è la rappresentazione più semplice possibile di un grafo, anche se non la più efficiente. Tutti gli oggetti vertice sono memorizzati in una lista non ordinata $V$, e tutti gli oggetti arco sono memorizzati in una lista non ordinata $E$.


\begin{figure}[!ht]
    \centering
    \includegraphics[width=0.75\textwidth]{immagini/Graphs/edge_list.png}
    \caption{(a) Un grafo $G$; (b) rappresentazione schematica della struttura edge list per $G$. Si noti che un oggetto arco fa riferimento ai due oggetti vertice che corrispondono ai suoi estremi, ma che i vertici non fanno riferimento agli archi incidenti per cui non mantengono questa informazione.}
    \label{fig:edge_list}
\end{figure}

\noindent
Per supportare i metodi dell'ADT Grafo, assumiamo le seguenti caratteristiche aggiuntive di una rappresentazione edge list. Le collezioni $V$ ed $E$ sono rappresentate come doubly linked list.

\begin{itemize}[nosep]
    \item Oggetto \textbf{Vertex}: rappresenta un vertice del grafo. Mantiene un riferimento al suo elemento (ad esempio, una stringa o un numero) e un riferimento alla sua posizione nella lista $V$ del grafo.
    \item Oggetto \textbf{Edge}: rappresenta un arco del grafo. Mantiene un riferimento al suo elemento (ad esempio, un peso o un'etichetta) e riferimenti ai due vertici estremi dell'arco. Inoltre, mantiene un riferimento alla sua posizione nella lista $E$ del grafo.
\end{itemize}

\begin{figure}[!ht]
    \centering
    \includegraphics[width=0.8\textwidth]{immagini/Graphs/summary_edgeList.png}
    \caption{Riepilogo tempi di esecuzione l'ADT Grafo con rappresentazione Edge List.}
    \label{fig:summary_edgeList}
\end{figure}








\cleardoublepage
\chapter{Graphs Traversal}
\label{cap:GraphsTraversal}



\end{document}